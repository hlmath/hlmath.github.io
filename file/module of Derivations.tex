\documentclass[UTF8]{article}
\usepackage{ctex}
\usepackage[colorlinks=true]{hyperref}
\title{\textbf{\huge{微分模}}}
\author{Lhzsl}
\date{}
\usepackage[b5paper,left=10mm,right=10mm,top=15mm,bottom=15mm]{geometry}
\usepackage{amsthm,amsmath,amssymb}
\usepackage{mathrsfs}
\usepackage{tikz}
\usepackage[all]{xy}
\usetikzlibrary{cd}
\usepackage{fancyhdr}
\usepackage{color}
\newtheorem{thm}{Theorem}[section]
\newtheorem{defn}{Definition}[section]
\newtheorem{cor}{Corollary}[section]
\newtheorem{prop}{Proposition}[section]
\newtheorem{exa}{Example}[section]
\newtheorem{lem}{Lemma}[section]
\newtheorem{Rem}{Remark}[section]
\begin{document}
	\maketitle
	设$A$是环,$M$是一个$A-$模,从$A$到$M$的一个\textbf{导数}是指一个映射
	$D:A\rightarrow M$满足:
	$$D(a+b)=D(a)+D(b),D(ab)=aD(b)+bD(a)$$
	对任意$a,b\in A$成立。
	所有这样的导数组成的集合记为$Der(A,M)$。规定$(D+D')a=Da+D'a,(aD)b=a(Db),\forall a,b \in A,$则$Der(A,M)$具有$A-$模结构。
	
	若$A$透过映射$f:k\rightarrow A$成为$k-$代数,若导数$D\in Der(A,M)$满足$D\circ f=0$,则称$D$为$k-$导数,所有这些$k-$导数组成的集合记为$Der_{k}(A,M)$,它是$Der(A,M)$的$A-$子模。对任意
	$D\in $ \\
	$Der(A,M)$,易知
	$D(1)=D(1)+D(1)$,从而$D(1)=0,$由此,若将$A$看作$\mathbb{Z}-$模,则$Der(A,M)=Der_{\mathbb{Z}}(A,M).$
	
	特别地,若$M=A$,则简记$Der_{k}(A,A)$为$Der_{k}(A).$
	
	设$A$是一个$k-$代数,$N$为$A-$模,则直和$A\oplus N$具有$k-$模结构,定义$A\oplus N$中元素的乘法为:
	$$
	(a,x)(a',x')=(aa',ax'+a'x),\forall a,a'\in A,x,x'\in N.
	$$
	则$A\oplus N$成为$k-$代数,用$A*N$表示该$k-$代数。
	
	
	一般地,给定$k-$代数组成的范畴中交换图
	$$
	\xymatrix{
B\ar[r]^{f}&A\\
&C\ar[lu]^{h}\ar[u]^{g}	
}
	$$
这里将$g$看作固定的,称$h$是$g$的一个\textbf{提升}。用$N$表示$B$的理想$Kerf$,若$h':C\rightarrow B$是$g$的另一个提升,则$(h-h')(C)\subseteq N$,从而$h-h'$可看作从$C$到$N$的映射。若$N^2=0$,则$N$是一个$f(B)\cong B/N-$模:若$a\in f(B)$,取$a$的一个原像$b$,对于任意$n\in N$,定义$an:=bn$,由于$N^2=0$,该定义与$a$的原像选取无关,从而是良好定义的。
进一步由映射$g:C\rightarrow f(B)\subseteq A$,$N$可看作$C-$模。

断言:$h-h':C\rightarrow N$是$C$到$C-$模$N$的一个$k-$导数。
$$
\xymatrix{
N\ar@{^{(}->}[rr]& &B\ar[rr]^{f}&&A\\
& &&&C\ar[ullll]^{h-h'}\ar[ull]_{h,h'}\ar[u]_{g}
}
$$
\begin{proof}
由于$h,h':C\rightarrow B$均为$k-$代数范畴里的态射,故自然有$h\circ (k\rightarrow C)=k\rightarrow B=h'\circ (k\rightarrow C)$,即$(h-h')\circ (k\rightarrow C)=0.$
剩下便只需证明:$(h-h')(ab)=a(h-h')b+b(h-h')a$对任意$a,b\in C$成立。
这里只需注意$C$中元素$a,b$是如何作用在$N$上。用上述作用的定义,$a$的作用,即相当于取$g(a)$在$f$下的任意一个原像去作用,对$b$同样是。于是
\[
\begin{split}
&a(h-h')b+b(h-h')a\\
&=h(a)(h-h')b+h'(b)(h-h')(a)\\
&=h(ab)-h'(ab)\\
&=(h-h')(ab).
\end{split}
\]
上述式中,$h(a),h'(b)$分别为$g(a),g(b)$在$f$下的一个原像。由上式看出,$h-h'$确实为$C$到$C-$模$N$的一个$k-$导数。
\end{proof}
设$k$是环,$A$为$k-$代数,用$\mathscr{M}_{A}$表示$A-$范畴,则$M\mapsto Der_{k}(A,M)$为$\mathscr{M}_{A}$到自身的协变函子。下面说明该函子为可表函子.
即存在$A-$模$M_{0}$及导数$d\in Der_{k}(A,M_{0})$满足下面泛性质:\\
对任意$A-$模$M$,
导数$D\in Der_{k}(A,M)$,存在唯一的$A-$线性映射$f:M_{0}\rightarrow M$,使得$D=f\circ d$.即有下交换图
$$
\xymatrix{
& &M_{0}\ar@{.>}[dd]^{\exists !f}\\
A\ar[rru]^{d}\ar[rrd]_{D}& &\\
&&M
}
$$

\begin{proof}
	证明是构造性的。定义$\mu:A\otimes_{k}A\rightarrow A$为$\mu(x\otimes y)=xy$;则$\mu$是$k-$代数同态。令
	$$I=Ker\mu,\ \ \Omega_{A/k}=I/I^2,\ \ B=(A\otimes_{k}A)/I^2;$$
	则$\mu$诱导处出同态$\mu':B\rightarrow A$且有下述$k-$代数正合列
	$$
	0\rightarrow \Omega_{A/k}\rightarrow B\stackrel{\mu'}{\rightarrow}A\rightarrow 0
	$$
	该正合列是分裂的(split)。事实上,定义映射$\lambda_{i}:A\rightarrow B,i=1,2$为
	$$
	\lambda_{1}(a)=a\otimes 1\ mod \ I^2\ ,\lambda_{2}(a)=1\otimes a\ mod \ I^2,
	$$
	这两个映射均为$1_{A}:A\rightarrow A$的提升。
	$$
	\xymatrix{
	I/I^{2}\ar@{^{(}->}[rr]& &(A\otimes_{k}A)/I^{2}\ar[rr]^{\mu'}&&A\\
	& &&&A\ar[ullll]^{d=\lambda_{1}-\lambda_{2}}\ar[ull]_{\lambda_{1},\lambda_{2}}\ar[u]_{id_{A}}
}
	$$
	故$d=\lambda_{2}-\lambda_{1}$是$A$到$\Omega_{A/k}$的一个导数。下面说明$(\Omega_{A/k},d)$满足前面所说的泛性质。
	
	若$D\in Der_{k}(A,M)$,定义$\varphi:A\otimes_{k} A\rightarrow A*M$为
	$\varphi(x\otimes y)=(xy,xDy)$,可验证$\varphi$保持乘法,将$\varphi$ $k-$线性扩充到整个$A\otimes_{k}A$上,则得到$k-$代数同态$\varphi$,
	注意到$$
	\mu(\sum x_{i}\otimes y_{i})=\sum x_{i} y_{i}=0\Rightarrow \varphi(\sum x_{i}\otimes y_{i})=(0,\sum x_{i}y_{i});
	$$
	故$\varphi$将$I$映到$M$(此处将$M$看作$A*M$的一个子代数),由$A*M$中乘法定义知,在$A*M$中$M^2=0$,故我们便得到映射$f:I/I^2\rightarrow \Omega_{A/k}\rightarrow M$.
	$$
	\xymatrix{
A\ar[r]^{d}\ar[rdd]&\Omega_{A/k}\ar@{.>}[d]\ar@{^{(}->}[r]&A\otimes_{k} A\ar[ld]^{\varphi}\\
	&A*M\ar[d]^{\pi}& \\
	&M&
}
	$$
	其中$\pi$为$A*M$到第二个分量的投射,而$f$即为竖直方向上两个映射的合成。
	
	对于$a\in A$,
	\[
	\begin{split}
	f(da)&=f(1\otimes a-a\otimes 1\ mod \ I^2 )=\varphi(1\otimes a)-\varphi(a\otimes 1)\\
	&=Da-a\cdot D(1)=Da,
	\end{split}
	\]
	于是$D=f\circ d$.
	
	定义$A$在$\Omega_{A/k}$上的作用为$a\in A$作用在$A\otimes_{k} A$上即为$1\otimes 1$乘以$A\otimes_{k}A$中元素(等价地,用$1\otimes a$去乘:$a\otimes 1-1\otimes a\in I$,$I$中元素作用在$\Omega_{A/k}$上为零),由此$\Omega_{A/k}$具有$A-$模结构。若$\xi=\sum x_{i}\otimes y_{i}\ mod \ I^2\in \Omega_{A/k}$,则$a\xi=\sum ax_{i}\otimes y_{i}\ mod \ I^2$,因此
	$f(a\xi)=\sum ax_{i}Dy_{i}=af(\xi)$,故$f$是$A-$线性的。
	
	对于$a,a'\in A$,
	$a\otimes a'=(a\otimes 1)(1\otimes a'-a'\otimes 1)+aa'\otimes 1,$
	于是,若$\omega=\sum x_{i}\otimes y_{i}\in I$,则
	$$\omega= \sum (x_{i}\otimes 1)(1\otimes y_{i}-y_{i}\otimes 1)+x_{i}y_{i}\otimes 1$$
	$$
	=\sum x_{i}dy_{i}+(\sum x_{i}y_{i}\otimes 1)
	=\sum x_{i}dy_{i}.
	$$
	故$\Omega_{A/k}$作为$A-$模由$\{da|a\in A\}$生成。
	满足$D=f\circ d$的$A-$线性映射$f$的唯一性是显然的。
\end{proof}
称上面构造的$A-$模$\Omega_{A/k}$为$A$在$k$上的\textbf{微分模(module of differentials)}或称\textbf{K$\ddot{a}$hler differentials}.称$da\in \Omega_{A/k}$为$a\in A$的微分(differential).用$d_{A/k}$表示$d:A\rightarrow\Omega_{A/k}$.从定义知有同构
$$
 Hom_{A}(\Omega_{A/k},M)\cong Der_{k}(A,M)
$$
$$
f\mapsto f\circ d_{A/k}.
$$
设$A$为$k-$代数,若$A$具有下列泛性质:
对任意$k-$代数$C$,$C$的理想$N$,$N^2=0,$及任意$k-$代数同态$u:A\rightarrow C/N$,存在$u$的提升$v:A\rightarrow C$,$v$是$k-$代数同态。换句话说,若有交换图
$$
\xymatrix{
A\ar[r]^{u}&C/N\\
k\ar[u]\ar[r]&C\ar[u]
}
$$
则存在$v$使得下图交换
$$
\xymatrix{
A\ar[rd]^{v}\ar[r]^{u}&C/N\\
k\ar[u]\ar[r]&C\ar[u],
}
$$
则称$A$为\textbf{0-smooth}(over k).若至多存在一个$v$,则称$A$为\textbf{0-unramified} over k。若$A$同时为0-smooth,0-unramified ,则称$A$为\textbf{0-etale}.

  $\Omega_{A/k}=0\Leftrightarrow A$是0-unramified.

$\Rightarrow )$若$u$存在两个提升$v_{1},v_{2}$,由前面证明过程知,$d=v_{1}-v_{2}:A\rightarrow N$是导数,从而存在$f\in Hom_{A}(\Omega_{A/k},N)$使得$v_{1}-v_{2}=f\circ d_{A/k}$,但此时$d_{A/k}=0,$故$v_{1}-v_{2}=0$,即$v_{1}=v_{2}$,矛盾!

$\Leftarrow )$
利用$\Omega_{A/k}$的构造过程,取$C=(A\otimes_{k}A)/I^{2},N=I/I^{2}$,这里$I=Ker(A\otimes_{k}A\rightarrow A)$,
由于$A$是$0-unramified $,故上面构造的$\lambda_{1}=\lambda_{2}$,从而$d_{A/k}=0,$于是$\Omega_{A/k}=0.$
\begin{thm}
	设有环同态$k\stackrel{f}{\rightarrow }A\stackrel{g}{\rightarrow}B$,则有$B-$模正合列
	$$
\Omega_{A/k}\otimes_{A}B\stackrel{\alpha}{\rightarrow}\Omega_{B/k}\stackrel{\beta}{\rightarrow}\Omega_{B/A}\rightarrow 0,
	$$
	这里$\alpha,\beta$分别为$\alpha(d_{A/k}a\otimes b)=bd_{B/k}g(a),\beta(d_{B/k}b)=d_{B/A}b,$其中$a\in A,b\in B$.进一步,若$B$在$A$上是0-smooth,则有分裂正合列
	$$
	0\rightarrow \Omega_{A/k}\otimes_{A}B\stackrel{\alpha}{\rightarrow}\Omega_{B/k}\stackrel{\beta}{\rightarrow}\Omega_{B/A}\rightarrow 0.
	$$
\end{thm}
下述证明过程不短,但原理简单。
\begin{proof}
	一般地,想要证明$B-$模序列
	$$
	N'\stackrel{\alpha}{\rightarrow}N\stackrel{\beta}{\rightarrow}N''
	$$
	是正合的,只需证明,对任意$B-$模$T$,诱导序列
	$$
		Hom_{B}(N',T)\stackrel{\alpha^*}{\leftarrow}Hom_{B}(N,T)\stackrel{\beta^*}{\leftarrow}Hom_{B}(N'',T)
	$$
	是正合的。
	事实上,取$T=N''$,则得到$\alpha^{*}\beta^{*}(1_{T})=0,$故$\beta\alpha=0;$取$T=N/Im\alpha$,则可知$Ker\beta =Im\alpha$.
	
	由上分析,我们只需证明
	$$
	0\rightarrow Hom_{B}(\Omega_{B/A},T)\stackrel{\beta^*}{\rightarrow}Hom_{B}(\Omega_{B/k},T)\stackrel{\alpha^*}{\rightarrow}Hom_{B}(\Omega_{A/k}\otimes_{A}B,T)
	$$
	是正合列。
	
	对于$A-$模$M$,$B-$模$T$,我们有同构
	$\Phi:Hom_{B}(M\otimes_{A}B,T)\cong Hom_{A}(M,T)$:任取$\psi\in Hom_{B}(M\otimes_{A}B,T)$,定义$\Phi(\psi)(m):=\psi(m\otimes 1),m\in M$。
	对任意$a\in A,$
	$$\Phi(\psi)(am)=\psi(am\otimes 1)=\psi(m\otimes g(a)1)
	=\psi(g(a)(m\otimes 1))
	$$
	$$=g(a)\psi(m\otimes 1):=a\psi(m\otimes 1)=a\Phi(\psi)(m).$$
	这里只需注意到$B$通过环同态$g:A\rightarrow B$
	成为$A-$模。此即$\Phi(\psi)\in Hom_{A}(M,T)$.
	
	任取$\varphi\in Hom_{A}(M,T)$,定义$\Phi^{-1}(\varphi)(m\otimes x):=x\cdot \varphi(m).$可验证$\Phi^{-1}$与$\Phi$互逆,故由上述同构。
	
	于是$Hom_{B}(\Omega_{A/k}\otimes_{A}B,T)\cong Hom_{A}(\Omega_{A/k},T).$
	利用微分模的典型同态,我们有
	$$
	\xymatrix{
	0\ar[r]&Hom_{B}(\Omega_{B/A},T)\ar[r]^{\beta*}\ar[d]&Hom_{B}(\Omega_{B/k},T)\ar[r]^{\alpha*}\ar[d]&Hom_{B}(\Omega_{A/k}\otimes_{A}B,T)\ar[d]\\
0\ar[r]&Der_{A}(B,T)\ar[r]^{\beta'}&Der_{k}(B,T)\ar[r]^{\alpha'}&Der_{k}(A,T)
}$$
其中竖直方向均为同构,$\alpha',\beta'$定义为使上图交换的映射.$Der_{A}(B,T)$中任意元素可表示为$f\circ d_{B/A}$,其中$f\in Hom_{B}(\Omega_{B/A},T),$由$\beta^*$定义可知
$$\beta'(f\circ d_{A/B})=f\circ \beta \circ d_{B/k}.$$
	对任意$h\circ d_{B/k}\in Der_{k}(B,T),h\in Hom_{B}(\Omega_{B/k},T)$,上面交换图最右侧竖线实际是映射
	$$Hom_{B}(\Omega_{A/k}\otimes_{A}B,T)\rightarrow Hom_{A}(\Omega_{A/k},T)\rightarrow Der_{k}(A,T)$$
	的合成,于是
	$$\alpha'(h\circ d_{B/k})=h\circ \alpha (d_{A/k}(\cdot)\otimes 1).$$
	
首先注意到$\beta\circ \alpha=0.$事实上,由定义,任取$d_{A/k}a\otimes b \in \Omega_{A/k}\otimes_{A}B,\beta\circ\alpha (d_{A/k}a\otimes b)=bd_{B/A}g(a)=0$(注意到$d_{B/A}\circ g=0$)。由此易知$\alpha'\circ \beta'=0.$


下证$
 Ker\alpha'\subseteq Im\beta'.
$设$h\in Hom_{B}(\Omega_{B/k},T)$,且$0=h\circ \alpha(d_{A/k}a\otimes 1)=h\circ d_{B/k}g(a)$对任意$a\in A$成立。于是$(h\circ d_{B/k})\circ g=0$,
易知$h\circ d_{B/k}:B\rightarrow T$是导数,进一步由定义$h\circ d_{B/k}\in Der_{A}(B,T),$
于是存在$f\in Hom_{B}(\Omega_{B/A},T)$,使得$h\circ d_{B/k}=f\circ d_{B/A}=f\circ \beta(d_{B/k})$.
于是
$$
h\circ \alpha(d_{A/k}a\otimes 1)=h\circ d_{B/k}d(a)=f\circ \beta d_{B/k}g(a)=f\circ \beta \circ \alpha(d_{A/k}a\otimes 1),\forall a\in A.
$$
此即$Ker\alpha'\subseteq Im\beta'$.
这便证明了定理中第一个正合列。

现在设$B$在$A$上是0-smooth.取$D\in Der_{k}(A,T)$,考虑如下交换图
$$
\xymatrix{
B\ar[r]^{1_{B}}&B\\
A\ar[u]^{g}\ar[r]^{\varphi}&B*T\ar[u]
}
$$
这里$\varphi(a)=(ga,Da)$.由假设在上图中可添加一映射$h:B\rightarrow B*T$使之仍然交换。可将$h$表示为$h(b)=(b,D'b)$,则$D=D'\circ g,$且由$B*T$中乘法定义知$D':B\rightarrow T$是导数,故有线性映射$\alpha':\Omega_{B/k}\rightarrow T$使得$D'=\alpha'\circ d_{B/k}$.

现取$T$为$\Omega_{A/k}\otimes B$,$D$为$D(a)=d_{A/k}(a)\otimes 1$,则$D=d_{A/k}(\cdot)\otimes 1=D'\circ g=\alpha'\circ d_{B/k}\circ g=\alpha'\circ \alpha(d_{A/k}(\cdot)\otimes 1).$即$\alpha'\circ \alpha=1$.即上面第二个正合列是分裂的。
\end{proof} 
\end{document}