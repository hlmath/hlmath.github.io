\documentclass[UTF8]{article}
\usepackage{ctex}
\usepackage[colorlinks=true]{hyperref}
\title{代数数论笔记(1)}
\author{Lhzsl}
\date{\today }
\usepackage[b5paper,left=10mm,right=10mm,top=15mm,bottom=15mm]{geometry}
\usepackage{amsthm,amsmath,amssymb}
\usepackage{mathrsfs}
\usepackage{fancyhdr}

%我们在 LaTeX 中先把 page style 设为fancy,再设置这个style中的页眉和页脚。但是它默认每章的第一页的page style是plain,需要单独处理。

% 设置 plain style 的属性

\fancypagestyle{plain}%
	
	\fancyhf{} % 清空当前设置
	
	% 设置页眉 (head)
	
	\fancyhead[RE]{\leftmark} % 在偶数页的右侧显示章名
	
	\fancyhead[LO]{\rightmark} % 在奇数页的左侧显示小节名
	
	\fancyhead[LE,RO]{~\thepage~} % 在偶数页的左侧,奇数页的右侧显示页码
	
	% 设置页脚:在每页的右下脚以斜体显示书名
	
	\fancyfoot[RO,RE]{\it }%Typesetting with \LaTeX}
	
	\renewcommand{\headrulewidth}{0.7pt} % 页眉与正文之间的水平线粗细
	
	\renewcommand{\footrulewidth}{0pt}
	
	\pagestyle{fancy} % 选用 fancy style
	
	% 其余同 plain style
	
	\fancyhf{}
	
	\fancyhead[RE]{\leftmark}
	
	\fancyhead[LO]{\rightmark}
	
	\fancyhead[LE,RO]{~\thepage~}
	
	\fancyfoot[RO,RE]{\it}% Typesetting with \LaTeX}
	
	\renewcommand{\headrulewidth}{0.7pt}
	
	\renewcommand{\footrulewidth}{0pt}
\begin{document}
	\maketitle
	\tableofcontents
\section{代数整数}
	\subsection{整性}
	一个\textbf{代数数域}K是有理数域Q的有限次扩张,K中的元素叫做\textbf{代数数} 。代数数叫做整的,如果它是一个首一整系数多项式$f(x)\in Z[x]$的零点。\\
	由于整性出现在代数的很多方面,下面定义更一般的定义。下面谈到环是总是指带有单位元1的交换环。\\
	\textbf{定义}:$A\subseteq B$是环扩张。$b \in B$叫做在A上整的,如果b是一首一方程$$x^n+a_1 x^{n-1}+\cdots +a_n=0,n \geq 1$$
	的零点,其中$a_i \in A$.如果B的所有元素都在A上是整的,那么称B在A上是整的。\\
	立即有问题产生,那就是:两个在A上是整的元素的和,积是否在A上也是整的?于是有下面的命题。\\
	\textbf{命题1.1}:有限多个元素$b_1,\cdots,b_n \in B$
	在A上都是整的当且仅当环$A[b_1,\cdots,b_n]$看作A模是有限生成的。\\
	命题的证明部分是线性代数的结果:$A=(a_{ij})$
	是任意环上的r阶矩阵,$A^{*}$是A的伴随矩阵,则有$$AA^*=A^*A=det(A)E.$$
	E是r阶单位矩阵,对任何向量$x=(x_1,\cdots ,x_r)$,$$Ax=0\Rightarrow (detA)x=0.$$
	假设$A-A[b_1,\cdots,b_n]$是有限生成的
	,$\omega_1,\cdots,\omega_r$是一组生成基,
	任意$b \in A[b_1,\cdots,b_n]$,
	有$$b\omega_i=\sum_{j=1}^ra_{ij}\omega_j,i=1,\cdots,r,a_{ij}\in A.$$
	于是有$det(bE-(a_{ij}))\omega_i=0,i=1,\cdots,r$,
	由于1能被写成$1=c_1\omega_1+\cdots+c_r\omega_r$,
	所以有$det(bE-(a_{ij}))=0$,
	这就给出了一个系数在A中的首一多项式,b带入为零。由此易推出\\
	\textbf{命题1.2}:$A \subseteq B \subseteq C$是两个环扩张
	,如果C在B上是整的,B在A上是整的,那么C在A上是整的。\\
	下面考虑集合$$\bar{A}=\{b \in B|b\quad integral\quad over\quad A\}$$由命题1.2知这形成了一个环,
	把这个环称作A在B中的整闭包。
	A在B中叫做整闭的,如果$A=\bar{A}$,
	由命题1.2立知$\bar{A}$在B中是整闭的。
	若A是整环,K是A的分式域,A在K中的闭包叫做A的正规化,
	此时如果$A=\bar{A}$,A简单的称为整闭的。\\
	一般情况下,A是一个整环,在其分式域中是整闭的,
	$L|K$是有限次域扩张,B是A在L中的整闭包。
	则每个元素$\beta \in L$可以写成形式$$\beta=\frac{b}{a},b \in B ,a\in A,$$
	事实上,如果$$a_n\beta^n+\cdots +a_1\beta +a_0=0,a_i\in A,a_n\neq 0,$$
	那么由方程$$(a_n\beta)^n+\cdots+a_1^{'}(a_n\beta)+a_0^{'}=0,a_i^{'}\in A$$知 $b=a_n\beta$在A上是整的.\\
	进一步分析,设$\beta \in L$在$A$上是整的,则其极小多项式$P(x)\in A[x].$ 事实上,设$\beta$ 是首一多项式$g(x)\in A[x]$的零点,则在$K[x]$中$p(x)$整除$g(x)$,故$p(x)$的所有零点$\beta_{1},\cdots ,\beta_{n}$
	都在$A$上整闭,因此$p(x)$的系数在$A$上整闭,再由$p(x)\in K[x]$知$p(x) \in A[x]$.由此推出:设A是整闭整环,$K$是其分式域,$f(x)\in A[x]$为首一多项式,则$f(x)$在$K[x]$中的首一因子都在$A[x]$中。\\
	下面引入迹和范数\\
	\textbf{定义:}$x \in L|K$的迹和范数定义为域K上线性空间L中线性变换
	$$T_{x}:L \rightarrow L,\quad T_{x}(\alpha)=x\alpha,$$
	的迹和行列式,即$$Tr_{L|K}(x)=Tr(T_x),\quad N_{L|K}(x)=det(T_x).$$
	设$T_x$的特征多项式为$$
	f_x(t)=det(t*id-T_x)=t^n-a_1 t^{n-1}+\cdots+(-1)^n a_n \in K[t],n=[L:K].$$
	可以看出$a_1=Tr_{L|K}(x)\in K,\quad a_n=N_{L|K}(x)\in K.$
	对任意$x,y \in L$,$$
	Tr_{L|K}(x+y)=Tr(T_{x+y})=Tr(T_x+T_y)=Tr(T_x)+Tr(T_y)=Tr_{L|K}(x)+Tr_{L|K}(y)$$ $$
	N_{L|K}(xy)=det(T_{xy})=det(T_x*T_y)=det(T_x)det(T_y)=N_{L|K}(x)N_{L|K}(y)$$于是有两个同态$$
	Tr_{L|K}:L \rightarrow K ,\quad N_{L|K}:L^{*}\rightarrow K^{*}$$
	在$L|K$是可分扩张时,有下面命题\\
	\textbf{命题1.3:}若$L|K$是可分扩张,$\sigma :L\rightarrow \bar{K}$遍历所有不同的K-嵌入,则有\\
	$(i)\quad f_x(t)=\prod_{\sigma}(t-\sigma x),$\\
	$(ii)\quad Tr_{L|k}(x)=\sum_{\sigma}\sigma x,$\\
	$(iii)\quad N_{L|K}(x)=\prod_{\sigma}\sigma x.$\\
	证明:设x在域K上的极小多项式为
	$$p_{x}(t)=t^m+c_{1}t^{m-1}+\cdots +c_m,\quad m=[K(x):K],$$
	于是$1,x,\cdots,x^{m-1}$是域$K(x)|K$的一组基,若$\alpha_{1},\cdots,\alpha_{d}$是域$L|K(x)$的一组基,则$$
	\alpha_{1},\alpha_{1}x,\cdots ,\alpha_{1}x^{m-1};\cdots;
	\alpha_{d},\alpha_{d}x,\cdots,\alpha_{d}x^{m-1}$$
	是域$L|K$的一组基。容易计算$T_x$在这组基下的矩阵是分块对角矩阵,每个矩阵块特征多项式是$p_{x}(t)$,从而$f_{x}(t)=p_{x}(t)^d$\\
	所有L的K-嵌入的集合$Hom_{K}(L,\bar{K})$被等价关系$$\sigma \sim \tau \Longleftrightarrow \sigma x=\tau x
	$$划分到m个等价类,每个等价类有d个元素,若$\sigma_1,\cdots,\sigma_{m}$是一代表系,则$$p_{x}(t)=\prod_{i=1}^{m}(t-\sigma_{i}x),$$
	$f_{x}(t)=\prod_{i=1}^{m}(t-\sigma_{i}x)^{d}=\prod_{i=1}^{m}\prod_{\sigma \sim \sigma_{i}}(t-\sigma x)=\prod_{\sigma}(t-\sigma x).$同样的可得到(ii),(iii).\\
	可分扩张$L|K$的一组基$ \alpha_{1},\cdots ,\alpha_{n}$的\textbf{判别式}定义为
	$$d(\alpha_{1},\cdots
	 ,\alpha_{n})=det((\sigma_{k}\alpha_{j}))^{2}$$
	$\sigma_{i},i=1.\cdots ,n$遍历K-嵌入$L\rightarrow \bar{K}$.\\
	下面继续讨论域,A是在其分式域中整闭的整环,$L|K$是有限可分域扩张,B是A在K中的闭包,若$x\in B$,则存在首一多项式$f(t)\in A[t]$,使得$f(x)=0$,任意$\sigma$为L到$\bar{K}$的K-嵌入,从而$\sigma f(x)=f(\sigma x)=0$,即$\sigma x$在A上是整的,
	由A是整闭的易得到$A=B\bigcap K$,由命题1.3知$Tr_{L|K}(x),N_{L|K}(x)
    \in K$,又由上分析知$x\in B$时,$\sigma x \in B$,从而$$	Tr_{L|K}(x),N_{L|K}(x)\in A$$
	\textbf{引理1.1} $\alpha_1,\cdots,\alpha_n$是域$L|k$的一组基,且$\alpha_{i}\in B$(由上面分析这是可以做到的),$d=d(\alpha_1,\cdots ,\alpha_n)$,则有
    $$dB\subseteq A\alpha_1+\cdots+A\alpha_n$$
	证明:若$\alpha=a_{1}\alpha_1+\cdots +a_{n}\alpha_{n}\in B,a_{j}\in k,$
	那么$a_{j}$是线性方程组$$
	Tr_{L|K}(\alpha_{i}\alpha)=\sum_{j}Tr_{L|K}(\alpha_{i}\alpha_{j})a_{j}$$的解。由于$Tr_{L|K}(\alpha_{i}\alpha)\in A,
	Tr_{L|K}(\alpha_{i}\alpha_{j})\in A$,因此易知$da_{j}\in A$,从而$$
	d\alpha \in A\alpha_{1}+\cdots+A\alpha_{n}. $$
	一组元素$\omega_{1},\cdots ,\omega_{n} \in B$叫做B在A上的一组\textbf{整基},如果$\forall b \in B$,b 能唯一写成	$\omega_{1},\cdots ,\omega_{n}$的A线性组合(即$b=a_{1}\omega_{1}+\cdots +a_{n}\omega_{n},a_{i}\in A$)。立知这一组元素是$L|K$的一组基,故元素个数总等于域扩张$[L:K]$的度数,整基的存在说明B是一个自由$A-$模,秩为$n=[L:K]$,一般情况下,整基并不存在,然而,若A是主理想整环,则有下面的命题。\\
	\textbf{命题1.4:}若$L|K$是可分扩张,A是主理想整环,那么每个L的有限生成B子模$M\neq 0$(指M,L作为B的模,$B-$模M是$B-$模L的子模,$M\subseteq L$)是秩
	为$[L:K]$的自由$A-$模,特别地,B在A上存在整基。\\
	证明:设$M\neq 0$是L的有限生成B子模,$\alpha_{1},\cdots,\alpha_{n}$
	是$L|K$的一组基,乘以A中的某一元素,可使其全部属于B,故不妨设$\alpha_{i} \in B$,由引理1.1知$dB\subseteq A\alpha_{1}+\cdots +A\alpha_{n}.$故
	$rank(B)\leq [L:K]$,由于$A-$模B的一组生成基也是$K-$模L的一组生成基,于是$rank(B)=[L:K]$,设$\mu_{1},\cdots,\mu_{r}\in M$是$B-$模M的一组生成系,存在$a \in A ,a\neq 0$,使得$a\mu_{i}\in B,i=1,\cdots ,r$,于是
	$aM\subseteq B$.并且$$adM \subseteq dB \subseteq A\alpha_{1}+\cdots +\alpha_{n}=M_{0} .$$
	根据主理想整环上有限生成模的主要定理知$M_{0}$是自由$A-$模,从而$adM,M$ 也是自由$A-$模。进而$$
	[L:K]=rank(B)\leq rank(M)=rank(adM)\leq rank(M_{0})=[L:K],
	$$
	因此,$rank(M)=[L:K].$\\
    考虑$Z\subseteq Q$在代数数域K中的整闭包$\mathcal{O}_{K}\subseteq K$,由命题1.4知K的每个有限生成$\mathcal{O}_{K}-$子模$\textbf{a}$有$Z-$基$\alpha_{1},\cdots,\alpha_{n},$ $$\textbf{a}=\mathbb{Z}\alpha_{1}+\cdots+\mathbb{Z}\alpha_{n}.$$
    基的判别式$$d(\alpha_{1},\cdots,\alpha_{n})=det((\sigma_{i}\alpha_{j}))^2$$与$\mathbb{Z}-$基的选取无关:如果$\alpha_{1}^{'},\cdots,\alpha_{n}^{'}$是另一组基,转换矩阵为$T=(a_{ij}),\alpha_{i}^{'}=\sum_{j}a_{ij}\alpha_{j}$,
    其系数为整数,逆矩阵的系数也是整数,因此行列式为1或-1,于是$$
    d(\alpha_{1}^{'},\cdots,\alpha_{n}^{'})=det(T)^{2}d(\alpha_{1},\cdots,\alpha_{n})=d(\alpha_{1},\cdots,\alpha_{n}).
    $$
    故记$$d(\textbf{a})=d(\alpha_{1},\cdots,\alpha_{n})$$
    特别地
	$\mathcal{O}_{K}$的整基为$\omega_{1},\cdots ,\omega_{n}$,我们得到代数数域K的判别式$$d_{K}=d(\mathcal{O}_{K})=d(\omega_{1},\cdots ,\omega_{n}).$$
    \subsection{理想 }
    \textbf{定理2.1:}环$\mathcal{O}_{K}$是诺特环(即每个理想都是有限生成的),整闭且其中每个非零素理想都是极大理想。\\
    证明:(1)由命题1.4,$\mathcal{O}_{K}$的每个非零理想都是有限生成$Z-$模,因此更是有限生成$\mathcal{O}_{K}-$模。\\
    (2)设$P$是$\mathcal{O}_{K}$的非零素理想。取$0\neq \alpha \in P$,令
    $m=N_{K|Q}(\alpha)\in Z-\{0\}.$由上分析知
    $m/\alpha \in \mathcal{O}_{K}$,于是$m=\alpha * m/\alpha \in P,$
    即$(m)\in P.$设$\omega_{1},\cdots ,\omega_{n}$是$\mathcal{O}_{K}$的一组整基,即$\mathcal{O}_{K}=Z\omega_{1}\oplus \cdots \oplus Z\omega_{n},n=[K:Q].$则$$
    \mathcal{O}_{K}/m\mathcal{O}_{K}=(Z\omega_{1}\oplus \cdots \oplus Z\omega_{n})/Zm\omega_{1}\oplus \cdots \oplus Zm\omega_{n}\\
    \cong Z/mZ\oplus \cdots \oplus Z/mZ
    $$
    由环的同构定理知道$\mathcal{O}_{K}/P\cong (\mathcal{O}_{K}/m\mathcal{O}_{K})/(P/m\mathcal{O}_{K})$,
    从而$|\mathcal{O}_{K}/P|\leq |\mathcal{O}_{K}/m\mathcal{O}_{K}|=m^{n}.$
    即$\mathcal{O}_{K}/P$是有限环,由于$P$是素理想,从而$\mathcal{O}_{K}/P$是有限整环,由熟知的定理(有限整环即是域)知$\mathcal{O}_{K}/P$是域,从而$P$是极大理想。\\
    (3)由于$\mathcal{O}_{K}$是Z在域K中的整闭包,因此$\mathcal{O}_{K}$在K中整闭,更在其分式域中整闭。\\
    \textbf{定义2.2}如果一个整环是诺特的,在其分式域中整闭,且每个素理想都是极大理想,那么称这个环叫做戴德金(Dedekind)整环.\\
    于是下面考虑$\mathcal{O}_{K}$的一般形式,即戴德金环$\mathcal{O}$,有下主要定理\\
    \textbf{定理2.3:}$\mathcal{O}$的除了$(0),(1)$每一个理想$P$都存在(不记次序)唯一的素理想分解$$
    P=P_{1}\cdots P_{r}
    $$
    为证明此定理需要下面的引理\\
    \textbf{引理2.4:}对$\mathcal{O}$的每个非零理想$P$,都存在非零素理想$P_{1},P_{2},\cdots ,P_{r}$,使得$$
    P_{1} P_{2}\cdots P_{r} \subseteq P$$
    证明略(一般代数数论书都会有证明)。\\
    \textbf{引理2.5:}$P$是$\mathcal{O}$的素理想,定义$$
    P^{-1}=\{x \in K|xP \subseteq \mathcal{O}\}.
    $$
    则对于每个非零素理想$Q$有$QP^{-1}:=\{\sum_{i}a_{i}x_{i}|a_{i}\in Q,x_{i}\in P^{-1}\}\neq Q$\\
    证明:设$a\in P,a\neq 0$,且$P_{1}P_{2}\cdots P_{r}\subseteq (a)\subseteq P$,$r$是使其成立的最小正整数,由于$P$是素理想,易推出必有必有$P_{i}$不妨设为$P_{1}\subseteq P$,
    戴德金环中素理想即为极大理想故$P_{1}=P$,由于$P_{2}\cdots P_{n}\nsubseteq (a)$,故存在$b \in P_{2}\cdots P_{n}$使得$b\notin a\mathcal{O}$,i.e,$a^{-1}b\notin \mathcal{O}$,另一方面,有$bP\subseteq (a),i.e,a^{-1}bP\subseteq  \mathcal{O}$,于是$a^{-1}b\in P^{-1}$,
    综上,$P^{-1}\neq \mathcal{O}$\\
    现在设$Q\neq 0$是 $\mathcal{O}$的一个非零理想,$\alpha_{1},\cdots ,\alpha_{n}$是其一组是生成系,假设$QP^{-1}=Q$,则对于每个$x\in P^{-1},$
    $$x\alpha_{i}=\sum_{j}a_{ij}\alpha_{j},a_{ij}\in  \mathcal{O}.$$
    记$A$为矩阵$(x\delta_{ij}-a_{ij})$,于是$A(\alpha_{1},\cdots ,\alpha_{n})^{t}=0$,记$d=det(A)$,则$d\alpha_{1}=\cdots=d\alpha_{n}=0
    $,因此$d=0$,得到$x$在$ \mathcal{O}$上是整的,于是$x\in  \mathcal{O}$
    由于$ \mathcal{O} \subseteq P^{-1}$,
    故推出$P^{-1}= \mathcal{O}$,矛盾!\\
    \textbf{定理2.3的证明:}(1)素理想分解的存在性。
    令$ \mathcal{M}$是所有除$(0),(1)$外不存在素理想分解的理想组成的集合。如果$ \mathcal{M}$非空,由于$ \mathcal{O}$是诺特的,每个理想升链都是有限的,$ \mathcal{O}$中的集合包含关系诱导$ \mathcal{M}$中的一个序,使其成为偏序集。故$ \mathcal{M}$中存在最大的元素,记为$Q$,它包含在一个极大理想$P$中,由于$ \mathcal{O}\subseteq P^{-1}$
    $$Q\subseteq QP^{-1}\subseteq PP^{-1}\subseteq\mathcal{O}.$$
    再由引理2.5,有$Q\varsubsetneq QP^{-1},P\varsubsetneq PP^{-1}\subseteq  \mathcal{O}$.但$P$是极大理想,从而得到$PP^{_1}= \mathcal{O}$,由于$Q$是$ \mathcal{M}$中的最大元,并且$Q\neq P,i.e,QP^{-1}\neq  \mathcal{O}$.得到$QP^{-1}$有素理想分解$QP^{-1}=P_{1}P_{2}\cdots P_{r}$,于是$Q=QP^{-1}P=P_{1}P_{2}\cdots P_{r}P.$矛盾!\\
    (2)唯一性的证明略(类似于整数的素因子分解)。\\
    下面引入分式理想,令$\mathcal{O}$是戴德金整环,$K$为其分式域。\\
    \textbf{定义2.6:}$K$的一个\textbf{分式理想}是$\mathcal{O}-$模$K$的有限生成非零子模\\
    例如,每一元素$a\in K^{*}$定义一个分式“主理想”$(a)=a\mathcal{O}$,
    显然地,由于$\mathcal{O}$是诺特环,一个$\mathcal{O}-$模$K$的子模$\mathcal{O}-$模$Q$是分式理想当且仅当存在$c\in \mathcal{O},c\neq 0$,使得$cQ \subseteq \mathcal{O}$是环$\mathcal{O}$的一个理想(有些书就用这做为分式理想的定义),此后我们把$\mathcal{O}$中的理想叫做$K$中的整理想\\
    \textbf{命题2.7:}分式理想形成Abel群,记为$J_{K}$,单位元是$(1)=\mathcal{O}$,其中元素$Q$的逆元为$$
          Q^{-1}=\{x\in K|xQ \subseteq \mathcal{O} \}
    $$
    证明:元素的结合性,交换性和$Q(1)=Q$都是明显的。对于每个素理想$P$,$P\varsubsetneq PP^{-1}\subseteq  \mathcal{O}$,在戴德金环中素理想都是极大理想,故$PP^{-1}= \mathcal{O}$。再设$Q=P_{1}\cdots P_{r}$是整理想,则
    $B=P_{1}^{-1}\cdots P_{r}^{-1}$是其逆元:$BQ=\mathcal{O}$暗示$B\subseteq Q^{-1}$.反之,如果$xQ \subseteq \mathcal{O}$,则$xQB\subseteq B$,由于$BQ=\mathcal{O}$
    得到$x\in B$,故$B=Q^{-1}$.最后,若$Q$是任意的分式理想,存在$c\in \mathcal{O},c\neq 0$,使得$cQ\subseteq \mathcal{O}$,于是$(cQ)^{-1}=c^{-1}Q^{-1}$是$cQ$的逆元,$QQ^{-1}=\mathcal{O}$\\
    查阅维基百科:分式理想在代数数论和戴德金环论中有不同定义,代数数论中的定义为:\\
    数域K中的非空子集合$I$叫做K中的分式理想,如果存在$\mu\in\mathcal{O}_{K}$使得$\mu I$
    是$\mathcal{O}_{K}$中的理想。用$J_{K}$表示所有K的分式理想组成的集合,可以证明$J_{K}$组成群。令$P_{K}$是K的分式主理想组成的群,后文将证明它们的商群$CL_{K}=J_{K}/P_{K}$为有限群。\\
    \subsection{理想的分解}
    就像整数理论中要研究整数的素数分解一样,代数数论中要研究理想的素理想分解。设$L|K$是数域的扩张,$Q$是$\mathcal{O}_{K}$的一个(整)理想,问题$\mathcal{O}_{L}$中的理想$Q\mathcal{O}_{L}$如何分解成$\mathcal{O}_{L}$中素理想的乘积?由于每个理想$Q$均是$\mathcal{O}_{K}$中一些素理想的乘积,因此我们只要对$\mathcal{O}_{K}$中的每个素理想$P$弄清楚$P\mathcal{O}_{L}$在$\mathcal{O}_{L}$中的素理想分解式就可以。这里事先说明,若L中的素理想$\mathcal{B}$出现在$P\mathcal{O}_{L}$的素理想分解式中,即$\mathcal{B}|P\mathcal{O}_{L}$,那么简记为$\mathcal{B}|P$。下面便是一些命题的证明。\\
    \textbf{引理3.1:}设$L|K$是数域的扩张,$\mathcal{B}$为$\mathcal{O}_{L}$的素理想,则:\\
    (1)$\mathcal{B}\cap \mathcal{O}_{K}$为$\mathcal{O}_{K}$的素理想,并且
    $\mathcal{B}\cap \mathcal{O}_{K}=P \Leftrightarrow \mathcal{B}|P$\\
    (2)若$\mathcal{B}\cap \mathcal{O}_{K}=P$,则$\mathcal{O}_{K}/P$和$\mathcal{O}_{L}/\mathcal{B}$均是有限域,并且前者可看成是后者的子域。\\
    证明:(1)易验证$\mathcal{B}\cap \mathcal{O}_{K}$是$\mathcal{O}_{K}$的理想,从而只需验证为素理想。$\forall a,b\in \mathcal{O}_{K},ab\in \mathcal{B}\cap \mathcal{O}_{K}$,由于$\mathcal{B}$是$\mathcal{O}_{L}$中的素理想,从而$a\in \mathcal{B}$或$b\in\mathcal{B}$,即有$a\in \mathcal{B}\cap \mathcal{O}_{K}$或$b\in \mathcal{B}\cap \mathcal{O}_{K}$.\\
    若$\mathcal{B}\cap \mathcal{O}_{K}=P$,则$P\subseteq \mathcal{B},$从而
    $P\mathcal{O}_{L}\subseteq \mathcal{B}$,于是$\mathcal{B}| P\mathcal{O}_{L}$,即
    $P\mathcal{B}|P$.反之,若P为$\mathcal{O}_{K}$中的素理想,且$\mathcal{B}| P\mathcal{O}_{L}$
    于是$P\subseteq  P\mathcal{O}_{L}\subseteq \mathcal{B},$从
    而$P=P\cap \mathcal{O}_{K}\subseteq \mathcal{B}\cap \mathcal{O}_{K}$,但$\mathcal{O}_{K}\cap\mathcal{B}$均是$\mathcal{O}_{K}$中的素理想,从而是极大理想,所以有$\mathcal{B}\cap \mathcal{O}_{K}=P$\\
    (2)作映射$$
    \phi:\mathcal{O}_{K}\rightarrow \mathcal{O}_{L}/\mathcal{B},\phi(x)=x+\mathcal{B}(x\in \mathcal{O}_{K})
    $$
    易知是环同态。$Ker\phi=\mathcal{O}_{K}\cap \mathcal{B}=P,$从而由同态$\phi$可将环$\mathcal{O}_{K}/P$看成是$\mathcal{O}_{L}/\mathcal{B}$
    的子环,由于$\mathcal{O}_{K}/P$和$\mathcal{O}_{L}/\mathcal{B}$均是有限环(元素个数分别是$N_{K}(P),N_{L}(\mathcal{B})$),由于$P$和$\mathcal{B}$分别是$\mathcal{K}$和$\mathcal{L}$的极大理想,从而于$\mathcal{O}_{K}/P$和$\mathcal{O}_{L}/\mathcal{B}$是有限域,前者是后者的子域。\\
    \subsection{戴德金环的扩张}
    \textbf{命题4.1:} 若$o$是戴德金整环,分式域为$K,$ $L|K$是有限域扩张,$\mathcal{O}$是o在L上的整闭包,那么$\mathcal{O}$也是戴德金整环.\\
    证明:(1)由于是o的整闭包,$\mathcal{O}$自然是整闭的。\\
      (2)设$\mathcal{B}$是$\mathcal{O}$的非零素理想,那么$P=\mathcal{O}\cap \mathcal{B}$是o的非零素理想,可以通过构造映射将整环$\mathcal{O}/\mathcal{B}$看作域$o/P$的扩张。因此这一整环也是域,若不然整环将有一非零素理想,其与域$o/P$的交集是的非零素理想。\\
      (3)若$L|K$是可分扩张,证明是容易的.令$\alpha_{1},\cdots ,\alpha_{n}$是$L|K$的包含在$\mathcal{O}$的一组基,判别式为$d=d(\alpha_{1},\cdots,\alpha_{n})$,那么由前面命题知$d\neq 0,$ $\mathcal{O}$包含在一有限生成$o-$模$o\alpha_{1}/d+\cdots+o\alpha_{n}/d$中。$\mathcal{O}$的每个理想也都包含在这个有限生成$o-$模中,于是$\mathcal{O}$也是有限生成$o-$模,更是有限生成$\mathcal{O}-$模,这就展示了$\mathcal{O}$是诺特的。一般情况后文将给出证明。\\
    %%  对于o的每个理想$P,$总是有$P\mathcal{O}\neq \mathcal{O}.$事实上,令$\pi \in P-P^{2}(P\neq 0)$($\pi $存在是由于理想的素分解的唯一性),
     o中一个素理想在环$\mathcal{O}$中有唯一的素理想分解
     $$
  P\mathcal{O}=\mathcal{B}_{1}^{e_{1}}\cdots  \mathcal{B}_{r}^{e_{r}} 
  $$
我们经常简记$P\mathcal{O}$为$P.$若$\mathcal{B}$是$\mathcal{O}$ 的素理想$P=\mathcal{B}\cap o,$由上面可知$\mathcal{B}|P,$我们称$\mathcal{B}$是$P$的一个素因子,指数$e_{i}$叫做分歧指数,域扩张的度数$f_{i}=[\mathcal{O}/\mathcal{B}:o/P]$叫做$\mathcal{B}$在$P$上的剩余类域次数。\\
\textbf{命题4.2:}若$L|K$是可分扩张,那么有等式
$$
\sum_{i=1}^{r}e_{i}f_{i}=n
$$
证明:由中国剩余定理知,
$$
\mathcal{O}/P\mathcal{O}\cong \oplus_{i=1}^{r}\mathcal{O}/\mathcal{B}_{i}^{e_{i}}.
$$
$\mathcal{O}/P\mathcal{O}$和$\mathcal{O}/\mathcal{B}_{i}^{e_{i}}$是域$o/P$上的向量空间(若无特殊说明,下文总是指这两个线性空间在域$o/P$上的),若证明$$
dim_{o/P}(\mathcal{O}/P\mathcal{O})=n,dim_{o/P}(\mathcal{O}/\mathcal{B}_{i}^{e_{i}})=e_{i}f_{i}
$$即证明了命题。\\
设$\bar{\omega}_{1},\cdots ,\bar{\omega}_{m}$是$\mathcal{O}/P\mathcal{O}$的一组基,$\omega_{1},\cdots,\omega_{m}\in \mathcal{O}$是其代表元(由命题4.1的证明过程知这一向量空间是有限维的)。只需证明$\omega_{1},\cdots,\omega_{m}$是$L|K$的一组基,假设$\omega_{1},\cdots,\omega_{m}$在域K上线性相关,从而在o上也线性相关。从而存在非零元素$a_{1},\cdots,a_{m}\in o$使得$$
a_{1}\omega_{1}+\cdots+a_{m}\omega_{m}=0.
$$
考虑理想$A=(a_{1},\cdots,a_{m})$,取$a\in A^{-1},$使得$a\notin A^{-1}P,$从而$aA$不包含在$P$中,于是$aa_{1},\cdots,aa_{m}$属于o,但不全属于$P,$同余式
$$
aa_{1}\omega_{1}+\cdots+aa_{m}\omega_{m}\equiv 0 (mod p)
$$
说明$\bar{\omega}_{1},\cdots ,\bar{\omega}_{m}$在域$o/P$上是线性相关的,矛盾!因此$\omega_{1},\cdots,\omega_{m}$在域K上线性无关。\\
为证明$\omega_{i}$是$L|K$的一组基,考虑$o-$模$M=o\omega_{1}+\cdots+o\omega_{m},N+\mathcal{O}/M.$由于$\mathcal{O}=M+P\mathcal{O},$我们有$PN=N.$由于$L|K$是可分扩张,由命题4.1证明过程知$\mathcal{O}$是有限生成$o-$模,从而N也是。设$\alpha_{1},\cdots,\alpha_{s}$是N的一个生成系,那么
$$
\alpha_{i}=\sum_{j}a_{ij}\alpha_{j},a_{ij}\in P.
$$
令$A=(a_{ij})-I$,$I$是单位矩阵,B是A的伴随矩阵。从而$A(\alpha_{1},\cdots,\alpha_{s})^{t}=0,BA=dI,d=det(A),$进而
$$
0=BA(\alpha_{1},\cdots,\alpha_{s})^{t}=(d\alpha_{1},\cdots,d\alpha_{s})^{t},
$$
因此$dN=0,$即$d\mathcal{O}\subseteq M=o\omega_{1}+\cdots+o\omega_{m}.$
由于$a_{ij}\in P,$故$d=det((a_{ij})-I)\equiv (-1)^{s}mod p,$即$d\neq 0,$
从而$L=dL\subseteq dK\mathcal{O}\subseteq K\omega_{1}+\cdots+K\omega_{m}.$综上$\omega_{1},\cdots,\omega_{m}$是$L|K$的一组基。\\
下面证明第二个等式。考虑$o/P$上线性空间的递降链
$$
\mathcal{O}/\mathcal{B}_{i}^{e_{i}}\supseteq\mathcal{B}_{i}/\mathcal{B}_{i}^{e_{i}}\supseteq \cdots \supseteq \mathcal{B}_{i}^{e_{i}-1}/\mathcal{B}_{i}^{e_{i}}\supseteq (0).
$$
下面我们证明$\mathcal{B}_{i}^{\nu}/\mathcal{B}_{i}^{\nu+1}$同构于$\mathcal{O}/\mathcal{B}_{i},$若$\alpha \in \mathcal{B}_{i}^{\nu}/\mathcal{B}_{i}^{\nu+1},$映射$$
\mathcal{O}\rightarrow \mathcal{B}_{i}^{\nu}/\mathcal{B}_{i}^{\nu+1},a\mapsto a\alpha ,
$$的核为$\mathcal{B}_{i},$由于$\mathcal{B}_{i}^{\nu}$是$\mathcal{B}_{i}^{\nu+1}$和$(\alpha)=\alpha\mathcal{O}$的最大公因数,因此$\mathcal{B}_{i}^{\nu}=\alpha\mathcal{O}+\mathcal{B}_{i}^{\nu+1},$从而这是满射,因为$f_{i}=[\mathcal{O}/\mathcal{B}:o/P],$我们得到$dim_{o/P}(\mathcal{B}_{i}^{\nu}/\mathcal{B}_{i}^{\nu+1})=f_{i},$进而$$
dim_{o/P}(\mathcal{O}/\mathcal{B}_{i}^{e^{i}})=\sum_{\nu=0}^{e_{i}-1}dim_{o/P}(\mathcal{B}_{i}^{\nu}/\mathcal{B}_{i}^{\nu+1})=e_{i}f_{i}.
$$
设$L|K$是可分扩张,$\theta\in\mathcal{O},$其极小多项式$p(X)\in o[X]$,且有
$L=K(\theta)$,定义环$o[\theta]$的导子(conductor)为
$\mathfrak{F}=\{\alpha\in \mathcal{O}|\alpha\mathcal{O}\subseteq o[\theta]\}.$\\
\textbf{命题4.3:}$P$是和$o[\theta]$的导子$\mathfrak{F}$互素,即是$\mathfrak{p}\mathcal{O}+\mathfrak{F}=\mathcal{O}$的o的素理想,
 令$$
    \bar{p}(X)=\bar{p}_{1}(X)^{e_{1}}\cdots\bar{p}_{r}(X)^{e_{r}}.  $$
    是多项式$\bar{p}(X)\equiv p(X)(modP)$在剩余类域$o/P$中的不可约多项式分解,$p_{i}(X)\in o[X]$是首一的。那么$$
  \mathcal{B}_{i}=P\mathcal{O}+p_{i}(\theta)\mathcal{O},i=1,\cdots,r
  $$是$\mathcal{O}$的不同的素理想,$\mathcal{B}$的剩余类次数$f_{i}$是$\bar{p}_{i}(X)$的次数,并且$$
 P=\mathcal{B}_{1}^{e_{1}}\cdots\mathcal{B}_{r}^{e_{r.}}
  $$
 证明见Neukirch.Algebraic number theory.p48.\\
 \textbf{命题4.4:}如果$L|K$是可分扩张,那么K中仅有有限个在L上分歧的素理想.\\
 证明:有限可分扩张是单扩张,于是有$\theta\in \mathcal{O}$使得$L=K(\theta).$设$p(X)\in o[X]$是其在K上的最小多项式。$p(X)$的判别式为
 $$
 d=d(1,\theta,\cdots,\theta^{n-1})=\prod_{i<j}(\theta_{i}-\theta_{j})^{2}\in o
 $$
 $\prod_{i<j}(\theta_{i}-\theta_{j})^{2}\in o$是由于$\prod_{i<j}(\theta_{i}-\theta_{j})^{2}$是关于$\theta_{i},i=1,\cdots,n-1$的对称多项式,从而能表示成其初等多项式的关系式,而其初等多项式的值是$p(X)$的系数。下面断言K中每个与d和$o[X]$的导子$\mathfrak{F}$都互素(即$d\notin \mathfrak{p}$且$\mathfrak{p}\mathcal{O}+\mathfrak{F}=\mathcal{O}$)的素理想
 $\mathfrak{p}$是不分歧的。事实上,在上述假设下,由上述命题4.3,分歧指数$e_{i}$等于1只需在
 $o/\mathfrak{p}$中$\bar{p}(X)=p(X)mod\mathfrak{p}$的分解式中$e_{i}$都是1,即$\bar{p}(X)$无重根,此时由于$\bar{p}(X)$的判别式$\bar{d}=d(mod \mathfrak{p})$非零,于是剩余类域扩张$\mathcal{O}/\mathfrak{P}_{i}|o/\mathfrak{p}$由元素$\bar{\theta}=\theta(mod\mathfrak{P}_{i})$生成,因此是可分的,从而$\mathfrak{p}$不分歧。下面需要证明K中不与d互素(即素理想P满足$d\in P$)或不与$o[X]$的导子$\mathfrak{F}$互素(即素理想$\mathfrak{p}$满足$P\mathcal{O}+\mathfrak{F}\neq\mathcal{O}$)的素理想均为有限个。\\
 由于o是Dedekind整环$do$有唯一的素理想分解$$
 do=\mathfrak{p}_{1}^{e_{1}}\cdots\mathfrak{p}_{s}^{e_{s}}.
 $$
 于是只有有限个素理想$\mathfrak{p}_{1},\cdots,\mathfrak{p}_{s}$包含d。\\
 因为$\mathcal{O}$是Dedekind整环,$\mathfrak{F}$有唯一的素理想分解
 $$\mathfrak{F}=\mathfrak{P}_{1}^{a_{1}}\cdots\mathfrak{P}_{t}^{a_{t}}.
 $$
 若$\mathfrak{p}$是o中素理想且$\mathfrak{p}\mathcal{O}+\mathfrak{F}\neq \mathcal{O}$,那么$\mathcal{O}$中存在素理想$\mathfrak{P}$使得$\mathfrak{p}\mathcal{O}+\mathfrak{F}\subseteq \mathfrak{P}.$
 Dedeking整环中包含等价于整除,于是$\mathfrak{P}|\mathfrak{p}\mathcal{O},\mathfrak{P}|\mathfrak{F}.$j进而$\mathfrak{p}=\mathfrak{P}\cap A$由理想分解的唯一性,
 $\mathfrak{P}=\mathfrak{P}_{i}$对某一i成立,综上便有$o$中素理想满足$\mathfrak{p}\mathcal{O}+\mathfrak{F}\neq \mathcal{O}$当且仅当$\mathfrak{p}=\mathfrak{P}_{i}\cap A.$对于某一i成立,$\mathfrak{P}_{i}$是导子$\mathfrak{F}$分解式中的素理想。\\
\textbf{命题4.5:}设$L|K$是数域的扩张,$L=K(\alpha),\alpha \in \mathcal{O}_{L},n=[L:K].f(x)=x^{n}+c_{1}x^{n-1}+\cdots +c_{n}\in \mathcal{O}_{K}[x]$是整数$\alpha$在K上的极小多项式,则\\
(1)$\mathcal{O}_{K}[\alpha]$是$\mathcal{O}_{L}$的子环,并且加法商群$\mathcal{O}_{L}/\mathcal{O}_{K}[\alpha]$是有限群;\\
证明:$\mathcal{O}_{K}[\alpha]$显然是$\mathcal{O}_{L}$的子环,由于加法群$\mathcal{O}_{L}$和$\mathcal{O}_{K}[\alpha]$均是秩为$[L:K]$的自由Abel群,由Abel群基本定理可以得出加法商群$\mathcal{O}_{L}/\mathcal{O}_{K}[\alpha]$是有限群。\\
(2)设p是素数,$P$是$\mathcal{O}_{K}$的素理想,$P|p$,则$\mathcal{O}_{K}/P$是特征为p的有限域;\\
证明:由于$P\cap Z=p,$并且$\mathcal{O}_{K}/P$是p元域$Z/pZ$的扩域,从而$\mathcal{O}_{K}/P$是特征为p的有限域.\\
(3)如果$p$不整除群$\mathcal{O}_{L}/\mathcal{O}_{K}[\alpha]$的阶,令$f(x)$在主理想整环$\mathcal{O}_{K}/P[x]$中分解为
$$
f(x)=p_{1}(x)^{e_{1}}p_{2}(x)^{e_{2}}\cdots p_{g}(x)^{e_{g}} (mod p)
$$
其中$p_{1}(x),\cdots,p_{g}(x)$均为$\mathcal{O}_{K}[x]$中的首一多项式,并且看作是$\mathcal{O}_{K}/P[x]$中的多项式时为两两不同的不可约多项式,则$P$在$\mathcal{O}_{L}$中的分解式为
$$
P\mathcal{O}_{L}=\mathcal{B}_{1}^{e_{1}}\cdots  \mathcal{B}_{g}^{e_{g}} 
$$
其中
$$
\mathcal{B}_{i}=(P,p_{i}(\alpha)),e_{i}=e(\mathcal{B}_{i}/P),f(\mathcal{B}_{i}/P)=degp_{i}(x)(1\leq i\leq g)
$$
证明:令$f_{i}=degp_{i}(x)(1\leq i \leq g).$我们证明下面事实;\\
对于每个i,或者$\mathcal{B}_{i}=\mathcal{O}_{L},$或者$\mathcal{O}_{L}/\mathcal{B}_{i}$是$|\mathcal{O}_{K}/p|^{f_{i}}$元域。这是因为:$p_{i}(x)$在$\mathcal{O}_{K}/P[x]$中不可约,从而$F_{i}=\mathcal{O}_{K}/P[x]/(p_{i}(x))$为域,自然同态$\phi :\mathcal{O}_{K}[x]\rightarrow \mathcal{O}_{K}/P[x]/(p_{i}(x))$是满同态,并且$Ker\phi=(P,p_{i}(x)),$从而有同构
$$
\mathcal{O}_{K}/(P,p_{i}(x))\simeq \mathcal{O}_{K}/P[x]/(p_{i}(x))=F_{i}
$$
从而左边也是域。因此$(P,p_{i}(x)) $是$\mathcal{O}_{K}[x]$的极大理想 ,再做映射$$
\pi :\mathcal{O}_{K}[x]\rightarrow \mathcal{O}_{L}/\mathcal{B}_{i},\pi (f(x))=f(\alpha)+\mathcal{B}_{i}
$$
这是环同态,由于$\mathcal{B}_{i}=(P,p_{i}(\alpha))$,从而$(P,p_{i}(x))\subseteq Ker\pi,$但是$(P,p_{i}(x))$是$\mathcal{O}_{K}[x]$的极大理想。因此$Ker\pi=(P,p_{i}(x))$或者$\mathcal{O}_{K}[x.]$\\
  我们再证$\pi$是满同态,这即证明$\mathcal{O}_{K}[\alpha]+\mathcal{B}_{i}=\mathcal{O}_{L}$即可,由于$p\in P\subseteq \mathcal{B}_{i},$从而$p\mathcal{O}_{L}\subseteq \mathcal{B}_{i},$于是只要证明$\mathcal{O}_{K}[\alpha]+p\mathcal{O}_{L}=\mathcal{O}_{L}$即可,这是因为$p\nmid|\mathcal{O}_{L}/\mathcal{O}_{K}[\alpha]|,$而$|\mathcal{O}_{L}/P\mathcal{O}_{L}|=p^{[L:Q]},$从而$|\mathcal{O}_{L}/\mathcal{O}_{K}[\alpha]+p\mathcal{L}|$可除尽$(|\mathcal{O}_{L}/\mathcal{O}_{K}[\alpha]|,|\mathcal{O}_{L}/P\mathcal{O}_{L}|)=1,$因此$|\mathcal{O}_{L}/\mathcal{O}_{K}[\alpha]+p\mathcal{L}|=1,$即$\mathcal{O}_{K}[\alpha]+p\mathcal{O}_{L}=\mathcal{O}_{L}$,从而$\pi$为满同态,于是$\mathcal{O}_{L}/\mathcal{B}_{i}$或者同构于$\mathcal{O}_{K}[x]/(P,p_{i}(x))\cong F_{i},$从而$\mathcal{O}_{L}/\mathcal{B}_{i}$是$|\mathcal{O}_{K}/P|^{f_{i}}$元域;或者同构于$\mathcal{O}_{K}[x]/\mathcal{O}_{K}[x]$,即$\mathcal{B}_{i}=\mathcal{O}_{L}.$\\
  当$i\neq j$时,$(\mathcal{B}_{i},\mathcal{B}_{j})=1.$这是因为$p_{i}(x),p_{j}(x)$是$\mathcal{O}_{K}/P[x]$中的不同的不可约多项式,而$\mathcal{O}_{K}/P[x] $是主理想整环,从而有$$
  h(x),k(x)\in \mathcal{O}_{K}/P[x]
  $$
使得$hp_{i}+kp_{j}\equiv 1 (modP)$,带入$x=\alpha $即知$$
p_{i}(\alpha)h(\alpha)+p_{j}(\alpha)k(\alpha)\equiv 1(mod P\mathcal{O}_{L})
$$
于是$1\in (P,p_{i}(\alpha),p_{j}(\alpha))=(\mathcal{B}_{i},\mathcal{B}_{j})$\\
$P\mathcal{O}_{L}|\mathcal{B}_{1}^{e_{1}}\cdots \mathcal{B}_{g}^{e_{g}}.$这是因为;令$\gamma_{i}=p_{i}(\alpha),$
则$\mathcal{B}_{i}=(P,\gamma_{i}),$由(2)知当$i\neq j$时$(P,\gamma_{i},\gamma_{j})=1.$令$A=(P,\gamma_{1}^{e_{1}}\cdots\gamma_{g}^{e_{g}}),$则$$
\mathcal{B}_{1}\mathcal{B}_{2}=(P,\gamma_{1})(P,\gamma_{2})=(P^{2},P\gamma_{1},P\gamma_{1},\gamma_{1}\gamma_{2})=(P(P,\gamma_{1},\gamma_{2}),\gamma_{1}\gamma_{2})=(P,\gamma_{1}\gamma_{2})
$$
$$
\mathcal{B}_{1}^{2}=(P,\gamma_{1})^{2}=(P^{2},P\gamma_{1},\gamma_{1}^{2})\subseteq (P,\gamma_{1}^{2})
$$
由此归纳下去,
$$
\mathcal{B}_{1}^{e_{1}}\cdots \mathcal{B}_{g}^{e_{g}}\subseteq (P,\gamma_{1}^{e_{1}}\cdots\gamma_{g}^{e_{g}})=A
$$
只需再证$A=P\mathcal{O}_{L}$即可。为此将$x=\alpha $带入$f(x)\equiv p_{1}(x)^{e_{1}}\cdots p_{g}(x)^{e_{g}}(mod P),$便得到$$
\gamma_{1}^{e_{1}}\cdots\gamma_{g}^{e_{g}}\equiv f(\alpha)=0(mod P\mathcal{O}_{L})
$$
即$\gamma_{1}^{e_{1}}\cdots\gamma_{g}^{e_{g}}\in P\mathcal{O}_{L},$从而$A=(P,\gamma_{1}^{e_{1}}\cdots\gamma_{g}^{e_{g}})=P\mathcal{O}_{L}.$\\
现在来证明(3):我们不妨假设$\mathcal{B}_{1},\cdots ,\mathcal{B}_{s}$均不为$\mathcal{O}_{L},$而$\mathcal{B}_{s+1},\cdots ,\mathcal{B}_{g}=\mathcal{O}_{L},$则$\mathcal{B}_{i}(1\leq i\leq s)$均为$\mathcal{O}_{L}$的素理想,并且$P\subseteq \mathcal{B}_{i}.f_{i}(\mathcal{B}_{i}/P)=[\mathcal{O}_{L}/\mathcal{B}_{i}:\mathcal{O}_{K}/P]=f_{i}(1\leq  i\leq s).$由上知$\mathcal{B}_{i}((1\leq i \leq s)$两两互异,由$P\mathcal{O}_{L}|\mathcal{B}_{1}^{e_{1}}\cdots \mathcal{B}_{g}^{e_{g}}$知$P\mathcal{O}_{L}=\mathcal{B}_{1}^{d_{1}}\cdots \mathcal{B}_{s}^{d_{s}},d_{i}\leq e_{i}(1\leq i\leq s).$由于$n=d_{1}f_{1}+\cdots d_{s}f_{s}\leq e_{1}f_{1}+\cdots e_{g}f_{g},$且$f(x)\equiv p_{1}(x)^{e_{1}}\cdots p_{g}(x)^{e_{g}}$
知$s=g,e_{i}=d_{i}(1\leq i\leq g),$命题证毕。
    \subsection{Hilbert分歧理论}
若数域扩张$L|K$是伽罗瓦扩张,素理想的分解问题将会变得更重要和有趣,下面记号仍与上面相同,即$o$是戴德金整环,分式域为$K,$ $L|K$是有限域扩张,$\mathcal{O}$是o在L上的整闭包,记伽罗瓦群为$G=G(L|K),$任给$a \in \mathcal{O},$a的共轭元$\sigma a \in \mathcal{O},\forall \sigma \in \mathcal{O}.$若$\mathfrak{P}$是$\mathcal{O}$中的素理想,且$\mathfrak{P}\cap o=\mathfrak{p},$则对于每个$\sigma \in G,\sigma \mathfrak{P}\cap 0=\sigma(\mathfrak{P}\cap o)=\sigma \mathfrak{p}=\mathfrak{p}.$理想$\sigma \mathfrak{P}$叫做$\mathfrak{P}$的共轭理想。\\
\textbf{命题5.1}伽罗瓦群在$\mathcal{O}$的所有卧于素理想$\mathfrak{p}$(即:满足$\mathfrak{P}\cap o=\mathfrak{p}$的$\mathfrak{P}$组成的集合)的素理想$\mathfrak{P}$组成的集合上的作用是可迁的.\\
证明:设$\mathfrak{P}$和$\mathfrak{P}^{'}$是卧于$\mathfrak{p}$上的两个素理想,若对任意$\sigma \in G,$都有$\mathfrak{P}^{'}\neq \sigma \mathfrak{P},$那么由中国剩余定理知存在$x\in \mathcal{O}$使得
$$
x\equiv 0\quad (mod \quad \mathfrak{P}^{'}),x\equiv 1\quad(mod\quad  \mathcal{\sigma \mathfrak{P}}),\sigma \in G
$$
于是范数$N_{L|K}(x)=\prod _{\sigma \in G}\sigma x$属于理想$\mathfrak{P}^{'}\cap o=\mathfrak{p}.$令一方面,对于所有$\sigma \in G,$$x\notin \sigma  \mathfrak{P}$,因此$\sigma x\notin \mathfrak{P},$推出$\prod _{\sigma \in G}\sigma x\notin \mathfrak{P}\cap o=\mathfrak{p},$矛盾!\\
\textbf{定义5.2:}如果$\mathfrak{P}$是$\mathcal{O}$的素理想,子群$$
G_{\mathfrak{P}}=\{\sigma \in G|\sigma \mathfrak{P}=\mathfrak{P}\}
$$
叫做域K上理想$\mathfrak{P}$的分解群,域$$
Z_{\mathfrak{P}}=\{x\in L|\sigma x=x ,\forall \sigma \in G_{\mathfrak{P}}\}
$$
叫做K上理想$\mathfrak{P}$的分解域。\\
由定义$Z_{\mathfrak{P}}$是$G_{\mathfrak{P}}$的不动域,即$Z_{\mathfrak{P}}=Inv(G_{\mathfrak{P}})$,由伽罗瓦理论知$Gal(L|Z_{\mathfrak{P}})=G_{\mathfrak{P}}$\\
下面再说一下一些定义,o中理想$\mathfrak{p}$在L中叫做\textbf{完全分裂}(totally split)的,如果$\mathfrak{p}$的分解式$\mathfrak{p}\mathcal{O}=\mathfrak{P}_{1}^{e_{1}}\cdots \mathfrak{P}_{r}^{e_{r}}$中$r=n=[L:K]$于是对于所有$i=1,\cdots ,r ,e_{i}=f_{i}=1;$理想称为分歧的,如果$\exists 1\leq i\leq r,e_{i}>1.$否则便称不分歧;若$r=1$,即$\mathfrak{p}\mathcal{O}=\mathfrak{P}^{n}$(有关系式$\sum_{i=1}^{r}e_{i}f_{i}=n$),则称理想是完全分歧,最后若$\mathfrak{p}\mathcal{O}=\mathfrak{P}(r=1,e(\mathfrak{P}/\mathfrak{p})=1,f(\mathfrak{P}/\mathfrak{p})=n)$称理想是惯性的。\\
设$\mathfrak{P}$是$\mathfrak{p}\mathcal{O}$在$\mathcal{O}$中素理想分解中的一理想,$\sigma$遍历$G/G_{\mathfrak{P}}$中元素的代表元,那么$\sigma(\mathfrak{P})$遍历卧于$\mathfrak{p}$上的所有素理想,且每个出现一次,即有$r=(G:G_{\mathfrak{P}}).$特别地,有$$
G_{\mathfrak{P}}=1\Leftrightarrow Z_{\mathfrak{P}}=L\Leftrightarrow \mathfrak{p}\quad is\quad totally\quad split 
$$
$$
G_{\mathfrak{P}}=G\Leftrightarrow Z_{\mathfrak{P}}=K\Leftrightarrow \mathfrak{p} \quad is \quad nonsplit
$$
$L|K$是伽罗瓦扩张时,剩余类域次数$f_{i}$,分歧指数$e_{i}$与$i$无关事实上,记$\mathfrak{P}=\mathfrak{P}_{1}$任意$\mathfrak{P}_{i},$存在$\sigma_{i} \in G$使得$\mathfrak{P}_{i}=\sigma_{i}\mathfrak{P},$(这可由命题5.1推得,)同构$\sigma_{i}:\mathcal{O}\rightarrow \mathcal{O}$诱导同构
$$
\mathcal{O}/\mathfrak{P}\rightarrow \mathcal{O}/\sigma_{i}\mathfrak{P},\quad 
a \quad mod \quad \mathfrak{P}\mapsto\sigma_{i}a\quad mod \sigma_{i}\mathfrak{P},
$$
故$$f_{i}=[\mathcal{O}/\sigma_{i}(\mathfrak{P}):o/\mathfrak{p}]=[\mathcal{O}/\mathfrak{P}:o/\mathfrak{p}],i=1,\cdots,r.$$
进一步,由于$\sigma_{i}(\mathfrak{p}\mathcal{O})=\mathfrak{p}\mathcal{O},$从而由$$
\mathfrak{P}^{v}|\mathfrak{p}\mathcal{O}\Leftrightarrow \sigma_{i}(\mathfrak{P}^{v})|\sigma_{i}(\mathfrak{p}\mathcal{O})\Leftrightarrow(\sigma_{i}\mathfrak{P})^{v}|\mathfrak{p}\mathcal{O}
$$
推出$e_{i},i=1,\cdots,r$相等。\\
于是o中理想$\mathfrak{p}$在$\mathcal{O}$中的素理想分解有形式$\mathfrak{p}=(\prod_{\sigma}\sigma\mathfrak{P})^{e},$其中$\sigma $遍历$G/G_{\mathfrak{P}}$的代表系。\\
令$\mathfrak{P}_{Z}=\mathfrak{P}\cap Z_{\mathfrak{P}}$,从而首先$\mathfrak{P}_{Z}\subset \mathfrak{P}\subset \mathcal{O}$,即$\mathfrak{P}_{Z}$在o上是整的,从而$\mathfrak{P}_{Z}$包含于o在域$Z_{\mathfrak{P}}$整闭包$\mathcal{O}_{Z_{\mathfrak{P}}},$且易见是其素理想.\\
为了进一步应用命题5.1,下面说明一下看法,由伽罗瓦理论知域扩张$L|Z_{\mathfrak{P}}$是伽罗瓦扩张,环$\mathcal{O}_{Z_{\mathfrak{P}}}$是
Dedekind整环,且可知$\mathcal{O}_{Z_{\mathfrak{P}}}$在L中闭包恰为$\mathcal{O}$(可证明两者相互包含)。命题5.1中基取在Dedekind整环o上(一般都取成整数环Z),由上分析还可将基取在$\mathcal{O}_{Z_{\mathfrak{P}}}$上,从而可得到下述命题\\
\textbf{命题5.3}(i)$\mathfrak{P}_{Z}$在L上不分裂,即$\mathfrak{P}$是L中唯一位于$\mathfrak{P}_{Z}$上的素理想。\\
(ii)$\mathfrak{P}$在$Z_{\mathfrak{P}}$上分歧指数为e,剩余类域次数为f.\\
(iii)$\mathfrak{P}_{Z}$在域K上分歧指数和剩余类域次数均为1.\\
证明:(i)由于$G(L|Z_{\mathfrak{P}})=G_{\mathfrak{P}}$卧于$\mathfrak{P}_{Z}$上的理想是$\sigma\mathfrak{P},\sigma \in G(L|Z_{\mathfrak{P}}),$都是$\mathfrak{P}.$\\
(ii)伽罗瓦扩张下,分歧指数,剩余类域次数均为常数,基本公式$n=efr,$这里$n:=|G|,r=(G:G_{\mathfrak{P}})$,于是$|G_{\mathfrak{P}}|=[L:Z_{\mathfrak{P}}]=ef.$令$e^{'},e^{''}$分别是$\mathfrak{P}$在$Z_{\mathfrak{P}}$和$\mathfrak{P}_{Z}$在K上的分歧指数,那么在$Z_{\mathfrak{P}}$中$\mathfrak{p}=\mathfrak{P}^{e^{''}}_{Z}\dots$,在L中$\mathfrak{P}_{Z}=\mathfrak{P}^{e^{'}}$.于是$\mathfrak{p}=\mathfrak{P}^{e^{''}e^{'}}\dots,$立即$e=e^{''}e^{'}.$相似的可得到等式$f=f^{'}f^{''}$.有理想$\mathfrak{P}_{Z}$在L中分解得基本公式得到$[L:Z_{\mathfrak{P}}]=e^{'}f^{'},$于是$e^{'}f^{'}=ef,$进而$e^{'}=e,f^{'}=f,e^{''}=f^{''}=1.$\\
对于每个$\sigma\in G_{\mathfrak{P}},\sigma\mathcal{O}=\mathcal{O},\sigma\mathfrak{P}=\mathfrak{P},$于是$\sigma$诱导出自同构$$
\bar{\sigma}:\mathcal{O}/\mathfrak{P}\longrightarrow \mathcal{O}/\mathfrak{P},a\quad mod \quad \mathfrak{P}\longmapsto \sigma a\quad mod \quad \mathfrak{P}
$$
令$\kappa(\mathfrak{P})=\mathcal{O}/\mathfrak{P},\kappa(\mathfrak{p})=o/\mathfrak{p},$有下面命题\\
\textbf{命题5.4:}域扩张$\kappa(\mathfrak{P})|\kappa(\mathfrak{p})$是正规扩张,$G_{\mathfrak{P}}\longrightarrow G(\kappa(\mathfrak{P})|\kappa(\mathfrak{p}))$是满射。\\
证明:由命题5.3知$[\mathcal{O}_{Z_{\mathfrak{P}}}/\mathfrak{P}_{Z}:o/\mathfrak{p}]=f^{''}=1,$于是两者相同,即有$\kappa(\mathfrak{p})=\mathcal{O}_{Z_{\mathfrak{P}}}/\mathfrak{P}_{Z}=\kappa(\mathfrak{P}_{Z}).$伽罗瓦群
$G(\kappa(\mathfrak{P})|\kappa(\mathfrak{p}))=G(\kappa(\mathfrak{P})|\kappa(\mathfrak{P}_{Z})).$于是可以假设$K=Z_{\mathfrak{P}},$这样做的好处是$G_{\mathfrak{P}}$是域扩张$L|Z_{\mathfrak{P}}=L|K$的伽罗瓦群,这在后面要用到。设$\theta\in\mathcal{O}$是$\bar{\theta}\in\kappa(\mathfrak{P})$的一代表元,$f(X),\bar{g}(X)$分别是$\theta$在域K上,$\bar{\theta}$在域$\kappa(\mathfrak{p})$上的最小多项式,那么$\bar{\theta}=\theta mod\mathfrak{P}$是多项式$\bar{f}(X)=f(X)mod \mathfrak{p}$的零点,于是$\bar{g}(X)$整除$\bar{f}(X)$由于$L|K$是正规扩张,$f(X)$在$\mathcal{O}$分解为一次因式,故$\bar{f}(X)$在$\kappa(\mathfrak{P})$中也分解成一次因式,$\bar{g}(X)$同样是,这说明$\kappa(\mathfrak{P})|\kappa(\mathfrak{p})$是正规扩张。\\
现在设$\bar{\theta}$是$\kappa(\mathfrak{P})|\kappa(\mathfrak{p})$的极大可分子扩张的本原元素(在代数数域的扩张都是可分扩张.这里的基取为Dedekind整环o,不是整数环Z),
$$
\bar{\sigma}\in G(\kappa(\mathfrak{P})|\kappa(\mathfrak{p}))=G(\kappa(\mathfrak{p})(\bar{\theta})|\kappa(\mathfrak{p})).
$$
那么$\bar{\sigma}\bar{\theta}$是$\bar{g}(X)$的根,因此也是$\bar{f}(X)$的根,故存在$f(X)$的零点$\theta^{'}$使得
$\theta^{'}\cong \bar{\sigma}\bar{\theta}mod\mathfrak{P}.$$\theta^{'}$与$\theta$是共轭的,即存在$\sigma\in G(L|K)$使得$\theta^{'}=\sigma\theta.$($G_{\mathfrak{P}}$是域扩张$L|Z_{\mathfrak{P}}=L|K$的伽罗瓦群),由于$\sigma\theta\cong \bar{\sigma}\bar{\theta}mod \mathfrak{P}.$故$\sigma$映射到$\bar{\sigma}.$这就证明了满射.\\
\textbf{定义5.5:}同态$G_{\mathfrak{P}}\longrightarrow G(\kappa(\mathfrak{P})|\kappa(\mathfrak{p}))$的核$I_{\mathfrak{P}}\subseteq G_{\mathfrak{P}}$叫做$\mathfrak{P}$在K上的惯性群,其不动域$T_{\mathfrak{P}}=\{x\in L|\sigma x=x,\forall \sigma\in I_{\mathfrak{P}}\}$叫做$\mathfrak{P}$在K上的惯性域。从而有域“塔”
$K\subseteq Z_{\mathfrak{P}}\subseteq T_{\mathfrak{P}}\subseteq L.$同时还看出$I_{\mathfrak{P}}$是$G_{\mathfrak{P}}$的正规子群,从而由伽罗瓦理论知$T_{\mathfrak{P}}|Z_{\mathfrak{P}}$是正规扩张且$G(T_{\mathfrak{P}}|Z_{\mathfrak{P}})\cong G_{\mathfrak{P}}/I_{\mathfrak{P}}.$实际上由于可分扩张的子扩张是可分扩张,从而$T_{\mathfrak{P}}|Z_{\mathfrak{P}}$是伽罗瓦扩张,再结合命题5.4知
$G(T_{\mathfrak{P}}|Z_{\mathfrak{P}})\cong G(\kappa(\mathfrak{P})|\kappa(\mathfrak{P})).$ \\
若剩余类域扩张$\kappa(\mathfrak{P})|\kappa(\mathfrak{p})$是可分扩张,那么$|G(\kappa(\mathfrak{P})|\kappa(\mathfrak{p}))|=[\kappa(\mathfrak{P}):\kappa(\mathfrak{p})]=f.$从而$[T_{\mathfrak{P}}:Z_{\mathfrak{P}}]=|G(T_{\mathfrak{P}}|Z_{\mathfrak{P}})|=(G_{\mathfrak{P}}:I_{\mathfrak{P}})=f.$再由$[L:Z_{\mathfrak{P}}]=ef,G(L|T_{\mathfrak{P}})=I_{\mathfrak{P}}$知$|I_{\mathfrak{P}}|=[L:T_{\mathfrak{P}}]=e.$\\
对域扩张$L|T_{\mathfrak{P}}$应用命题5.4,即考虑映射
$G_{\mathfrak{P}}\longrightarrow G(\kappa(\mathfrak{P})|\kappa(\mathfrak{B}_{T})),\mathfrak{P}_{T}=\mathfrak{P}\cap T_{\mathfrak{P}}.$注意这里$G_{\mathfrak{P}}$即为$I_{\mathfrak{P}}.$这一映射的核同样是$I_{\mathfrak{P}}$于是得到$G(\kappa(\mathfrak{P})|\kappa(\mathfrak{P}_{T}))=1.$由上假设知$\kappa(\mathfrak{P})|\kappa(\mathfrak{P}_{T})$是可分扩张,在结合命题5.4知是伽罗瓦扩张,进而$\kappa(\mathfrak{P})=\kappa(\mathfrak{P}_{T}).$即$\mathfrak{P}$在$\mathfrak{P}_{T}$上的剩余类域次数为1.再由基本公式知分歧指数e.还可得到$\mathfrak{P}_{T}$在$\mathfrak{P}_{Z}$上的分歧指数为1,剩余类域次数是f.\\
    \subsection{Minkowski理论}
    下面首先给出格的定义,然后证明Minkowski的一个定理\\
    \textbf{定义3.1:}$V$是n维实线性空间,$V$中的一个\textbf{格}是形如
    $$\Gamma=\mathcal{Z}\nu_{1}+\cdots +\mathcal{Z}\nu_{m}$$的子群。
    其中$\nu_{1},\cdots \nu_{m}$是$V$中线性无关的向量。集合$$
    \boldsymbol{\Phi}=\{x_{1}\nu_{1}+\cdots x_{m}\nu_{m}|x_{i}\in R,0\leq x_{i}<1\}
    $$
    称为格的基本网.格称为\textbf{完备}的如果$m=n.$\\
    设$e_{1},\cdots ,e_{n}$是$V$的一组标准正交基,则$ \boldsymbol{\Phi}=\{x_{1}\nu_{1}+\cdots x_{n}\nu_{n}|x_{i}\in R,0\leq x_{i}<1\}$的体积为$vol(\Phi)=|det(A)|$,其中矩阵$A=(a_{ij})$是从基
    $e_{1},\cdots ,e_{n}$到基$\nu_{1},\cdots ,\nu_{n}$的转换矩阵,$\nu_{i}=\sum_{k}a_{ik}e_{k}.$\\
    \textbf{Minkowski格点定理:} 设$\Gamma$是欧式空间$V$中的完备格,$X$是$V$中关于原点对称的凸集,若$$vol(X)>2^{n}vol(\Gamma),$$则$X$至少包含$\Gamma$的一个格点$\gamma \in \Gamma$。\\
    证明:只需证明存在两格点$\gamma_{1},\gamma_{2}\in \Gamma,$使得$$
    (\frac{1}{2}X+\gamma_{1})\cap (\frac{1}{2}X+\gamma_{2})\neq \emptyset.
    $$
    为此,假设所有$\frac{1}{2}X+\gamma,\gamma \in \Gamma$,是互不相交的,则集合$\Phi\cap (\frac{1}{2}X+\gamma),\gamma \in \Gamma $对于所有$\gamma \in \Gamma $也是互不相交的,于是有$$
    vol(\Phi)\geq \sum_{\gamma \in \Gamma}vol(\Phi\cap (\frac{1}{2}X+\gamma)).
    $$
    集合$\Phi\cap (\frac{1}{2}X+\gamma)$是集合$(\Phi-\gamma)\cap (\frac{1}{2}X)$通过平移得到的,因此两者有相同的体积,但$\Phi-\gamma,\gamma \in \Gamma$覆盖整个空间$V$,因此得到$$
    vol(\Phi)\geq \sum_{\gamma \in \Gamma}vol((\Phi-\gamma)\cap \frac{1}{2}X)=vol(\frac{1}{2}X)=\frac{1}{2^{n}}vol(X),
    $$这一矛盾证明了命题。\\
    下面定义整理想的范数\\
     设A是$\mathcal{O}_{K}$的非零整理想,$\{\omega_{1},\cdots ,\omega_{n}\}$是$\mathcal{O}_{K}$的一组整基,$\{\alpha_{1},\cdots,\alpha_{n}\}$是A的一组整基,则$\alpha_{i}\in \mathcal{O}_{K}$,$\{\omega_{1},\cdots ,\omega_{n}\}$的整线性组合
     ,于是$$
     (\alpha_{1},\cdots,\alpha_{n})^{t}=T(\omega_{1},\cdots ,\omega_{n})^{t},T=(t_{ij}),t_{ij}\in Z,detT\neq 0
     $$
     如果$\{\omega_{1}^{'},\cdots ,\omega_{n}^{'}\}$和$\{\alpha_{1}^{'},\cdots,\alpha_{n}^{'}\}$分别是$\mathcal{O}_{K}$和A的另一组整基,则$$
     (\alpha_{1}^{'},\cdots,\alpha_{n}^{'})^{t}=M(\alpha_{1},\cdots,\alpha_{n})^{t},
     (\omega_{1},\cdots ,\omega_{n})^{t}=N(\omega_{1}^{'},\cdots ,\omega_{n}^{'})^{t}
     $$
     其中M和N均为n阶整方阵,并且$|detM|=1,|detN|=1$而
     $$
     (\alpha_{1}^{'},\cdots,\alpha_{n}^{'})^{t}=MTN(\omega_{1}^{'},\cdots ,\omega_{n}^{'})^{t}
     $$由于$|det(MTN)|=|detT|$,这就表明正整数$|detT|$与$\mathcal{O}_{K}$和A的整基选取是无关的,即它是理想本身的不变量,在称它为整理想的范数,表示成$N_{K}(A)=N_{K/Q}(A).$\\
     设$\sigma_{1},\cdots,\sigma_{n}$是数域K到C中的n个嵌入,$n=[K:Q]$.由于M是整数矩阵,由上可得
     $$
     (\sigma_{i}(\alpha_{1}),\cdots,\sigma_{i}(\alpha_{n}))^{t}=M(\sigma_{i}(\omega_{1}),\cdots,\sigma_{i}(\omega_{n}))^{t}
     $$
     从而有矩阵等式:$(\sigma_{i}(\alpha_{j}))=M(\sigma_{i}(\alpha_{j})),$于是$$
     d_{K}(\alpha_{1},\cdots,\alpha_{n})=det(\sigma_{i}(\alpha_{j}))^{2}=(detM)^{2}(det(\sigma_{i}(\alpha_{j}))^{2})=N_{K}(A)^{2}d_{K}(\omega_{1},\cdots ,\omega_{n})=N_{K}(A)^{2}d(K)
     $$
     关于理想的范数还有下面命题\\
     设A为数域K中非零整理想,$A=P_{1}^{e_{1}}\cdots P_{r}^{e_{r}},$其中$P_{1},\cdots,P_{r}$是$O_{K}$中不同的素理想,$e_{i}\leq 1,$则$N_{K}(A)=|O_{K}/A|,N_{K}(A)=N_{K}(P_{1})^{e_{1}}\cdots N_{K}(P_{r})^{e_{r}}$ \\
      (证明请看冯克勤著《代数数论入门》) \\
    为将$Minkowski$格点定理应用到数论中,须构造映射,需要下面结论
    (证明请看冯克勤著《代数数论入门》)  
    :\\
    (1)每个数域扩张$L|K$都是单扩张,即存在$\gamma \in L$,使得$L=K(\gamma)$\\
    (2)设$K$是n次数域,即$[K:Q]=n$,则恰有$n$个从$K$到$\mathcal{C}$的$Q-$嵌入 ,其中设有$r_{1}$个实嵌入$\sigma_{i}:K\rightarrow R (1\leq i\leq r_{1})$和$r_{2}$个复嵌入$\sigma_{r_{1}+j}=\bar{\sigma}_{r_{1}+r_{2}+j}:K\rightarrow C(1\leq j\leq r_{2}),r_{1}+2r_{2}=n.$由此得到映射
    $$\sigma :K\rightarrow R^{n},\sigma(\alpha)=(\sigma_{1}(\alpha),\cdots,\sigma_{r_{1}}(\alpha ),Re(\sigma_{r_{1}+1}(\alpha)),\cdots,
    Re(\sigma_{r_{1}+r_{2}}(\alpha)),Im(\sigma_{r_{1}+1}(\alpha)),\cdots ,  Im(\sigma_{r_{1}+r_{2}}(\alpha)))
    $$
    其中$Re(\gamma),Im(\gamma)$分别表示复数$\gamma$的实部和虚部。从而$\sigma$为嵌入,称为$K$到$R^{n}$中的正则嵌入。\\
    \textbf{引理3.2:}设$\mathcal{P}$是n次数域$K$中的非零整理想,则$\sigma(\mathcal(P))$是$R_{n}$中的格,并且$Vol(\sigma(P))=2^{-r_{2}}N(P)|d(K)|^{1/2}$\\
    证明:存在$\alpha_{1},\cdots ,\alpha_{n} \in K,$使得$\mathcal(P)=Z\alpha_{1}\oplus\cdots \oplus Z\alpha_{n}.$取$e_{1},\cdots,e_{n}$为$R_{n}$的标准基,则$\sigma(\alpha_{i})=\sum_{j=1}^{n}x_{ij}e_{j}$,其中$$(x_{i1},\cdots,x_{in})=(\sigma_{j}(\alpha_{i}),\cdots,\sigma_{r_{1}}(\alpha_{i}),Re(\sigma_{r_{1}+1}(\alpha_{i})),\cdots,Re(\sigma_{r_{1}+r_{2}}(\alpha_{i})),Im(\sigma_{r_{1}+r_{2}+1}(\alpha_{i})),\cdots,Im(\sigma_{n}(\alpha_{i})))$$
    于是$\sigma(P)=Z\sigma(\alpha_{1})+\cdots+Z\sigma(\alpha_{n})$,并且$$
    Vol(\sigma(P))=|det(x_{ij})|=2^{-r_{2}}|det(\sigma_{j}(\alpha_{i}))|=2^{-r_{2}}|d_{K}(\alpha_{1},\cdots,\alpha_{n})|^{1/2}=2^{-r_{2}}N(P)|d(K)|^{1/2}
    $$
    由于上式右边不为零,即$det(x_{ij})\neq 0$,这表明$\sigma(\alpha_{1}),\cdots,\sigma(\alpha_{n})$是$R-$线性无关的,从而$\sigma(P)=Z\sigma(\alpha_{1})+\cdots+Z\sigma(\alpha_{n})$是$R^{n}$中的格。\\
    \textbf{引理3.3:}设P是数域K中的整理想,$[K:Q]=r_{1}+2r_{2}.$则\\
    (1)存在$0\neq x \in P,$使得$$
    |N_{K/Q}(x)|\leq (\frac{4}{\pi})^{r_{2}}\frac{n!}{n^{n}}|d(K)|^{\frac{1}{2}}N_{K/Q}(A)
    $$
    (2)K的每个理想类C中均有整理想$\mathcal{B},$使得$$
    N_{K/Q}(B)\leq (\frac{4}{\pi})^{r_{2}}\frac{n!}{n^{n}}|d(K)|^{\frac{1}{2}}
    $$
    证明:对于$y=(y_{1},\cdots ,y_{n})\in R^{n},$定义$$
\lambda(y)=\sum_{i=1}^{r_{1}}|y_{i}|+2\sum_{j=1}^{r_{2}}(y_{r_{1}+j}^2+y_{r_{1}+r_{2}+j}^2)^{1/2}
    $$
    对于$t>0,$定义集合$B_{t}=\{y=(y_{1},\cdots ,y_{n})\in R^{n}|\lambda(y)\leq t\},$可知这是关于原点对称的紧凸集,令$X(t)=\{y|\lambda(y)\leq t,y_{1}\geq 0,\cdots,y_{r_{1}}\geq 0\}$,则由对称性可得$vol(B_{t})=2^{r_{1}}vol(X(t)),$用极坐标变换后$n-r_{1}$个变量,即令$$
    y_{r_{1}+j}=\frac{1}{2}\rho_{j}cos\theta_{j}, y_{r_{1}+r_{2}+j}=\frac{1}{2}\rho_{j}sin\theta_{j},
    $$
    变换的雅可比行列式绝对值为$\rho_{j}/4,$由于对称性易得到$$
    vol(X(t))=2^{r_{1}}4^{-r_{2}}(2\pi)^{r_{2}}\int_{Z}\rho_{r_{1}+1}\cdots\rho_{r_{1}+r_{2}}dy_{1}\cdots dy_{r_{1}}d\rho_{r_{1}+1}\cdots d\rho_{r_{1}+r_{2}}  $$
    这里$$Z=\{(y,\rho)\in R^{r+s}|y_{i},\rho_{i}\neq 0,\sum y_{i}+\sum \rho_{i}\leq t \}$$ 
    问题便归结为积分$\int_{Z}\rho_{r_{1}+1}\cdots\rho_{r_{1}+r_{2}}dy_{1}\cdots dy_{r_{1}}d\rho_{r_{1}+1}\cdots d\rho_{r_{1}+r_{2}} $的计算,令$$y_{i}=tx_{i},1\leq r_{1},\rho_{r_{i}+j}=tx_{r_{i}+j},1\leq j\leq r_{2},$$
    从而$$
    \int_{Z}\rho_{r_{1}+1}\cdots\rho_{r_{1}+r_{2}}dy_{1}\cdots dy_{r_{1}}d\rho_{r_{1}+1}\cdots d\rho_{r_{1}+r_{2}}=t^{n}\int_{Z^{'}}x_{r_{i}+1}\cdots x_{r_{i}+r_{2}}dx_{1}\cdots dx_{r_{1}}dx_{r_{1}+1}\cdots dx_{r_{1}+r_{2}}
    $$
    这里$Z^{'}=\{(x_{i})\in R^{r_{1}+r_{2}}|\sum x_{i}\leq 1\},$\\
    更一般地,积分$$
    I(a_{1}.\cdots,a_{m},t)=\int_{Z(t)}x_{1}^{a_{1}}\cdots x_{m}^{a_{m}}dx_{1}\cdots dx_{m}
    $$这里$Z(t)=\{x\in R^{m}|x_{i}\geq 0,\sum x_{i}\leq t\}$,积分可以通过减少积分变量,并不断把t利用变量替换变为1求出。最后结果为$$
     I(a_{1}.\cdots,a_{m},t)=t^{\sum a_{i}+m}\frac{\Gamma(a_{1}+1)\cdots \Gamma(a_{m}+1)}{\Gamma(a_{1}+\cdots +a_{m}+m+1)}.
    $$
    回到上面可得出$vol(X(t))=2^{r_{1}}4^{-r_{2}}(2\pi)^{r_{2}}t^{n}/n!$\\
    根据引理3.2,对于K中非零整理想$P,$ $\sigma(P)$为$R^{n}$中的格,并且$$
    Vol(\sigma(P))=2^{-r_{2}}N(P)|d(K)|^{1/2}
    $$
    当$t^{n}=(\frac{4}{\pi})^{r_{2}}N(P)|d(K)|^{1/2}n!$时,$vol(B_{t})=2^{n}vol(\sigma(P)),$从而由Minkowski定理可知存在非零$x\in P,$使得$\sigma(x)\in B_{t},$即$$
    \lambda(\sigma(x))\leq t
    $$
    于是$$
    |N_{K/Q}(x)|=\prod_{i=1}^{n}|\sigma_{i}(x)|\leq (\frac{1}{n}\sum_{i=1}^{n}|\sigma_{i}(x)|)^{n}=\frac{1}{n^{n}}( \lambda(\sigma(x)))^{n}\leq \frac{1}{n^{n}}t^{n}=(\frac{4}{\pi})^{r_{2}}\frac{n!}{n^{n}}N(P)|d(K)|^{1/2}
    $$
    (2)设$P^{'}\in C.$由于$P^{'}$除以任何整数之后仍为理想类C中的理想,因此不妨可以设$P=P^{'}$是整理想,由(1)知有$$
    0\neq x\in P,N(x)\leq (\frac{4}{\pi})^{r_{2}}\frac{n!}{n^{n}}N(P)|d(K)|^{1/2}
    $$
    令$\mathcal{B}=xP^{-1}=xP^{'},$由于$x\in P$可知$\mathcal{B}$为C中的整理想,并且$$
    N(\mathcal{B})=N(x)N(P^{'})\leq (\frac{4}{\pi})^{r_{2}}\frac{n!}{n^{n}}N(PP^{'})|d(K)|^{1/2}
    $$由于$N(PP^{'})=N(\mathcal{O}_{K})=1,$证毕。\\
    应用Minkowski格点定理可以证明下述定理。\\
    \textbf{类数有限定理}:理想类群$Cl_{K}=J_{K}/P_{K}$是有限群,它的阶数叫做代数数域$K$的类数。(这里$J_{K}$是K的所有分式理想,$P_{K}$是K的分式主理想)\\
    证明:若$P$是$\mathcal{O}_{K}$中的素理想,$P\cap Z=pZ$,那么$\mathcal{O}_{K}/P$是$Z/pZ$的有限域扩张。次数设为$f\geq 1,$从而$N_{K/Q}(P)=p^{f}.$给定一个p,这里仅有有限个$P$使得$P\cap Z=pZ$,这是因为这意味着$P|(p).$由此可知仅有有限个素理想$P$使得$N_{K/Q}(P)=p^{f}.$因为每个整理想有素理想表示$A=P_{1}^{\mu_{1}}P_{2}^{\mu_{2}}\cdots P_{r}^{\mu_{r}},\mu_{i}>0.$并且$$
   N_{K/Q}(A)=N_{K/Q}(P_{1})^{\mu_{1}}N_{K/Q}(P_{2})^{\mu_{2}}\cdots N_{K/Q}(P_{r})^{\mu_{r}}
    $$
    可知给定一上界$M>0,$ $\mathcal{O}_{K}$仅有有限个理想A,使得$N_{K/Q}(A)\leq M.$因此我们可以通过选定M证明定理。由引理3.3(2)知若令$M=(\frac{4}{\pi})^{r_{2}}\frac{n!}{n^{n}}|d(K)|^{\frac{1}{2}},$则K的每个理想类中都有整理想满足这一条件,从而证明了K只有有限个理想类。
    \subsection{单位定理}
    \subsection{分圆域}
    \subsection{局部化}
    \textbf{定义1:}仅有唯一极大理想的环称为\textbf{局部环}.\\
    若A是局部环,其极大理想为$\mathcal{M},$则任意$a\in \mathcal{M}$是A中单位,这是由于主理想$(a)$不包含在任何极大理想中,从而是整个环,即可逆,进而还有$A^{*}=A-\mathcal{M}.$\\
    \textbf{定义2:}离散赋值环是一个仅有一个不为零的极大理想的主理想整环(即局部主理想整环。)\\
    离散赋值环中的极大理想形式为$P=(\pi)=\pi \mathcal{O},$$\pi$为素元,由于每个不属于$P$的元素都是单位(即可逆),从而在相伴的意义下,$\pi$是$\mathcal{O}$唯一的素元.$\mathcal{O}$中的非零元素因此能被写成形式$\varepsilon\pi^{n},\varepsilon \in \mathcal{O}^{*},n\geq 0.$更一般地,分式域$K$中非零元素$a\neq 0$能被唯一写成形式$$
    a=\varepsilon\pi^{n},\varepsilon\in \mathcal{O}^{*},n\in Z. 
    $$
    这里指数n叫做a的值,记为$\nu(a),$明显的有$(a)=P^{\nu(a)}.$\\
    赋值是一函数$\nu:K^{*}\rightarrow Z.$令$\nu(0)=\infty .$通过简单的计算,得到$\nu(ab)=\nu(a)+\nu(b),\nu(a+b)\geq min\{\nu(a),\nu(b)\}$\\
    \textbf{命题:}若$\mathcal{O}$是戴德金整环,$S\subseteq \mathcal{O}-\{0\}$是乘性子集,那么$\mathcal{O}S^{-1}$也是戴德金整环 。\\
    证明。\\
    \textbf{命题:}$\mathcal{O}$是诺特整环,$\mathcal{O}$是戴德金整环当且仅当对于每个非零素理想$P\neq 0,,$环的局部化$\mathcal{O}_{P}$是离散赋值环。\\
    证明:\\
    $\mathcal{O}$是戴德金整环,对于每个非零素理想$P\neq 0$,有离散赋值环$\mathcal{O}_{P},$和赋值$\nu_{P}:K\rightarrow Z.$赋值在理想的分解的有下述作用:如果$x\in K^{*}$,并且$(x)=\prod_{P}P^{\nu_{P}}$是主理想$(x)$的素理想的分解,那么对于每个$P,$有$\nu_{P}=\nu_{P}(x).$
    事实上,对于每个$\mathcal{O}$的素理想$Q\neq 0$
    由于$P\neq Q$时,$P\mathcal{O}_{Q}=\mathcal{O}_{Q},$因此$$
    x\mathcal{O}_{Q}=(\prod_{P}P^{\nu_{P}})\mathcal{O}_{Q}=Q^{\nu_{Q}}\mathcal{O}_{Q}=\mathcal{M}_{Q}^{\nu_{Q}}
    $$
    因此$\nu_{Q}(x)=\nu_{Q}$\\
    $\mathcal{O}$是戴德金环,令$$
    \mathcal{O}(X)=\{\frac{f}{g}|f,g\in \mathcal{O},g\notin P,P\subseteq X\},
    $$
    这里X是$\mathcal{O}$的一些不为零素理想为其元素组成的集合,$\mathcal{O}(X)$的非零素理想为$P_{X}=P\mathcal{O}(X)=\{\frac{f}{g}|f\in P,g\notin P,P\subseteq X\},P\subseteq X,$则可验证$\mathcal{O}_{P}=\mathcal{O}(X)_{PX},$事实上,
    $$\mathcal{O}(X)_{PX}=\{\frac{f}{g_{1}}/\frac{f_{2}}{g_{2}}|f\in \mathcal{O},f_{2}\in \mathcal{O}-P,g_{1},g_{2}\notin X\},\mathcal{O}_{P}=\{\frac{f}{g}|f\in \mathcal{O},g\in \mathcal{O}-P\}$$
    剩下的只需验证两者相互包含。\\
    \textbf{命题8.1:}$$1\rightarrow \mathcal{O}^{*}\rightarrow \mathcal{O}(X)^{*}\rightarrow \oplus_{P\notin X}K^{*}/\mathcal{O}^{*}_{P}\rightarrow CL(\mathcal{O})\rightarrow CL(\mathcal{O}(X))\rightarrow 1$$ 为正合列,并且$K^{*}/\mathcal{O}_{P}^{*}\cong Z$\\
    证明:这里第二个箭头为包含映射,第三个箭头是包含映射$\mathcal{O}(X)^{*}\rightarrow K^{*},$和投射$K^{*}\rightarrow K^{*}/\mathcal{O}^{*}_{P}$的合成。若$a\in \mathcal{O}(X)^{*}$属于该复合映射的核,那么对于$P\notin X,a\in \mathcal{O}_{P},$由于$\mathcal{O}_{P}=\mathcal{O}(X)_{PX},$故对于$P\in X,$同样有$a\in \mathcal{O}_{P},$于是$a\in \cap_{P}\mathcal{O}^{*}_{P}=\mathcal{O}^{*}.$这就证明了此处的正合性。\\
    箭头$$\oplus_{P\notin X}K^{*}/\mathcal{O}_{P}^{*}\rightarrow CL(\mathcal{O})$$
    为映射$$
    \oplus_{P\notin X}\alpha_{P}\quad mod \mathcal{O}_{P}^{*}\mapsto  \prod _{P\notin X}P^{\nu_{P}(\alpha_{P})}
    $$
    这里$\nu_{P}:K^{*}\rightarrow Z$是关于$\mathcal{O}_{P}$的$K^{*}$上的赋值。设$ \oplus_{P\notin X}\alpha_{P}\quad mod \mathcal{O}_{P}^{*}$是该映射核中元素,即像为一主理想设为$(\alpha)$,$\prod _{P\notin X}P^{\nu_{P}(\alpha_{P})}=(\alpha)=\prod _{P\in X}P^{\nu_{P}(\alpha)},\alpha\in K^{*}$
    由于理想分解的唯一性,上述意味着对于$P\in X,\nu_{P}(\alpha)=0,$对于$P\notin X,\nu_{P}(\alpha_{P})=\nu_{P}(\alpha),$
    进而可推出$\alpha\in \cap_{P\in X}\mathcal{O}^{*}_{P}=\mathcal{O}(X)^{*},\alpha\equiv \alpha_{P}mod(\mathcal{O}_{P}^{*}).$这就证明了此处的正合性。\\
    箭头$CL(\mathcal{O})\rightarrow CL(\mathcal{O}(X)).$是映射$Q\mapsto Q\mathcal{O}(X).$$X$中的素理想$P$映射到$\mathcal{O}(X)$中的素理想,由于$CL(\mathcal{O}(X))$是被这种形式的理想生成的,因此映射是满射。若$P\notin X,$我们有$P\mathcal{O}=(1),$这因为这该映射的核包含形如$\prod _{P\notin X}P^{\nu_{P}(\alpha_{P})}$的理想,这是前一个映射的像,因此这里也是正合的。最后,域的赋值$\nu_{P}:K^{*}\rightarrow Z$给出了同构$K^{*}/\mathcal{O}_{P}^{*}\cong Z.$
    
    
    \subsection{order}
    \textbf{定义1:}$K|Q$是n次代数数域,K的一个\textbf{order}是$\mathcal{O}_{K}$的一个包含长度为n的整基的子环,环$\mathcal{O}_{K}$叫做$K$的极大\textbf{order}。\\
    \textbf{命题:}K的一个\textbf{order}是一维(Krull维数)(每个素理想是极大理想)诺特整环\\
    证明:\\
    在下面,我们设$\mathcal{O}$是一维诺特整环,K是其分式域。环的分式理想不再形成群,当可以考虑可逆理想,即对于分式 理想$A,$存在分式理想$B,$使得$AB=\mathcal{O}.$分式理想$A$的逆仍为理想$A^{-1}=\{x\in K|xA\subseteq \mathcal{O}\}$\\
     \textbf{命题:}$\mathcal{O}$的分式理想$A$是可逆的当且仅当对于每个素理想$P\neq 0,$$A_{P}=A\mathcal{O}_{P}$是$\mathcal{O}_{P}$的主分式理想。\\
    证明:设$A$是可逆理想,$AB=\mathcal{O}.$$1=\sum _{i=1}^{r}a_{i}b_{i},a_{i}\in A,b_{i}\in B,$显然存在$a_{i}b_{i}\in \mathcal{O}_{P}$但不属于极大理想$P\mathcal{O}_{P},$故不妨假设$a_{1}b_{1}$是$\mathcal{O}_{P}$的单位,于是$A_{P}=a_{1}\mathcal{O}_{P},$这是由于若$x\in A_{P},xb_{1}\in A_{P}B=\mathcal{O}_{P},$因此$x=xb_{1}(b_{1}a_{1})^{-1}a_{1}\in a_{1}\mathcal{O}_{P}.$\\
    反之,设对于每个素理想$P$,$A_{P}=A\mathcal{O}_{P}$是主理想$a_{P}\mathcal{O}_{P},a_{P}\in K^{*},$我们可设$a_{P}\in A,$则可断定分式理想$A^{-1}=\{x\in K|xA\subseteq \mathcal{O}\}$是$A$的逆,假如不是,那么存在极大理想$P,$使得$AA^{-1}\subseteq P\subseteq \mathcal{O}.$设$a_{1},\cdots ,a_{n}$是理想$A$的生成系,由于$a_{i}\in A_{P}\mathcal{O}_{P},$我们可写$a_{i}=a_{P}\frac{b_{i}}{s_{i}},b_{i}\in \mathcal{O},s_{i}\in \mathcal{O}-P.$于是$s_{i}a_{i}\in a_{P}\mathcal{O}$,令$s=s_{1}\cdots s_{n},$有$sa_{i}\in a_{P}\mathcal{O},i=1,\cdots ,n$于是$sa_{P}^{-1}A\subseteq \mathcal{O},$故$sa_{P}^{-1}\in A^{-1},$这将导致$s=sa_{P}^{-1}a_{P}\in A^{-1}A\subseteq P,$矛盾!\\
   记$\mathcal{O}$的可逆理想组成的群为$J(\mathcal{O})$,它包含分式主理想$a\mathcal{O},a\in K^{*}$组成的群$P(\mathcal{O}).$\\
    \textbf{定义:}商群$$
    Pic(\mathcal{O})=J(\mathcal{O})/P(\mathcal{O})
    $$叫做环$\mathcal{O}$的Picard群。\\
    当$\mathcal{O}$是戴德金环时,Picard群无非是理想类群$CL_{K}.$一般情况下,对于$J(\mathcal{O}),Pic(\mathcal{O}).$我们有下面描述\\
    \textbf{命题:}映射$A\mapsto (A_{P})=(A\mathcal{O}_{P})$给出同构$$
    J(\mathcal{O})\cong \oplus_{P}P(\mathcal{O}_{P})
    $$
    证明:对于每个$A\in J(\mathcal{O}),A_{P}=A\mathcal{O}_{P}$是主理想,由于$A$仅包含在有限多个素理想(极大理想)$P$中,我们有态射$$
    J(\mathcal{O})\rightarrow \oplus_{P}P(\mathcal{O}_{P}),A\mapsto (A_{P})=(A\mathcal{O}_{P})
    $$
    单射:若对任意素理想$P,$有$\mathcal{O}_{P}=A_{P}$.那么$A\subseteq \cap_{P}\mathcal{O}_{P}=\mathcal{O}.$于是就有$A=\mathcal{O}$,不然存在极大理想$P$,使得$A\subseteq P\subset \mathcal{O}.i.e A_{P}\subseteq P\mathcal{O}_{P}\neq \mathcal{O}_{P}.$\\
    为证明满射,任给$(a_{P}\mathcal{O}_{P})\in \oplus_{P}P(\mathcal{O}_{P})$。$\mathcal{O}-$模$A=\cap_{P}a_{P}\mathcal{O}_{P}$是K的分式理想:事实上,对于几乎所有$P$,有$a_{P}\mathcal{O}_{P}=\mathcal{O}_{P}$,于是存在$c\in \mathcal{O}$,使得$ca_{P}\in \mathcal{O}_{P}$对于所有$P$成立,即是$cA\subseteq \cap _{P}\mathcal{O}_{P}=\mathcal{O}.$下面需要证明的便是$$
    A\mathcal{O}_{P}=a_{P}\mathcal{O}_{P},\forall P
    $$
    由$A$的定义知$\subseteq$的证明是平凡的。需要证的是$a_{P}\mathcal{O}_{P}\subseteq A\mathcal{O}_{P}$,
    
    
    环$\mathcal{O}$在K中的正规化(即在K中的闭包)记为$\bar{\mathcal{O}},$能够证明是戴德金环,有下引理\\
    \textbf{引理:}$\mathcal{O}$是一维诺特整环,$\bar{\mathcal{O}}$是其正规化,那么对于$\mathcal{O}$的每个非零理想$A\neq 0,$商环$\bar{\mathcal{O}}/A\bar{\mathcal{O}}$是有限生成$\mathcal{O}-$模。
    \subsection{一维概型}   
    \subsection{习题}
    1.(Stickelberger)代数数域K的判别式$d_{K}\equiv 0$或$\equiv 1\quad (mod 4)$\\
    2.设$d$无平方因子整数,$p$是不能整除$2d$的素数,$\mathcal{O}$是$\mathcal{Q}(\sqrt{d})$的整数环.证明$(p)=p\mathcal{O}$是$\mathcal{O}$的素理想当且仅当同余式$x^{2}\equiv d(modp)$无解。\\
    3.证明:只有有限个素理想的Dedekind整环是主理想整环。\\
    4.若A是Dedekind整环,$I\subset A$是非零理想,那么$A/I$的每个理想都是主理想。\\
    5.Dedekind整环的每个理想能被两个元素生成。\\
    6.设D是整环,证明下述条件等价:\\
    (i)D是Dedekind整环;\\
    (ii)D的每个分式理想可逆;\\
    (iii)D的每个非零理想有唯一的素理想分解\\
    7.设K是代数数域,$\mathcal{O}_{K}$是Z在K中的整闭包,有命题:Dedekind整环是UFD当且仅当是PID,从而研究UFD转换为探究PID。在代数中探究环是否为PID可能更为容易研究于是便定义出理想类群,下面给出另一种理想类群的定义:$\mathcal{O}_{K}$中理想$I$等价于$J$当且仅当存在$\alpha,\beta \in \mathcal{O}_{K}$使得$$\alpha I=\beta J.$$易验证这是等价关系,进一步定义等价类之间的乘法形成群,单位元是所有主理想形成的等价类,下面是两个练习:\\
    (i)验证所有主理想形成一等价类,即若$I$是使得$\alpha I=(\beta)$成立的理想,那么$I$是主理想。\\
    (ii)验证上述理想类群的定义与通常定义等价.(第一同构定理)\\
    8.设$\mathfrak{a}$是K的整理想,$\mathfrak{a}^{m}=(\alpha).$证明$\mathfrak{a}$在域$L=K(\sqrt[m]{\alpha})$中是主理想,即$\mathfrak{a}\mathcal{O}_{L}=(\beta),\beta\in\mathcal{O}_{L}.$\\
    9.证明对于每个数域K,存在有限域扩张$L$,使得K的每个理想是主理想。\\
    10.若代数数域扩张$L|K$是伽罗瓦扩张,其伽罗瓦群不是循环群,那么K至多有有限个不分裂的素理想。\\
    11.若代数数域$L|K$是伽罗瓦扩张,$\mathfrak{P}$是在K上不分歧的素理想(即$\mathfrak{p}=\mathfrak{P}\cap K$在L上不分歧),那么有且仅有一个子同构$\phi_{\mathfrak{P}}\in G(L|K)$使得$$
    \phi_{\mathfrak{P}}a\equiv a^{q}\quad mod \quad \mathfrak{P} \quad \forall a\in \mathcal{O}
    $$
    这里$q=|\kappa(\mathfrak{p})|.$这个自同构叫做Frobenius自同构.分解群$G_{\mathfrak{P}}$是循环的,$\phi_{\mathfrak{P}}$是$G_{\mathfrak{P}}$的一生成元。\\
    12.(Dirichlet's prime number theorem)对于每个自然数n存在无限个素数$p\equiv1\quad mod \quad n.$\\
    13.对于每个有限Abel群G,存在伽罗瓦扩张$K|Q$使得$G(K|Q)\cong G.$\\
    14.每个二次域$Q(\sqrt{d})$都包含在某个分圆域$Q(\zeta_{n}).$$\zeta_{n}$是n次本原单位根。\\
    15.设$L|K$是代数数域的有限域扩张(不必是伽罗瓦扩张),$N|K$是$L|K$的正规闭包,证明:$K$中理想$\mathfrak{p}$在L上完全分裂当且仅当它在N上完全分裂.(对于G的子群U和V,考虑G中的等价关系$\sigma \thicksim \sigma^{'}\iff \sigma^{'}=u\sigma v,\exists u\in U,v\in V$,对应的等价类$U\sigma V=\{u\sigma v|u\in U,v\in V\}$,叫做G关于U,V的双陪集,所有这些双陪集组成的集合记作$U\setminus G/V$)\\
    
    
    \textbf{解答}\\
    3.设R是Dedekind整环,$\mathfrak{p}_{1},\cdots,\mathfrak{p}_{n}$是R的所有素理想,对任意$1\leq i\neq j\leq n,\mathfrak{p}_{i}+\mathfrak{p}_{j}=R$(这是由于Dedekind整环中,素理想都是极大理想,而$\mathfrak{p}_{i}+\mathfrak{p}_{j}$是包含极大理想$\mathfrak{p}_{i}$的理想,从而是整个环),同样可知$\mathfrak{p}_1^2,\mathfrak{p}_2,\ldots,\mathfrak{p}_n$两两互素,取$\pi \in \mathfrak{p}_{1}\setminus\mathfrak{p}_{1}^{2}$(由Dedekind整环中理想分解的唯一性,$\mathfrak{p}_{1}\setminus\mathfrak{p}_{1}^{2}$非空),由中国剩余定理存在$x\in R$使得$$x\equiv \pi\,(\textrm{mod } \mathfrak{p}_1^2),\;\; x\equiv 1\,(\textrm{mod } \mathfrak{p}_k),\; k=2,\ldots,n
    $$
    设主理想$(x)=\mathfrak{p}_1^{e_1}\cdots \mathfrak{p}_n^{e_n},e_{i}\in N,$若有$e_{i}\geq 1,i\geq 2$,则$x\in (x)\subset \mathfrak{p}_{i}$,但是$x\equiv 1(mod \mathfrak{p}_{i})$,因此这是不可能的,故$(x)=\mathfrak{p}_{1}^{e_{1}},e^{1}\geq 1$,然而$x\notin\mathfrak{p}_{1}^{2} $,从而由$x\in(x)=\mathfrak{p}_{1}^{e_{1}}$推出$e_{1}=1$,于是$(x)=\mathfrak{p}_{1}.$同样可知R中每个素理想都是主理想,于是R的每个理想是主理想。\\
    4.设$I = \displaystyle\prod_{i =1}^n \mathfrak{p}_i^{e^i}.$,由中国剩余定理得到$A/I \cong \displaystyle\bigoplus_{i =1}^n A/\mathfrak{p}_i^{e^i}.$,从而只需证明$A/\mathfrak{p}_i^{e^i}$的理想是主理想,考虑投射$\pi:A \rightarrow A/\mathfrak{p}_i^{e^i}.$$A/\mathfrak{p}_i^{e^i}$的所有理想为$\mathfrak{p}_{i}^{n}(1\leq n\leq e_{i})$在$\pi$下的像,若$\pi(\mathfrak{p}_{i})=\pi(\mathfrak{p}_{i}^{2})$,那$\pi(\mathfrak{p}_{i})=0$,此时$A/\mathfrak{p}_i^{e^i}$是域,否则取$\alpha \in\pi( \mathfrak{p}_i) \setminus \pi(\mathfrak{p}_i^2).$那么$(\alpha )$是真理想,且$(\alpha) \not\subset \pi(\mathfrak{p}_i^n),n\geq 2$,由此推出$(\alpha)=\pi(\mathfrak{p}_{i})$,故$\pi(\mathfrak{p}_{i}^{n})=(\alpha^{n})$.因此$A/\mathfrak{p}_i^{e_i}$主理想。\\
    5.设R是Dedekind整环,$I$是R中理想,任取$a\in I\setminus\{0\}$,令$J=Ra,$则$J\subset RI=I,$考虑商环$R/J$,由上题知$R/J$中理想$I/J$是主理想,即有$b\in R$使得$I=Rb+J,$但由于$J=Ra$,故$I=<a,b>$\\
    8.首先$(\mathfrak{a}\mathcal{O}_L)^m = \alpha \mathcal{O}_L = (\sqrt[m]{\alpha}\mathcal{O}_L)^m$,$\mathcal{O}_{L}$中每个理想都有唯一的素理想分解,于是$\mathfrak{a}\mathcal{O}_L=\prod_{i=1}^s \mathfrak{p}_i^{k_i}$,$\mathfrak{p}$是素理想,$k_{i}\in Z,$从而$(\sqrt[m]{\alpha}\mathcal{O}_L)^m = (\mathfrak{a}\mathcal{O}_{L})^m = \prod_{i=1}^s \mathfrak{p}_i^{mk_i}$,进而$\sqrt[m]{\alpha}\mathcal{O}_L = \prod_{i=1}^s \mathfrak{p}_i^{mk_i / m} = \mathfrak{a}\mathcal{O}_L$,取$\beta=\sqrt[m]{\alpha}$即为所证命题。\\
    9.设$|Cl_{K}|=n$(类数有限定理),记$Cl_{K}$的元素为$[I_{1}],\cdots,[I_{n},]$对于每个$1\leq k\leq n$取$J_{k}\in [I_{k}].$存在整数$m_{k}$,$\alpha_{k}\in \mathcal{O}_{K}$使得$J_k^{m_k} = (\alpha_k)$,由上题知$J_{1},\cdots,J_{k}$在域$L = K(\sqrt[m_1]{\alpha_1},\ldots, \sqrt[m_n]{\alpha_n}).$的整数环中都是主理想,剩下的只需验证若$I\subset \mathcal{O}_{K},I\simeq J_{1},$则$I$是$\mathcal{O}_{K}$中主理想,如果$I\simeq J_{1}$,那么存在$x,y\in \mathcal{O}_{K}$使得$xI=yJ_{1}$,于是$$
     x^{m_1}I^{m_1} = y^{m_1}J_1^{m_1}\\
    = (y^{m_1}\alpha_1)
    $$
    因此$xI\mathcal{O}_L = y\sqrt[m_1]{\alpha_1}\mathcal{O}_L$,从而存在$z \in I\mathcal{O}_L$使得$xz = y\sqrt[m_1]{\alpha_1}$,断言$I=(z)$,显然$(z)\subseteq I\mathcal{O}_{L}$,反之,任取$w \in I.$那么$xw = y\sqrt[m_1]{\alpha_1}v = xzv,v\in \mathcal{O}_{L}$,由于$\mathcal{O}_{L}$是整环,因此$zv=w$,所以$I\subseteq (z)$,故$I$是$\mathcal{O}_{L}$中主理想。\\
    10.由于可分扩张$L|K$中只有有限个素理想分歧,故不妨只需证明不分裂且不分歧的素理想只有有限个,设$\mathfrak{p}$是这样的素理想,于是$\mathfrak{p}\mathcal{O}=\mathfrak{P},\mathfrak{P}\in \mathcal{O}.f[\mathcal{O} /\mathfrak{P}:o/\mathfrak{p}]=[L:K].$
    由$\mathfrak{p}$不分裂知$G_{\mathfrak{P}}=G.$再由$\mathfrak{p}$不分歧知$I_{\mathfrak{P}}={1}.$从而$G\cong Gal(\mathcal{O} /\mathfrak{P}|o/\mathfrak{p}).$但有限域扩张$\mathcal{O} /\mathfrak{P}|o/\mathfrak{p}$的伽罗瓦群是循环群,而根据假设G不是循环群,矛盾!于是不存在这样的素理想,即K中不分裂的素理想是分歧的,从而有有限个.\\
    11.域扩张$\kappa(\mathfrak{P})|\kappa(\mathfrak{p})$是伽罗瓦扩张,有下述关系$$
    I_{\mathfrak{P}}=1\iff T_{\mathfrak{P}}=L\iff \mathfrak{p} \quad is \quad unramified \quad in \quad L
    $$
    进而$G(\kappa(\mathfrak{P})|\kappa(\mathfrak{p}))\cong G_{\mathfrak{P}}.$有限域扩张$\kappa(\mathfrak{P})|\kappa(\mathfrak{p})$的伽罗瓦群是循环群,从而$G_{\mathfrak{P}}$是循环群。由有限域的伽罗瓦理论知$Gal(\kappa(\mathfrak{P})|\kappa(\mathfrak{p}))$的生成元是映射$\sigma :x\mapsto x^{q}.(x\in \kappa(\mathfrak{p})),q=|\kappa(\mathfrak{p})|$.在同构对应下,记$\sigma$在$G_{\mathfrak{P}}$
    中对应元素为$\phi_{\mathfrak{P}}.$则
    $$\phi_{\mathfrak{P}}a\equiv a^{q}(mod \mathfrak{P}),\forall a\in \mathcal{O}.$$
    这是由于如果$\tau \in G_{\mathfrak{P}}$并且对于每个$a\in \mathcal{O}$均有$\tau a\equiv a^{q}(mod \mathfrak{P}),$则在同构下$\tau \mapsto \bar{\tau}$,对于每个$\bar{a}\in \bar{L}$均有$\bar{\tau}(\bar{a})=\bar{a}^{q}.$即$\bar{\tau}=\sigma,$从而$\tau=\phi_{\mathfrak{P}}.$\\
    12.设$n\geq 2.$用反证法,若只存在有限个这样的素数,记为$p_{1},\cdots,p_{m},$令$q=\prod_{1\leq i\leq m}p_{i}.$考虑$$\Phi_{n}(xnq),x\in Z,$$ $\Phi_{n}(X)$是n次分圆多项式,显然存在$x\in Z$使得$\Phi_{n}(xnq)>1.$此时存在素数$P$
使得$p|\Phi_{n}(xnq).$ 由于$\Phi_{n}(X)$的常数项为1,且$\Phi_{n}(X)$为整系数多项式,故$p\nmid xnq,$从而$p\neq p_{i}.$因为$xnq\in F_{p}$是n次本原单位根,由拉格朗日定理得$n|p-1,$于是$p\equiv 1(mod\quad n).$矛盾!从而证明命题。\\
    13.$G$是有限$Abel$群,根据有限$Abel$群结构定理,存在$n_{1},\cdots,n_{k}$使得
    $$
    G\cong (\mathbb{Z}_{n_{1}},+)\oplus(\mathbb{Z}_{n_{2}},+)\oplus(\mathbb{Z}_{n_{k}},+)
    $$
对任意$n_{i}(i=1,\cdots,k)$由$Dirichlet$素数定理,存在素数$p_{i}$使得$n_{i}|p_{i}-1,$从而有满射$(\mathbb{Z}/p_{i}\mathbb{Z})^{\times}\rightarrow (\mathbb{Z}_{n_{i}},+)$.
令$n=p_{1}\cdots p_{k},$由中国剩余定理
$$
(\mathbb{Z}/n\mathbb{Z})^{\times}\cong \prod_{i=1}^{k}(\mathbb{Z}/p_{i}\mathbb{Z})^{\times},
$$
于是有典范态射$\phi:(\mathbb{Z}/n\mathbb{Z})^{\times}\rightarrow G,$ $\phi$是满射,$(\mathbb{Z}/n\mathbb{Z})^{\times}/ker(\phi)\cong G$。已知$Gal(\mathbb{Q}(\xi_{n})/\mathbb{Q})=(\mathbb{Z}/n\mathbb{Z})^{\times}$(这里把同构看作相等),这里$\xi_{n}$是$n$次本原单位根,由$Galois$理论知存在$\mathbb{Q}(\xi_{n})$的包含$\mathbb{Q}$的子域$K$使得$Gal(\mathbb{Q}(\xi_{n})/K)= ker(\phi),$由于$Gal(\mathbb{Q}(\xi_{n})/\mathbb{Q})$是$Abel$群,$K$是$\mathbb{Q}$的$Galois$扩张,从而
$$
Gal(K/\mathbb{Q})\cong Gal(\mathbb{Q}(\xi_{n})/\mathbb{Q})/Gal(\mathbb{Q}(\xi_{n})/K)\cong G.
$$
证毕。\\

 15.设$\mathfrak{p}$是K上的素理想,$P_{\mathfrak{p}}$是L中所有卧$\mathfrak{p}$上素理想组成的集合,设$N|K$是$L|K$的正规闭包,令$G=Gal(N|K),H=Gal(N|L).$设$\mathfrak{P}$是N中卧于$\mathfrak{p}$上的一个素理想,$G_{\mathfrak{P}}=\{\sigma \in G|\sigma \mathfrak{P}=\mathfrak{P}\}$
    是$\mathfrak{P}$的分解群,则有G的双陪集$H\setminus G/G_{\mathfrak{P}}$到$P_{\mathfrak{p}}$的双射$$
    H\sigma G_{\mathfrak{P}}\mapsto \sigma \mathfrak{P}\cap L
    $$
    (后文给出证明)$\mathfrak{p}$是完全分裂的等价于$G_{\mathfrak{P}}$是平凡的,因此只需证明$G_{\mathfrak{P}}$是平凡的当且仅当$\mathfrak{p}$在L上完全分裂.\\
    若$G_{\mathfrak{P}}$是平凡的(即$\mathfrak{p}$在N上完全分裂),那么双陪集即是G关于H的陪集,于是由由伽罗瓦理论知$[G:H]=[L:K],$这意味着L中有$[L:K]$个卧于$\mathfrak{p}$上的素理想,因此$\mathfrak{p}$在L上完全分裂.\\
    相反地,如果$\mathfrak{p}$在L上完全分裂,那么双陪集的个数等于$[L:K]=[G:H],$这于H的陪集的个数相同;由于每一个双陪集分解成H的右陪集无交并,从而对于任意$\sigma \in G,H\sigma G_{\mathfrak{P}}=H\sigma,$于是$G_{\mathfrak{p}}$关于G的共轭便包含在H中,即$G_{\mathfrak{p}}$生成的正规子群包含在H中。\\
    但由于$N|K$是$L|K$的正规闭包,H对应于L,由伽罗瓦理论知G无非平凡正规子群,从而$G$中包含在H中的正规子群是平凡的,即是$\{1\},$进而$G_{\mathfrak{P}}=\{1\}$\\
    下面证明$H\sigma G_{\mathfrak{P}}$到$\sigma \mathfrak{P}\cap L$是双射.\\
    首先,映射良定义:若$\tau \in G_{\mathfrak{P}},$那么$\tau \mathfrak{P}=\mathfrak{P},$因此$\sigma \mathfrak{P}\cap L=\sigma \tau \mathfrak{P}\cap L.$如果$\rho\in H,$那么$\rho $固定L中元素,于是$\rho \sigma \mathfrak{P}\cap L=\rho(\sigma \mathfrak{P}\cap L)=\sigma \mathfrak{P}\cap L.$因此$\rho\sigma \tau $与$\sigma $对应相同的集合。\\
    满射:任给$L$中卧于$\mathfrak{p}$上的素理想$\mathfrak{q}$,N中存在素理想$\mathfrak{Q}$卧于$\mathfrak{q}$上,从而存在$\sigma \in G$使得$\sigma \mathfrak{P}=\mathfrak{Q}.$因此$H\sigma G_{\mathfrak{P}}$映射到$\sigma \mathfrak{P}\cap L=\mathfrak{Q}\cap L=\mathfrak{q}.$\\
    单射:若$\sigma\mathfrak{P}\cap L = \phi\mathfrak{P}\cap L = \mathfrak{q}.$那么$\sigma \sigma\mathfrak{P}\cap L = \phi\mathfrak{P}\cap L = \mathfrak{P}$和$\phi_{\mathfrak{P}}$都是卧于$\sigma\mathfrak{P}\cap L = \phi\mathfrak{P}\cap L = \mathfrak{q}$上的素理想,因此存在$\rho \in Gal(N|L)=H$使得$\rho\sigma\mathfrak{P} = \phi\mathfrak{P}.$因此$\phi^{-1}\rho\sigma\mathfrak{P} = \mathfrak{P}.$即$\phi^{-1}\rho\sigma\in G_{\mathfrak{P}}.$因此存在
    $\rho \in G_{\mathfrak{P}}$使得$\tau \sigma \rho ^{-1}=\phi.$故$\phi $属于双陪集$H\sigma G_{\mathfrak{P}}.$从而$H\phi G_{\mathfrak{P}}=H\sigma G_{\mathfrak{P}}.$
    
\section{赋值}
    \subsection{p进数域}
  设p为素数,有理数a的p进赋值$ord_{p}(a)$定义如下:在$a\neq 0$的情况下,将其表示为$a=p^{m}\frac{v}{u}(m\in Z,p\nmid u,p\nmid v)$,$ord_{p}(a)=m.$令$ord_{p}(0)=\infty,$则以下公式成立\\
  $(i)ord_{p}(ab)=ord_{p}(a)+ord_{p}(b).$\\
  $(ii)ord_{p}(a+b)\geq min(ord_{p}(a),ord_{p}(b)).$\\
  (iii)如果$ord_{p}(a)\neq ord_{p}(b),$则$ord_{p}(a+b)=min(ord_{p}(a),ord_{p}(b)).$\\
  定义有理数数列$(x_{n})_{n\geq 1}$按p进收敛于有理数a为当$n\rightarrow \infty $时,$ord_{p}(x_{n}-a)\rightarrow \infty .$\\
  以上"p进收敛"可以看成如下那样的“在度量空间中的收敛”:对于$a\neq 0$定义范数为$$
 |a|_{p}=p^{-ord_{p}(a)}, 
  $$
  $|0|_{p}=0.$
  由此可以定义度量:令有理数a和b之间的距离为$$
  d_{p}(a,b)=|a-b|_{p},
  $$
  可验证$d_{p}$满足正定性,对称性,三角不等式,于是$(Q,d_{p})$成为度量空间.\\
  像有理数集Q利用完备化得到实数集R那样,Q在距离$d_{p}$下也有完备化,记为$Q_{p},$称之为p进数域,其子集$$
  Z_{p}=\{a\in Q_{p}:ord_{p}(a)\geq 0\}.
  $$
  中的元素叫做p进整数。\\
  下面定义逆向极限\\
  \textbf{定义:}当给出集合$X_{n}(n=1,2,\cdots,)$和映射$f_{n}:X_{n+1}\rightarrow X_{n}(n=1,2,\cdots,)$的系统
  $$\cdots X_{4}\rightarrow X_{3}\rightarrow X_{2}\rightarrow X_{1}$$
  时,称乘积集合$\prod_{n\geq 1}X_{n}$的子集合
  $$
  \{(a_{n})_{n\geq 1}\in \prod_{n\geq 1}X_{n}|\forall n\geq 1,f(a_{n+1})=a_{n}\}
  $$
  为该系统的逆向极限(inverse limit),记为$\lim\limits_{\leftarrow}X_{n}.$\\
  在定义中,取$X_{n}=Z/p^{n}Z,$取$f_{n}$为从$Z/p^{n+1}Z$到$Z/p^{n}Z$的自然投射,系统$$
  \cdots\rightarrow Z/p^{4}Z\rightarrow Z/p^{3}Z\rightarrow Z/p^{2}Z\rightarrow Z/pZ
  $$
  的逆向极限为$\lim\limits_{\leftarrow}Z/p^{n}Z.$\\
  \textbf{命题1:}(i)$Z_{(p)}\subseteq Z_{p},$在$Q_{p}$中有$Q\cap Z_{p}=Z_{(p)}.$\\
  (ii)设m是整数,则$$
  p^{m}Z_{p}=\{a\in Q_{p}:ord_{p}(a)\geq m\}.
  $$
  (iii)对于所有整数$m\geq 0,$有$$
  Z/p^{m}Z\cong Z_{(p)}/p^{m}Z_{(p)}\cong Z_{p}/p^{m}Z_{p}.
  $$
  (iiii)$Z_{p}$是$Z$在$Q_{p}$中的闭包。\\
 证明:(i)由定义$Z_{(p)}=\{\frac{a}{b}:a,b\in  Z,p\nmid b\}$,故显然有$Z_{(p)}\subseteq Z_{p},$后半部分,可以证明等号两边相互包含,从而两者相等。\\
 (ii)这是明显的。\\
 (iii)对于第一个同构,考虑映射$$\phi :Z_{(p)}\rightarrow Z/p^{n}Z:\frac{a}{b}\mapsto \frac{a \quad mod \hspace{1em} p^{n} }{b \quad mod \quad p^{n}}(a,b\in Z,p\nmid b).
 $$
 这里注意到$b\quad mod \quad  p^{n}$是$Z/p^{n}Z$中可逆元,映射是满射:$\forall z\in Z/p^{n}Z,\phi(\frac{z}{1})=z.$再有$$
 \phi(\frac{a}{b})=0\iff  \frac{a \quad mod \hspace{1em} p^{n} }{b \quad mod \quad p^{n}}=0\iff a \quad mod \hspace{1em} p^{n}=0\iff \frac{a}{b}\in p^{n}Z_{(p)}
 $$
 于是$ker(\phi)=p^{n}Z_{(p)}.$于是$Z_{(p)}/p^{n}\cong Z/p^{n}Z.$\\
 对于第二个同构,
 注意到$Z_{(p)}\subset Z_{p}.$$Z_{(p)}\cap p^{m}Z_{p}=p^{m}Z_{(p)},$故由嵌入诱导的映射$Z_{(p)}/p^{m}Z_{(p)}\rightarrow Z_{p}/p^{m}Z_{p}$为单射,另外设$a\in Z_{p},$由于$Q$在$Q_{p}$中稠密,故存在$x\in Q$使得$ord_{p}(x-a)\geq m.$由于$x-a\in p^{m}Z_{p},m\geq 0,a\in Z_{p},$故$x\in Q\cap Z_{p}=Z_{(p)},$因此$a=x+(a-x)\in Z_{(p)}+p^{m}Z_{p}.$从而上述映射是满射。\\
 (iiii)用定义验证即可,可见Neukirch, Algebraic number theory p112.\\
 \textbf{命题2:}$$\lim\limits_{\leftarrow}Z/p^{n}Z\cong Z_{p}.$$
 证明:为此需构造两者的映射,首先给出映射$\lim\limits_{\leftarrow}Z/p^{n}Z\rightarrow Z_{p}.$对于每个$n\geq 1,$取整数$x_{n}$使得$x_{n}$的像为$a_{n},$由于当$m,n\geq N$时$x_{m}\equiv x_{n}\quad mod \quad p^{N}$(即$|x_{m}-x_{n}|_{p}\leq \frac{1}{p^{N}}$),故$(x_{n})_{n\geq 1}$是个p进Cauchy序列,从而在$Q_{p}$中收敛.因为对所有的n,$ord_{p}(x_{n})\geq 0,$所以这个极限属于$Z_{p}.$\\
 	下面对于每个正整数n考虑映射
 	$$
 	Z_{p}\rightarrow Z_{p}/p^{n}Z_{p}\rightarrow Z_{(p)}/p^{n}Z_{(p)}\rightarrow Z/p^{n}Z
 	$$
 	后三项由上命题知是同构的,其间的映射是同构映射。$\forall a\in Z_{p},$由于Z在$Z_{p}$中稠密,从而存在$x\in Z$使得$ord_{p}(x-a)\geq n,$即$x-a\in p^{n}Z_{p}.$从而从而上述映射中具体元素对应为
 	$$
 	a\mapsto x+p^{n}Z_{p}\mapsto x\mapsto \phi(x)=x\quad mod \quad p^{n},x\in Z	
 	$$
 	那么由于$a\equiv x\quad mod \quad p^{n},\phi(x)\equiv x\quad mod \quad p^{n} $得到$a\equiv \phi(x)\quad mod\quad p^{n}.$记$\psi_{n}(a):=\phi (x)$从而
 	由此得到的序列$\{\psi_{n}(a)\}$收敛到$a.$这就证明了映射的合成
 	$
 	Z_{p}\rightarrow \lim\limits_{\leftarrow}Z/p^{n}Z\rightarrow Z_{p}
 	$是$Z_{p}$上的恒等映射。\\
 	设$\{x_{n}\}\in \lim\limits_{\leftarrow}Z/p^{n}Z$收敛到$s,$由于当$m\geq n$时$x_{m}\equiv x_{n}\quad mod \quad p^{n}$,即$|x_{m}-x_{n}|_{p}\leq \frac{1}{p^{n}}$,固定n,令m趋于正无穷,由范数的连续性得到$|s-x_{n}|_{p}\leq \frac{1}{p^{n}},$即
 	$x_{n}\equiv s\quad mod \quad p^{n}$对任意n成立,于是在$Z/p^{n}Z$中$x_{n}=\psi_{n}(s).$这说明复合映射$\lim\limits_{\leftarrow}Z/p^{n}Z\rightarrow Z_{p}  \rightarrow \lim\limits_{\leftarrow}Z/p^{n}Z$是恒等映射。\\
 	该命题证明也可见Neukirch, Algebraic number theory p114.\\
 	上面两命题中,$Z_{p}/p^{n}Z_{p}\cong Z/p^{n}Z$也可直接证明得到:考虑映射$a\mapsto a \quad mod p^{n}Z_{p}.$其核为$p^{n}Z_{p}.$是满射,
 	事实上,$\forall a\in Z_{p},$由于Z在$Z_{p}$中稠密,从而存在$x\in Z$使得$ord_{p}(x-a)\geq n,$即$x-a\in p^{n}Z_{p}.$因此$Z_{p}/p^{n}Z_{p}\cong Z/p^{n}Z.$注意到$x\mapsto a$给出了逆映射。\\
 	p进数首先是由Hensel引进的,给出的定义为:\\
\textbf{定义:}对于每一素数p,p进整数是一个形式无穷级数$$
a_{0}+a_{1}p+a_{2}p^{2}+\cdots ,
$$
这里$0\leq a_{i}<p,i=0,1,2,\cdots,$所有p进整数组成的集合记为$Z_{p}.$
后文用到下面命题:\\
\textbf{命题3:}$Z/p^{n}Z$中剩余类$a\quad mod \quad p^{n}$能被 唯一表示成形式$$
a\equiv a_{0}+a_{1}p+a_{2}p^{2}+\cdots+a_{n-1}p^{n-1}mod \quad p^{n}
$$
这里$0\leq a_{i}<p,i=0,\cdots,n-1.$
证明用数学归纳法。这里略去\\
对于每个整数,或更一般地,对于任意$f\in Z_{(p)}.$定义剩余类序列$$
\bar{s_{n}}=f\quad mod p^{n}\in Z/p^{n}Z,n=1,2,\cdots,
$$
由上命题知$s_{n}=a_{0}+a_{1}p+a_{2}p^{2}+\cdots+a_{n-1}p^{n-1},n=1,2,...,$这定义了p进整数$
\sum_{v=0}^{\infty }a_{v}p^{v}\in Z_{p}.
$叫做f的p进展开,类似于洛朗级数,扩展p进整数到形式级数$$
\sum_{v=-m}^{\infty }a_{v}p^{v}=a_{-m}p^{-m}+\cdots+a_{-1}p^{-1}+a_{0}+a_{1}p+\cdots,
$$
这里$m\in Z,0\leq a_{i}<p.$这样的级数叫做p进数,所有这样数组成的集合记为$Q_{p}.$
有理数的p进展开给出了映射$Q\mapsto Q_{p},$将Z映到$Z_{p}$内,若将$Q$与其像等同,则可写$Q\subseteq Q_{p},Z\subseteq Z_{p}.$于是对于$f\in Q$,有等式$f=\sum_{v=-m}^{\infty }a_{v}p^{v}.$\\
令$s_{n}=\sum_{v=0}^{n-1}a_{v}p^{v}\in Z,$其在$Z/p^{n}Z$中的剩余类记为$\bar{s_{n}}=s_{n}mod \quad p^{n}.
$\\
\textbf{命题4:}$f=\sum_{v=0}^{\infty }a_{v}p^{v}\mapsto (\bar{s_{n}}=\sum_{v=0}^{n-1}a_{v}p^{v}mod \quad p^{n})_{n\in N}$是$Z_{p}$到$\lim\limits_{\leftarrow}Z/p^{n}Z$的双射。\\
证明由上命题立知。\\
由于$\lim\limits_{\leftarrow}Z/p^{n}Z$是$\prod_{n=1}^{\infty }$的子环,从而通过同构可赋予$Z_{p}$环结构,使其成为环。因为任意$f\in Q_{p}$,f可表示成$f=p^{-m}g,g\in Z_{p},$从而将加法乘法扩展到$Q_{p}$上,$Q_{p}$便成为$Z_{p}$的分式域。
    \subsection{赋值}
  下面讨论更一般域上的赋值,\\
  \textbf{定义:}域K的一个\textbf{赋值}是一个函数$$
  |\cdot |:K\rightarrow R
  $$
 满足下面性质\\
 (i)$|x|\geq 0,$若$|x|=0\iff x=0,$\\
 (ii)$|xy|=|x||y|,$\\
 (iii)$|x+y|\leq |x|+|y|.$\\
     定义K中两点间的距离是$$
     d(x,y)=|x-y|
     $$
     这是K成为度量空间,因此也成为一拓扑空间。\\
     \textbf{定义:}如果K的两个赋值诱导相同的拓扑空间,那么称它们是等价地。\\
     \textbf{命题:}K的两个赋值$|\cdot|_{1},|\cdot|_{2}$等价当且仅当存在实数$s\textgreater0$使得对任意$x\in K$有$|x|_{1}=|x|_{2}^{s}.$\\
     证明略。\\
     \textbf{逼近定理:}设$|\cdot |_{1},\cdots,|\cdot|_{n}$是K的两两互不等价地赋值,任给$a_{1},\cdots,a_{n}\in K,$那么对任意$\epsilon\textgreater0,$存在$x\in K$使得$$
     |x-a_{i}|_{i}\textless\epsilon,\forall i=1,\cdots,n,
     $$
     证明略。\\
     \textbf{定义:}若对所有$n\in N$,赋值$|n|$有界,则称赋值是非阿基米德的,否则,称为阿基米德的。\\
     \textbf{命题:}赋值是非阿基米德的当且仅当赋值满足强三角不等式$$|x+y|\leq max\{|x|,|y|\}.$$
     注记:由$|-x||-x|=|x^{2}|=|x||x|$得到$|-x|=|x|.$对于任意$|x|\neq |y|,$不妨设$|x|\leq |y|$.首先$|x+y|\leq max\{|x|,|y|\}=|y|,$其次
     $|y|=|x+y-x|\leq max\{|x+y|,|-x|\}= max\{|x+y|,|x|\},$由此推出$|x+y|=|y|=max\{|x|,|y|\}.$\\
     \textbf{命题:}Q的每个赋值等价于赋值$|\cdot|_{p}$或$|\cdot|,$后者是通常的绝对值赋值。\\
     证明略。\\
     设$|\cdot|$是域K的非阿基米德赋值,令$v(x)=-log|x|(x\neq 0),v(0)=\infty.$我们得到函数$V:K\rightarrow R\cup \{\infty\},$它满足下面性质\\
     (i)$v(x)=\infty\iff x=0.$\\
     (ii)v(xy)=v(x)+v(y),\\
     (iii)$v(x+y)\geq min\{v(x),v(y)\},$\\
     这里我们约定对于$a\in R,$若$a\textless\infty ,a+\infty =\infty ,\infty +\infty =\infty .$\\
     定义在K上且满足上面三个条件的函数叫做K的一个指数赋值。我们不考虑函数$v(x)=0(x\neq 0),v(0)=\infty .$\\
     K的两个指数赋值$v_{1},v_{2}$叫做等价的:若$v_{1}=sv_{2},0<s\in R$.
     对于每个指数赋值v,我们可以通过令$|x|=q^{-v(x)}$得到一个赋值,这里q是大于1的实数。为了与v区分,我们称$|\cdot|$叫做相应的乘法赋值,或者绝对值赋值。由上注记知,$v(x)\neq v(y)\Longrightarrow v(x+y)=min \{v(x),v(y)\}$\\
     \textbf{命题:}(i)$o=\{x\in K|v(x)\geq 0\}=\{x\in K||x|\leq 1\}$是K的子环;(ii)其全体可逆元为$o^{*}=\{x\in K|v(x)=0\}=\{x\in K||x|=1\},$(iii)唯一的极大理想是$\mathfrak{p}=\{x\in K|v(x)\textgreater0\}=\{x\in K||x|\textless 1\}.$\\
     证明:(1)是显然的((ii)注意对任意$0\neq x\in K,v(x^{-1})=-v(x)$(iii)这是因为$o-\mathfrak{p}=o^{*}.$而$o^{*}$中元素是可逆元。\\
     上面的$o$是整环,K是其分式域,且有对任意$x\in K,$有$x\in o$或者$x^{-1}\in o.$这样的环叫做赋值环。它唯一的极大理想是$\mathfrak{p}=\{x\in o|x^{-1}\notin o\}.$域$o/\mathfrak{p}$叫做o的剩余类域。赋值环是整闭的:若$x\in K$在o上是整的,则有方程式$$
     x^{n}+a_{1}x^{n-1}+\cdots+a_{n}=0,a_{i}\in o
     $$
     假设$x\notin o,$那么$x^{-1}\in o,$从而$x=-a_{1}-a_{2}x^{-1}-\cdots-a_{n}(x^{-1})^{n-1}\in o,$矛盾,这说明$x\in o.$\\
      指数赋值v称为离散的,如果存在正实数s,使得$v(K^{*})=sZ.$如$s=1,$则称为正则的,通过除以s,我们总可以由离散赋值得到正则离散赋值,这不改变$o,o^{*},\mathfrak{p}.$假设已经这样做了,o中元素$\pi \in o,v(\pi)=1$叫做素元,任意元素$x\in K^{*}$有唯一的分解$$
      x=u\pi^{*},m \in Z,u\in o^{*}
            $$
     这是因为若$v(x)=m,$那么$v(x\pi^{-m})=0,$因此$u=x\pi^{-m}\in o^{*}.$\\
     \textbf{命题:}如果v是K的一个离散指数赋值,那么$$
     o=\{x\in K|v(x)\geq 0\}
     $$
     是主理想整环,因此是离散赋值环(其定义是:有唯一极大理想的主理想整环)\\
     假设v是正则的,那么o的非零理想由$$
     \mathfrak{p}^{n}=\pi^{n}o=\{x\in K|v(x)\geq n\},n\geq 0
     $$
     给出,这里$\pi$是素元,即$v(\pi)=1$.有$$
     \mathfrak{p}^{n}/\mathfrak{p}^{n+1}\cong o/\mathfrak{p}.
     $$
     证明:设$\mathfrak{a}\neq 0$是o的一个理想,$x\neq $是o中具有最小赋值的元素,设$v(x)=n.$那么$x=u\pi^{n},u\in o^{*},$于是$\pi^{n}o\subseteq \mathfrak{a}.$如果$y=\epsilon \pi^{m}\in \mathfrak{a},\epsilon\in o^{*}$是$\mathfrak{a}$中任意元素,那么$m=v(y)\geq n$,因此$y=(\epsilon \pi^{m-n})\pi^{n}\in \pi^{n}o,$因此$\mathfrak{a}=\pi^{n}o.$同构$$
     \mathfrak{p}^{n}/\mathfrak{p}^{n+1}\cong o/\mathfrak{p}
     $$来自于映射$a\pi^{n}\mapsto a\quad mod \quad \mathfrak{p}.$\\
     在离散赋值域K中,链$$
     o\supseteq \mathfrak{p}\supseteq \mathfrak{p}^{2}\supseteq \mathfrak{p}^{3}\supseteq \cdots .
     $$
     组成了零元素的邻域基。事实上,如果v是正则离散赋值,$|\cdot|=q^{-v}(q>1)$是相应的乘法赋值,那么$$
     \mathfrak{p}^{n}=\{x\in K||x|\textless\frac{1}{q^{n-1}}\}.
     $$
     相似的$1$在$K^{*}$有链$$
     o^{*}\supseteq U^{(0)}\supseteq U^{(1)}\supseteq U^{(2)}\supseteq\cdots .
     $$
     组成邻域基。这里$$
     U^{(n)}=1+\mathfrak{p}^{n}=\{x\in K^{*}||1-x|\textless\frac{1}{q^{n-1}}\},n>0
     $$
     注意到$1+\mathfrak{p}^{n}$在乘法运算下是闭的:如果$x\in U^{(n)},$那么$|1-x^{-1}|=|x|^{-1}|x-1|=|1-x|\textless\frac{1}{q^{n-1}}$,
于是$x^{-1}\in U^{(n)}.$     
\textbf{命题:}对于$n\geq 1,$$o^{*}/U^{(n)}\cong (o/\mathfrak{p}^{n})^{*}.$($(o/\mathfrak{p}^{n})^{*}$表示$o/\mathfrak{p}^{n}$的乘法群.)  $U^{(n)}/U^{(n+1)}\cong o/\mathfrak{p}.$\\
证明:第一个同构由$$
o^{*}\rightarrow (o/\mathfrak{p}^{n})^{*},u\mapsto u\quad mod \quad \mathfrak{p}^{n},
$$诱导,易见映射是满射,若$u\in o^{*}$,且$u\equiv 1 \quad \mathfrak{p}^{n},$则$u\in 1+\mathfrak{p}^{n}=U^{(n)}.$从而该映射核为$U^{(n)}.$\\
对于第二个同构,一旦选定素元$\pi$,映射$$
U^{(n)}=1+\pi^{n}o\rightarrow o/\mathfrak{p},1+\pi^{n}a\mapsto a\quad mod \quad \mathfrak{p},
$$
的核为$U^{(n+1)},$且为满同态。\\
    \subsection{完备化}
    
    开头先写下上节中的一些记号,下面将用到
    $$o=\{x\in K|v(x)\geq 0\}=\{x\in K||x|\leq 1\},$$
    $$o^{*}=\{x\in K|v(x)=0\}=\{x\in K||x|=1\},$$
    $$\mathfrak{p}=\{x\in K|v(x)\textgreater0\}=\{x\in K||x|\textless 1\}.$$
    \textbf{定义:}赋值域$(K,|\cdot|)$称为完备的,若K中每个Cauchy列$\{a_{n}\}_{n\in \mathbb{N}}$收敛到K中元素a,即$\lim\limits_{n\rightarrow \infty }|a_{n}-a|=0.$\\
    对于任何赋值域$(K,|\cdot|)$,我们可以通过完备化得到完备域$(\widehat{K},|\cdot|).$若$(\widehat{K^{'}},|\cdot|^{'})$是一个以$K$为稠密子集的完备域,则存在K-同构$$\sigma :\widehat{K}\rightarrow  \widehat{K^{'}}$$
    $$|\cdot|-\lim\limits_{n\rightarrow \infty }a_{n}\mapsto |\cdot|^{'}-\lim\limits_{n\rightarrow \infty }a_{n}$$
    其中$|\cdot|-\lim\limits_{n\rightarrow \infty }a_{n}$表示$\{a_{n}\}$在$\widehat{K}$中的极限,$|\cdot|^{'}-\lim\limits_{n\rightarrow \infty }a_{n}$表示$\{a_{n}\}$在$\widehat{K^{'}}$中的极限。这样$|a|=|\sigma a|^{'}.$这里注意完备域中的范数是由原范数的扩张。\\
    \textbf{定理(Ostrowski)}设K是具有阿基米德赋值$|\cdot|$的完全域,那么存在从K到$\mathbb{R}$或$\mathbb{C}$的的同构$\sigma $满足$$
    |a|=|\sigma a|^{s},\forall a\in K
    $$
    这里常数$s\in (0,1].$\\
    可以说上述定理已经说明了具有阿基米德赋值的完备域的结构,下面我们将聚焦于域的非阿基米德赋值,为了方便考虑指数赋值,设$v$是域K的指数赋值,$\widehat{K}$是K的完备化,$\forall a\in \widehat{K},$令$\widehat{v}(a)=\lim\limits_{n\rightarrow \infty }v{a_{n}},$这里$a=\lim\limits_{n\rightarrow \infty }a_{n}\in \widehat{K},a_{n}\in K.$于是获得$\widehat{K}$的一个指数赋值。这里注意到,存在$n_{0},$使得当$n>n_{0}$时,$\widehat{v}(a-a_{n})>\widehat{v}(a)(\lim\limits_{n\rightarrow \infty}a_{n}=a,v(0)=\infty ).$由上节的注记知$v(a_{n})=\widehat{v}(a_{n}-a+a)=min\{\widehat{v}(a_{n}-a),\widehat{v}(a)\}=\widehat{v}(a).$因此$v(K^{*})=\widehat{v}(\widehat{K^{*}})$.进而有若$v$是正则离散的,$\widehat{v}$也是正则离散的.\\
    与$(\mathbb{Q},v_{p})$类似,有下面命题.\\
    \textbf{命题}$o\subseteq K,\widehat{o}\subseteq \widehat{K}$分别是关于$v,\widehat{v}$的赋值环,$\mathfrak{p},\widehat{\mathfrak{p}}$是极大理想,那么有$$
    \widehat{o}/\widehat{\mathfrak{p}}\cong o/\mathfrak{p}
    $$
    并且若$v$是离散的,进一步有$$
    \widehat{o}/\widehat{\mathfrak{p}}^{n}\cong o/\mathfrak{p}^{n},n\geq 1.
    $$
    证明:考虑映射$o\rightarrow \widehat{o}/\widehat{\mathfrak{p}},x\mapsto x\quad mod \quad \widehat{\mathfrak{p}}$,该映射核为$\mathfrak{p},$且为满射,事实上,任意$x\in\widehat{o},$存在$a\in o$使得$\widehat{v}(x-a)>0$,即$x-a\in \widehat{\mathfrak{p}},$从而$x\equiv a\quad mod \quad \widehat{\mathfrak{p}}$,此即证明满射。\\
    若$v$是离散的,即$v(K^{*})=sZ,$不妨设$s=1,$若不然,则考虑赋值$v/s,$总可化为正则离散赋值,此时,由上节命题知,$\mathfrak{p}^{n}=\pi ^{n}o=\{x\in K|v(x)\geq n\},n\geq 0,$这里$\pi $是o中素元,即$v(\pi )=1.$注意到$\pi$同样也是$\widehat{K}$中素元,$\widehat{v}(\pi )=1,$同样$\widehat{\mathfrak{p}}^{n}=\pi^{n}\widehat{o}=\{x\in \widehat{K}|\widehat{v}(x)\geq n\}.$像上一部分那样可证明映射$$
    o\rightarrow \widehat{o}/\widehat{\mathfrak{p}}^{n}
    $$
    $$
    x\mapsto x\quad mod\quad \widehat{\mathfrak{p}}^{n}
    $$
    是满射,且核为$\mathfrak{p}^{n}.$\\
    若$v$是K的离散赋值,我们有下命题。\\
    \textbf{命题:}设$R\subseteq o$是陪集$\kappa =o/\mathfrak{p}$的所有代表元组成的集合,且$0\in R,$$\pi\in o$是素元,那么对于任意$0\neq x\in \widehat{K},$存在唯一的收敛级数表示
    $$
    x=\pi^{m}(a_{0}+a_{1}\pi +a_{2}\pi^{2}+\cdots ),a_{i}\in R,a_{0}\neq 0,m\in \mathbb{Z}.
    $$
    证明:设$x=\pi^{m}u,u\in \widehat{o}^{*},$由于上面命题知$\widehat{o}=o+\widehat{\mathfrak{p}}$,因此存在$a\in o,b\in \widehat{o}$使得$x=a+\pi b,$取$a_{0}\in R$使得$a=a_{0}+\pi b^{'},b^{'}\in o$则$x=a_{0}+\pi(b^{'}+b),$
令$b_{1}=b^{'}+b$,则$x=a_{0}+\pi b_{1},a_{0}\in R,b_{1}\in \widehat{o},$  注意到由于$u\in \widehat{o}^{*},$则$a_{0}\neq 0 $,否则$u=\pi b_{1}\in \widehat{\mathfrak{p}}$,矛盾!再注意到$R\cap \widehat{\mathfrak{p}}=\{0\}$(注意两者中元素的指数赋值),从而上述表法唯一。\\    
 下面假设$a_{0},\cdots,a_{n-1}\in R$已经确定,使得$$
 u=a_{0}+a_{1}\pi +\cdots +a_{n-1}\pi^{n-1}+\pi^{n}b_{n},b_{n}\in\widehat{o}
 $$
 且$a_{i}$是唯一的,和上面一样,$b_{n}$能被唯一的写成$b_{n}=a_{n}+\pi b_{n+1},b_{n+1}\in \widehat{o}.$
 因此$$
 u=a_{0}+a_{1}\pi +\cdots +a_{n-1}\pi^{n-1}+\pi^{n}a_{n}+\pi^{n+1}b_{n+1}
 $$
 继续下去,我们可得到唯一的级数$\sum_{v=0}^{\infty}a_{v}\pi^{v},$且收敛到$u$(注意到余项$\pi^{n+1}b_{n+1}$收敛到0).\\
 对于每个n,有自然同态$o\longrightarrow o/\mathfrak{p}^{n}$并且有同态链$$
 o/\mathfrak{p}\longleftarrow o/\mathfrak{p}^{n}\longleftarrow o/\mathfrak{p}^{3}\longleftarrow \cdots,
 $$
 逆极限是$\lim\limits_{\leftarrow_{n}}o/\mathfrak{p}^{n}=\{(x_{n})\in \prod_{n=1}^{\infty }o/\mathfrak{p}^{n}|\lambda_{n}(x_{n+1})=x_{n}\}.
 $
 这里$\lambda_{n}$是自然同态。我们有下面命题\\
 \textbf{命题:}同态$o\longrightarrow \lim\limits_{\leftarrow_{n}}o/\mathfrak{p}^{n},o^{*}\longrightarrow \lim\limits_{\leftarrow_{n}}o^{*}/U^{(n)}$是同构。\\
 证明:映射是单射是由于$\cap_{n=1}^{\infty}\mathfrak{p}^{n}=\{0\},$下面证明满射,由上一命题证明过程知,元素$a\quad mod \quad \mathfrak{p}^{n}$能被唯一表示成形式$$
 a\equiv a_{0}+a_{1}\pi+\cdots+a_{n-1}\pi^{n-1}\quad mod \quad \mathfrak{p}^{n},a_{i}\in R
 $$
 每个$s\in\lim\limits_{\leftarrow_{n}}o/\mathfrak{p}^{n}$因此由求和式组成的序列$$
 s_{n}=a_{0}+a_{1}\pi +\cdots +a_{n-1}\pi^{n-1},n=1,2,\cdots,
 $$
 组成,这里$a_{i}\in R$是固定的,因此$s$是$x=\lim_{n\rightarrow \infty}s_{n}=\sum_{v=0}^{\infty}a_{v}\pi^{v}\in o$的像。第二个同构是$$
 o^{*}\cong(\lim\limits_{\leftarrow_{n}}o/\mathfrak{p}^{n})^{*}\cong \lim\limits_{\leftarrow_{n}}(o/\mathfrak{p}^{n})^{*}\cong \lim\limits_{\leftarrow_{n}}o^{*}/U^{(n)}.
 $$
 我们的目标是研究完备赋值域K的有限扩张$L|K$,为此必需考虑代数方程的分解因式问题,下面设$K$是非阿基米德完备赋值域,$o$是赋值环,$\mathfrak{p}$ 是极大理想,即剩余类域为$\kappa=o/\mathfrak{p}.$定义多项式$f(x)=a_{0}+a_{1}x+\cdots+a_{n}x^{n}\in o[x]$的范数为$|f|=max\{|a_{0}|,\cdots,|a_{n}|\}$,我们说$f(x)$是本原的,如果$f(x)\equiv 0\quad mod \mathfrak{p},$即$\exists a_{i}\notin \mathfrak{p}$也即是$|f|=max\{|a_{0}|,\cdots,|a_{n}|\}=1.$\\
 \textbf{Hensel 引理:}如果本原多项式$f(x)\in o[x]$有模$\mathfrak{p}$分解$$
 f(x)\equiv \bar{g}(x)\bar{h}(x)\quad mod \quad \mathfrak{p}
 $$
 其中$\bar{g},\bar{h}\in \kappa[x]$是互素多项式,那么$f(x)$有因式分解$f(x)=h(x)g(x),g,h\in o[x]$且$$deg(g)=deg(\bar{g}),g(x)\equiv \bar{g}(x)\quad \mathfrak{p},h(x)\equiv \bar{h}(x)\quad mod\quad \mathfrak{p}.$$
 证明:设$d=deg(f),m=deg(\bar{g}),$那么$d-m\geq deg(\bar{h}).$设$g_{0},h_{0}\in o[x]$,且$g_{0}\equiv \bar{g}\quad mod \quad \mathfrak{p},h_{0}\equiv \bar{h}\quad mod\quad \mathfrak{p},deg(g_{0})=m,deg(h_{0})\leq d-m.$因为$(\bar{g},\bar{h})=1,$存在多项式$a(x),b(x)\in o[x]$满足$ag_{0}+bh_{0}\equiv 1\quad mod \quad \mathfrak{p}.$从而两个多项式$f-g_{0}h_{0},ag_{0}+bh_{0}-1$均属于$\mathfrak{p}[x],$令$\epsilon=max\{|f-g_{0}h_{0}|,|ag_{0}+bh_{0}-1|\},$若$\epsilon=0,$那么$f=g_{0}h_{0},$证毕,从而考虑$\epsilon\neq 0$,此时有两个多项式中的某个系数设为$\pi ,$使得$|\pi|=\epsilon.$从而$\pi^{-1}(f-g_{0}h_{0})\in o[x],\pi^{-1}(ag_{0}+bh_{0}-1)\in o[x]$
 (这里注意到若$f_{i}$是多项式$f-g_{0}h_{0}$的系数,那么$|\pi^{-1}f_{i}|=|\pi^{-1}||f_{i}|\leq |\pi^{-1}||\pi|=1$,从而$\pi^{-1}f_{i}\in o$)\\
 下面说明证明的想法,注意到若g和h若有以下形式$$g=g_{0}+p_{1}\pi+p_{2}\pi^{2}+\cdots,$$
 $$
 h=h_{0}+q_{1}\pi+q_{2}\pi^{2}+\cdots,
 $$
 这里$p_{i},q_{i}\in o[x]$且$deg(p_{i})<m,deg(q_{i})\leq d-m.$我们下面逐步决定多项式$g_{n-1}=g_{0}+p_{1}\pi+\cdots+p_{n-1}\pi^{n-1},
 h_{n-1}=h_{0}+q_{1}\pi+\cdots+q_{n-1}\pi^{n-1},$那么$f\equiv g_{n-1}h_{n-1}\quad mod\quad \pi^{n} $(意思是存在多项式$k(x)\in o[x]$使得$f-g_{n}h_{n}=\pi^{n}k(x)$.)令n趋于无穷大,则$f=gh.$\\
 下面我们便开始构造上述形式,对于n=1,由上分析已经存在(即$f-g_{0}h_{0}=(\pi^{-1}(f-g_{0}h_{0}))\pi$)。\\
 设我们已经对于n证明了上式$f\equiv g_{n-1}h_{n-1}\quad mod\quad \pi^{n}.$下面我们用待定系数确定$p_{n},q_{n}.$
 设$g_{n}=g_{n-1}+p_{n}\pi^{n},h_{n}=h_{n-1}+q_{n}\pi^{n},$那么$f_{n}-g_{n}h_{n}\equiv (g_{n-1}q_{n}+h_{n-1}q_{n}\pi^{n})\quad mod \quad \pi^{n+1}.$两边整除$\pi^{n},$得到$$
 g_{n-1}q_{n}+h_{n-1}p_{n}\equiv g_{0}q_{n}+h_{0}p_{n}\equiv f_{n}\quad mod \quad \pi 
 $$
 这里$f_{n}=\pi^{-n}(f-g_{n-1}h_{n-1})\in o[x].$因为$g_{0}a+h_{0}b\equiv 1\quad mod \quad \pi.$因此有$$
 g_{0}af_{n}+h_{0}bf_{n}\equiv f_{n}\quad mod\quad \pi .
 $$
 由于$g_{0}\equiv g\quad mod \quad \mathfrak{p}$且$deg(g_{0})=deg(\bar{g})$得到$g_{o}$的最高项系数是o中可逆元,从而类似于域中带余除法存在$q(x)\in o[x],p_{n}\in o[x]$使得$b(x)f_{n}(x)=q(x)g_{0}(x)+p_{n}(x),deg(p_{n})<deg(g_{0})=m$(这里只需回忆带余除法的证明过程即知),于是$$
 g_{0}(af_{n}+h_{0}q)+h_{0}p_{n}\equiv f_{n}\quad mod\quad \pi 
 $$
 省略$af_{n}+h_{0}q$中能被$\pi$的整除的系数得到多项式$q_{n}.$那么$g_{0}q_{n}+h_{0}p_{n}\equiv f_{n}\quad mod\quad \pi$,这里由于$deg(f_{n})\leq d,deg(h_{0}p_{n})<(d-m)+m=d,deg(g_{0})=m$得到$deg(q_{n})\leq d-m.$证毕。\\
 例:多项式$x^{p-1}-1\in \mathbb{Z}_{p}[x]$在剩余类域$\mathbb{Z}_{p}/p\mathbb{Z}_{p}=\mathbb{F}_{p}$分解成不同的线性因此,因此由Hensel引理,它在$\mathbb{Z}_{p}$中也分解成不同的线性因子,从而$\mathbb{Q}_{p}$包含(p-1)次单位根。\\
 \textbf{推论:}设域$K$是非阿基米德赋值$|\cdot|$完备域,对于每个不可约多项式$f(x)=a_{0}+a_{1}x+\cdots+a_{n}x^{n}\in K[x],a_{0}a_{n}\neq 0,$那么$|f|=max\{|a_{0}|,|a_{n}|\}.$特别地,$a_{n}=1,a_{0}\in o$暗示$f\in o[x].$\\
 证明:乘以K中合适的元素,我们就可以假设$f\in o[x],|f|=1.$设$a_{r}$是$a_{0},\cdots,a_{n}$中第一个使得$|a_{r}|=1$的系数,那么我们有
 $$
 f(x)\equiv x^{r}(a_{r}+a_{r+1}x+\cdots+a_{n}x^{n-r})\quad mod \mathfrak{p}.
 $$
 如果$max\{|a_{0}|,|a_{n}|\}<1,$那么$0<r<n,$这与Hensel引理矛盾。\\
 从这个推论中我们能导出下面赋值扩张的定理\\
 
 \section{抽象类域论}
 \subsection{无限Galois扩张}
   设$K|k$是无限Galois扩张,一般我们就取$K$是$k$的代数闭包。记$G=Gal(K|k),$对于中间域$k\subset E\subset K$记$H_{E}=Gal(K|E).$定义集合
   $\mathcal{I}=\{E:E$是$K|k$的中间域,且$E|k$是有限Galois扩张$\}$.
   $\mathcal{N}=\{H:H=Gal(K|E),E\in \mathcal{I}\}$.\\
   \textbf{命题1:}$(1)\cap_{H\in\mathcal{N}}H=\{e\}.(2)\cap_{H\in \mathcal{N}}\sigma H=\{\sigma\}(\forall \sigma \in G)$.\\
   证明:(1)任取$\sigma \in \cap_{H\in\mathcal{N}}H$,对任意$\alpha\in K$,设$E$是$k(\alpha)|k$在$K|k$中的正规闭包,则$E\in \mathcal{I},H_{E}=Gal(K|E)\in \mathcal{N}$,特别地$\sigma \in H_{E}$,对$\alpha \in E,\sigma(\alpha)=\alpha,$即$\sigma $在$K$是恒等映射.\\
   (2)$\cap_{H\in \mathcal{N}}\sigma H=\sigma \cap_{H\in\mathcal{N}}H={\sigma}$.\\
   \textbf{命题2:}设$H_{1},H_{2}\in \mathcal{N},$则$H_{1}\cap H_{2}\in \mathcal{N}.$\\
   证明: 由$\mathcal{N}$的定义,存在$E_{1},E_{2}\in \mathcal{I}$使得$H_{1}=Gal(K|E_{1}),H_{2}=Gal(K|E_{2}).$由于$E_{1}E_{2}|k$是有限Galois扩张$E_{1}E_{2}\in \mathcal{I}.$由Galois理论知$H_{1}\cap H_{2}=Gal(K|E_{1}E_{2})$于是$H_{1}\cap H_{2}\in \mathcal{N}.$\\
   定义$G$上的$Krull$拓扑:规定$\{\sigma H:\sigma \in G,H\in \mathcal{N}\}$为$G$上的一个拓扑基。即$G$中子集$H^{'}$为开集当且仅当$H^{'}$为上述拓扑基元素之并。\\
   \textbf{定理:}$G$在上述拓扑基下为Hausdorff,紧致且完全不连通的拓扑群。\\
   证明:(i)完全不连通。\\
     设$X\subset  G$,且$|X|\geq 2$,取$\sigma,\tau\in X,$且$\sigma \neq \tau.$
     由$\cap_{H\in \mathcal{N}}\sigma H=\{\sigma\}$知$\tau \notin \cap_{H\in \mathcal{N}}\sigma H$,从而$\exists H_{0}\in \mathcal{N}$使得$\tau \notin \sigma H_{0},$即
     $\tau\in G-\sigma H_{0}$
    注意到
    $$X=X\cap G=X\cap (\sigma H_{0}\cup (G-\sigma H_{0}))=(X\cap \sigma H_{0})\cup (X\cap (G-\sigma H_{0}))$$
   $G$关于子群$H$有陪集分解$G=\cup_{i\in I}\sigma_{i}H$,由此知若$H$是开集,由于$G$是拓扑群,对任意$\sigma \in G,$$\sigma H$为开集,从而$H$为其所有非平凡陪集的补集,为闭集。
   注意到$\sigma \in X\cap \sigma H_{0} ,\tau \in X\cap (G-\sigma H_{0}),$且$
   \sigma H_{0},G-\sigma H_{0}$均为开集,这就得到$X$是完全不连通的。特别地,$G$是完全不连通的,此处还可以看出,$G$是hausdorff空间。\\
   
   另证:若$\sigma,\tau \in G$且$\sigma \neq \tau,$则存在有限Galois子扩张$E|k$使得$\sigma|_{E}\neq \tau|_{E}$(注意到任取$x\in K$,必存在包含$x$的$K|k$的有限Galois子扩张$E|k$,例如$E$取$k(x)|k$在$K|k$中的代数闭包。若对任意有限Galois子扩张$E|k$有$\sigma|_{E}=\tau|_{E},$则对任意$x\in K,\sigma(x)=\tau(x)$.矛盾!)因此$\sigma Gal(K|E)\neq \tau Gal(K|E),$因此$\sigma Gal(K|E)\cap \tau Gal(K|E)=\emptyset.$\\
   对于$G$的紧性,较快的证明需用逆向极限,这里先不证明。\\

     注:设$G$关于闭子群$H$有陪集分解$G=\cup_{i\in I}\sigma_{i}H$,
     则由$G$的紧致性,$H$是$G$的开子集当且仅当$(G:H)$有限。\\
   \textbf{定理:}设$H\leq G$,记$H^{'}=Gal(K|K^{H}),$则$H^{'}=\bar{H}$($H$在$G$中的闭包.)\\
   证明:显然,$H\leq H^{'}.$下证$H^{'}$为$G$中的闭集,只需证$G-H^{'}$为开集.\\
   
   任取$\sigma\in G-H^{'}$,必有$\alpha \in K^{H}$使得$\sigma(\alpha)\neq \alpha.$对于$\alpha\in K,$有$E\in\mathcal{I}$使得$\alpha\in E,$于是取$H_{0}=Gal(K|E)\in\mathcal{N}.$对于$\forall\tau\in H_{0},$有$\tau\alpha=\alpha ,$于是$\sigma(\tau\alpha)=\sigma\alpha\neq \alpha $,即$$\sigma\tau(\alpha)\neq\alpha 
 \Rightarrow\sigma\tau\in G-H^{'}\Rightarrow\sigma H_{0}\in G-H^{'}\Rightarrow G-H^{}\quad is\quad open\Rightarrow H^{'}is\quad  closed.$$
 下证$\bar{H}=H^{'}.$需证$\forall \sigma \in H^{'},N\in \mathcal{N}.$都有$\sigma N\cap H\neq \emptyset.$\\
 由定义,取$E\in \mathcal{I}$使得$N=Gal(K|E)$,令$H_{0}=\{\rho|_{E}:\rho\in H\},$于是$K^{H_{0}}=K^{H}\cap E,$由有限Galois基本定理到$H_{0}=Gal(E|K^{H}\cap E),$由$\sigma \in H^{'},\sigma|_{K^{H}}=id,$因此$\sigma|_{E}\in H_{0}.$存在$\rho\in H$使得$
 \rho|_{E}=\sigma|_{E}.$于是$\sigma^{-1}\rho \in Gal(K|E)=N$,即$\rho\in \sigma N\cap H.$$\sigma N\cap H\neq \emptyset.$\\
   
   \textbf{命题:}设$K|k$是无限Galois扩张,任取$K|k$的一个中间域,则$H_{E}=Gal(K|E)$是$G$的一个闭子群。\\
   证:$H_{E}\leq G,$则$K^{Gal(K|E)}=E\Rightarrow H_{E}=Gal(K|E)=Gal(K|K^{H_{E}})=\bar{H_{E}}.$\\
   \textbf{无限Galois扩张基本定理:}设$K|k$是无限Galois扩张,令$G=Gal(K|k),$
   \subsection{Hilbert定理90和群的上同调}
    设$K$是一个域,$G$为群,交叉态射$f:G\rightarrow K^{*}$是指一个函数$f$满足对任意的$\sigma ,\tau\in G,f(\sigma\tau)=f(\sigma)\cdot\sigma(f(\tau)).$\\
    \textbf{定义:}设$G$是一个群,$K$是域,一个特征是一个从$G$到$K^{*}$的群同态。\\
    
    若令上面定义中$G=K^{*}$,我们能看出任何一个域$K$关于其子域$F$的$F-$自同态都是一个特征。\\
    \textbf{Dedekind引理:}设$\tau_{1},\cdots,\tau_{n}$是$G$到$K^{*}$的$n$个不同的特征。则$\tau_{i}$在$K$上是线性独立的;即若$\sum_{i}c_{i}\tau_{i}(g)=0(c_{i}\in K)$对任意$g\in G$成立,则$c_{i}=0.$\\
    证明略。可查阅相关代数书。\\
   \textbf{命题:}设$K|F$是Galois扩张,令$G=Gal(K|F),$设$f:G\rightarrow K^{*}$是交叉态射,则存在$a\in K$使得对任意$\sigma \in G$有$f(\tau)=\tau(a)/a.$\\
   证明:因$f(\sigma)\neq 0$对任意$\sigma\in G$成立,故由Dedekind无关性引理存在$c\in K$,使得$\sum_{\sigma \in G}f(\sigma)\sigma(c)\neq 0.$令
   $b=\sum_{\sigma \in G}f(\sigma)\sigma(c).$
   则$\tau(b)=\sum_{\sigma \in G}\tau(f(\sigma))(\tau\sigma)(c),$于是
   $$
   f(\tau)\tau(b)=\sum_{\sigma \in G}f(\tau)\tau(f(\sigma))(\tau\sigma)(c)
   =\sum_{\sigma \in G}f(\tau\sigma)(\tau\sigma)(c)=b.
   $$
   此即$f(\tau)=b/\tau(b).$令$a=b^{-1}$即得到结论。\\
   \textbf{Hilbert定理90:}设$K|F$是循环Galois扩张,$\sigma$是$Gal(K|F)$的生成元。若$u\in K,$则$N_{K|F}(u)=1$当且仅当
   存在$a\in K$使得$u=\sigma(a)/a$成立.\\
   
   证明:有一侧是显然的。若$u=\sigma(a)/a,$则$N_{K|F}(\sigma(a))=N_{K|F}(a),$因此$N(u)=1.$
   反过来,如果$N_{K|F}(u)=1,$定义映射$f:G\rightarrow K^{*}$,
   令$f(id)=1,f(\sigma)=u,f(\sigma^{i})=u\sigma(u)\cdots\sigma^{i-1}(u)(i<n)$.
   若说明$f$是交叉映射,则由上述命题,存在$a\in K$使得
   $f(\sigma^{i})=\sigma^{i}(a)/a$对所有$i$成立,从而$u=f(\sigma)=\sigma(a)/a.$\\
   
   
   
     \section{附录}
 \subsection{Gauss互反律}
   设$p$是奇素数,a为不能被p除尽的整数;二次剩余记号$(\frac{a}{p})\in \{1,-1\}$定义为当在$F_{p}$中存在a的平方根时(即$x^{2}\equiv a \quad mod \quad p$的整数x存在时)令$(\frac{a}{p})=1,$当其不存在时,$(\frac{a}{p})=-1.$该记号称为Legendre记号。满足的性质有\\
   (i)$(\frac{a}{p})\equiv a^{\frac{p-1}{2}}mod \quad p.$\\
   (ii)$(\frac{ab}{p})=(\frac{a}{p})(\frac{b}{p}).$\\
   (iii)(Gauss互反律.)对于两个不同的奇素数$p,q$,$(\frac{q}{p})(\frac{p}{q})=(-1)^{\frac{p-1}{2}\frac{q-1}{2}}.$\\
 下面引入Hilbert记号。\\
 首先做些准备.对于素数p,定义$Q$的子环$Z_{(p)}$为$$
 Z_{(p)}=\{\frac{a}{b}:a,b\in Z,p\nmid b\}.
 $$
 $Z_{(p)}$中的可逆元全体$(Z_{(p)})^{\times}$等于$\{\frac{a}{b}:p\nmid a,p\nmid b\}.$非零有理数可以唯一表示成$p^{m}u(m\in Z,u\in (Z_{(p)})^{\times}).$
 对于素数p与$a,b\in Q^{\times},$我们来定义Hilbert记号$(a,b)_{p}.$记
 
 
 
 
    \textbf{中国剩余定理:}$A_{1},\cdots,A_{n}$是环$\mathcal{O}$的理想,且有$A_{i}+A_{j}=\mathcal{O},i\neq j$.令$A=\cap_{i=1}^{n}A_{i}$.则有
    $$
    \mathcal{O}/A\cong \oplus_{i=1}^{n}\mathcal{O}/A_{i}
    $$
    证明即是考虑映射$\mathcal{O}\rightarrow\oplus_{i=1}^{n}\mathcal{O}/A_{i},a\mapsto \oplus_{i=1}^{n}a\quad mod \quad a_{i}. $映射的核为$A=\cap_{i=1}^{n}A_{i},$剩下只须证明是满射。\\
    下面是一个类似的命题\\
    \textbf{命题:}如果$A\neq 0$是$\mathcal{O}$的理想,那么$$
    \mathcal{O}/A\cong \oplus_{P}\mathcal{O}_{P}/A\mathcal{O}_{P}=\oplus_{P\supseteq A}\mathcal{O}_{P}/A\mathcal{O}_{P}
    $$
    证明:令$\bar{A_{P}}=\mathcal{O}\cap A\mathcal{O}_{P}.$除有限个素理想外,都有$P\nsupseteqq A$,因此$A\mathcal{O}_{P}=\mathcal{O}_{P}$(两者相互包含),从而$\bar{A_{P}}=\mathcal{O}$,进一步分析可知$A=\cap_{P}\bar{A}_{P}=\cap_{P\supseteq A}\bar{A}_{P}.$事实上:任意$a\in \cap_{P}\bar{A}_{P},$理想$B=\{x\in \mathcal{O}|xa\in A\} $不包含在任意极大理想中(事实上:对任意素理想P,$a\in \bar{A}_{P=}\mathcal{O}\cap A\mathcal{O}_{P},$从而a可表示为$a=\frac{a^{'}}{s_{P}},a^{'}\in A,s_{P}\notin P$,且有$s_{P}a=a^{'}\in A$),上述将导致$B=\mathcal{O},i.e,a=1\cdot a\in A$.\\
    如果$P\supseteq A,$那么$P$是唯一一个包含$\bar{A}_{P}$的素理想(事实上:由于$P=\mathcal{O}\cap P\mathcal{O}_{P}$($\subseteq $是平凡的,反方向是由于若$\frac{q}{s}=a\in \mathcal{O}\cap P\mathcal{O}_{P},$那么$q=sa\in P$,由于$s\notin P$,得到$a\in P$)$\bar{A}_{P}=\mathcal{O}\cap A\mathcal{O}_{P}\subseteq \mathcal{O}\cap P\mathcal{O}_{P}=P$),从而任给两个不同的素理想$P,Q,$理想$\bar{A}_{P}+\bar{A}_{Q}$不包含在任何极大理想中,因此$\bar{A}_{P}+\bar{A}_{Q}=\mathcal{O}.$从而由中国剩余定理
    得到同构$\mathcal{O}/A\cong \oplus_{P\supseteq A}\mathcal{O}/\bar{A}_{P}.$再由于$\mathcal{O}/\bar{A}_{P}=\mathcal{O}_{P}/A\mathcal{O}_{P}$,即得到命题。
    
\end{document}
