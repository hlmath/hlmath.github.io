\documentclass[UTF8]{article}
\usepackage{ctex}
\usepackage{tikz}
\usepackage[all]{xy}
\usepackage[colorlinks=true]{hyperref}
\title{范畴}
\author{Lhzsl}
\date{\today }
\usepackage[b5paper,left=10mm,right=10mm,top=15mm,bottom=15mm]{geometry}
\usepackage{amsthm,amsmath,amssymb}
\usepackage{mathrsfs}
\begin{document}
	\maketitle
	本文适用于对范畴零基础的读者\\
 范畴的定义\\
  \textbf{定义:}一个\textbf{范畴}$\mathfrak{C}$包含以下结构\\
  1.集合$Ob(\mathfrak{C})$,其元素称作$\mathfrak{C}$的对象,\\
  2.对于$\mathfrak{C}$中任意两个对象,给定一个集合$[A,B]_{\mathfrak{C}}$,其中元素称为范畴$\mathfrak{C}$内$A$到$B$的\textbf{态射},该集合常简记为$Hom_{\mathfrak{C}}(A,B).$集合$Moc(\mathfrak{C})=\cup_{A,B\in \mathfrak{C}}Hom_{\mathfrak{C}}(A,B)$中元素称为$\mathfrak{C}$的态射。\\
  3.对每个对象X给定元素$id_{X}\in Hom_{\mathfrak{C}}(X,X),$称为X到自身的恒等映射。\\
  4.对于任意$X,Y,Z\in Ob(\mathfrak{C})$,给定态射间的\textbf{合成映射}\\
 $$ \circ:Hom_{\mathfrak{C}}(Y,Z)\times Hom_{\mathfrak{C}}(X,Y)\longrightarrow Hom_{\mathfrak{C}}(X,Z)
  $$
  $$
  (f,g)\mapsto f\circ g,
  $$
  常将 $f\circ g$简记为$fg$.它满足\\
  (i)结合律:对于任意态射$h,g,f\in Mor(\mathfrak{C})$,若合成$f(gh)$和$(fg)h$都有定义,则$$
  f(gh)=(fg)h.
  $$
  故两边可写成同时$fgh$。\\
  (ii)对于任意态射$f\in Hom_{\mathfrak{C}}(X,Y),$有$$
  f\circ id_{X}=f=id_{Y}\circ f.
  $$
  设有范畴$\mathfrak{C},\mathfrak{B}$.若有$Ob(\mathfrak{B})\subseteq Ob(\mathfrak{C}),$对于任意$A,B\in Ob(\mathfrak{B}),Hom_{\mathfrak{B}}(A,B)\subseteq Hom_{\mathfrak{C}}(A,B)$及$\mathfrak{C},\mathfrak{B}$有相同的合成映射,那么称$\mathfrak{B}$是$\mathfrak{C}$的子范畴,再者,如果对于任意$A,B\in Ob(\mathfrak{C})$有$Hom_{\mathfrak{C}}(A,B)=Hom_{\mathfrak{B}}(A,B),$则称$\mathfrak{B}$是$\mathfrak{C}$的全子范畴。\\
  这里仅举一个范畴的例子:把所有的群都放在一起得到$\mathfrak{Gp}.$对于任意两个群,所有从A到B的群同态组成$Hom_{\mathfrak{Gp}}(A,B),g\circ f$是映射的合成,这样便得到群范畴$\mathfrak{Gp}.$\\
  设$m$是范畴$\mathfrak{C}$的态射,如果对于任意$f,g\in Mor(\mathfrak{C})$,从$mf=mg$得到$f=g,$则称m为单态射,另一方面如果对于任意$f,g\in Mor_{\mathfrak{C}},$从$fm=gm$得到$f=g$,则称m为满态射。我们称$f\in [A,B]_{\mathfrak{C}}$是同构,如果 存在$g\in [B,A]_{\mathfrak{C}}$使得$gf=1_{A},fg=1_{B}.$\\
  设$\mathfrak{C},\mathfrak{D}$是两个范畴,由$\mathfrak{C}$到$\mathfrak{D}$的一个(共变(covariant))函子(functor)F是指:\\
  (i)对于$\mathfrak{C}$中任意对象X,F规定了$\mathfrak{D}$中的相应的对象$F(X).$\\
  (ii)设$X,Y$是$\mathfrak{C}$中任意两个对象,对于任一$f\in [X,Y]_{\mathfrak{C}},$F规定$[F(X),F(Y)]_{\mathfrak{D}}$中的一个元素(态射)F(f),满足:\\
  $$F(g\circ f)=F(g)\circ F(f),\forall f\in[X,Y]_{\mathfrak{C}},g\in [Y,Z]_{\mathfrak{C}} 
  $$
  以及$$
  F(1_{X})=1_{F(X)}.
  $$
  如果将上述条件(ii)改成\\
  (ii')对于任一$f\in [X,Y]_{\mathfrak{C}},$F规定了$[F(Y),F(X)]_{\mathfrak{D}}$中的一个元素(态射)F(f),满足:\\
  $$
  F(g\circ f)=F(f)\circ F(g),\forall f\in [X,Y],g\in [Y,Z]
  $$
  以及$$
  F(1_{X})=1_{F(X)}
  $$
  则称F为$\mathfrak{C}$到$\mathfrak{D}$的反变函子(contravariant functor)。\\
  对于任何范畴$\mathfrak{C}$,设$((sets))$是全体集合组成的范畴,任取$X\in Ob(\mathfrak{C})$,若规定$h_{X}:\mathfrak{C}\rightarrow ((sets)),Y\mapsto Hom_{\mathfrak{C}}(Y,X),$任意态射$f:Z\rightarrow Y,$
  $$h_{X}(f):Hom_{\mathfrak{C}}(Y,X)\rightarrow Hom_{\mathfrak{C}}(Z,X),$$
  $$g\mapsto g\circ f$$
  则$h_{X}$(也可写作$Hom_{\mathfrak{C}}(\cdot,X)$)是$\mathfrak{C}$到$((sets))$的反变函子。\\
  联系两个函子的概念是“函子态射”(或称"自然变换").设$\mathfrak{C}$和$\mathfrak{D}$是两个范畴,F和G是$\mathfrak{C}$到$\mathfrak{D}$的两个函子。由F到$G$的函子态射$\Phi$是指:对于$\mathfrak{C}$的任何一个对象X,给定一个态射$\Phi_{X}:F(X)\rightarrow G(X)$,使得下图交换
  $$
\xymatrix{F(X)\ar[r]^{\Phi_{X}}\ar[d]_{F(f)}&G(X)\ar[d]^{G(f)}\\
F(Y)\ar[r]^{\Phi_{Y}}&G(Y)}
  $$
  其中$X,Y$是$\mathfrak{C}$中任意两个对象,f是$X$到$Y$的任意态射。我们记函子态射$\Phi$为$\Phi:F\rightarrow G,$当所有$\Phi_{X}:FX\rightarrow GX$是同构时,称函子同态是函子同构。\\
  称函子$T:\mathfrak{C}\rightarrow \mathfrak{D}$是范畴等价:若存在函子$S:\mathfrak{D}\rightarrow \mathfrak{C}$及函子同构
  $$
  \Phi: 1_{\mathfrak{D}}\backsimeq TS,\Psi :ST\backsimeq 1_{\mathfrak{C}}.
  $$
  称函子$T:\mathfrak{C}\rightarrow \mathfrak{D}$是要满函子:若对于任意$Y\in Ob(\mathfrak{D})$,
存在$X\in Ob(\mathfrak{C})$使得$TX$与$Y$同构。\\
称函子T是忠实的(faithful),如果从任意对象$A,B\in Ob(\mathfrak{C})$所定的映射$$
T:Hom_{\mathfrak{C}}(A,B)\rightarrow Hom_{\mathfrak{D}}(T(A),T(B))
$$
$$
(A\stackrel{f}{\rightarrow }B)\mapsto (T(A)\stackrel{T(f)}{\rightarrow}T(B))
$$
是单射。如果此映射是满射,则说函子T是全忠实函子(fully faithful).\\
  \textbf{命题:}函子$F:\mathfrak{C}_{1}\rightarrow \mathfrak{C}_{2}$等价当且仅当F是全忠实要满函子。\\
  证明:$(\Rightarrow)$设拟逆函子$G:\mathfrak{C}_{2}\rightarrow \mathfrak{C}_{1}$和函子同构$\psi:id_{\mathfrak{C}_{1}} \approx GF,\phi: id_{\mathfrak{C}_{2}}\approx FG .$对于$\mathfrak{C}_{2}$中任何对象$Z$,都有同构$\phi_{Z}:F(GZ)\rightarrow Z,$故$F$是本质满的。\\
  观察到$$
  Hom_{\mathfrak{C}_{1}}(X,Y)\stackrel{F}\rightarrow Hom_{\mathfrak{C}_{2}}(FX,FY)\stackrel{G}\rightarrow Hom_{\mathfrak{C}_{1}}(GF(X),GF(Y))\rightarrow Hom_{\mathfrak{C}_{1}}(X,Y)
  $$
$$
f\longmapsto Ff\longmapsto GF(f)\longmapsto \psi_{Y}^{-1}GF(f)\psi_{X}
$$  
合成是恒等映射,事实上,由函子同构$\psi :id_{\mathfrak{C}_{1}}\stackrel{\sim}{\rightarrow} GF$,有交换图表
$$\xymatrix{ X\ar[r]^{\psi_X}\ar[d]_{f} & GF(X)\ar[d]^{GF(f)}\\
Y\ar[r]_{\psi_Y} & GF(Y)  }   $$    
从而$\psi_{Y}f=GF(f)\psi_{X}.$由于$\psi_{Y}$是同构,因此$f=\psi_{Y}^{-1}GF(f)\psi_{X}.$这就说明F有逆映射,从而是单射,同样可得G是忠实的。\\
下面证明是满射:首先对任意的$v\in Hom_{\mathfrak{C}_{2}}(FX,FY)$,令$f=(\psi_{Y})^{-1}G(v)(\psi_{X}).$对于该映射,上交换图依然交换,于是$GF(f)=G(v),$由于$G$是忠实的,得到$F(f)=v,$从而是满射。\\
$(\Leftarrow)$首先对于$\mathfrak{C}_{2}$中每一对象M,$\mathfrak{C}_{1}$中存在对象$A_{M}$(可能有多个,选取其中一个)及同构$v_{M}:M\rightarrow F(A_{M}).$\\
定义函子$G:\mathfrak{C}_{2}\rightarrow \mathfrak{C}_{1}$如下:若$M\in \mathfrak{C}_{2},$设$G(M)=A_{M}.$若$w:M\rightarrow N$,利用$v_{M}$是同构得到$w^{'}:FA\rightarrow FB$使有交换图表:
$$
\xymatrix{M\ar[r]^{v_{M}}\ar[d]_{w}&FA_{M}\ar[d]^{w^{'}}\\
	N\ar[r]_{v_{N}}&FA_{N}
}
$$ 
由于F是全忠实函子,故有$w_{1}:A\rightarrow B$使得$F(w_{1})=w^{'}.$设$G(w)=w_{1}.$\\
取$A\in \mathfrak{C}_{1}$,设$B=GFA,$于是有同构$v_{FA}:FA\rightarrow FB.$因为F是全忠实函子,有唯一$\psi_{A}:A\rightarrow B=GFA$使得$F\psi_{A}=v_{FA}.$注意由于$v_{FA}$是同构,F是全忠实函子,得到$\psi_{A}:A
\rightarrow GFA$是同构,这就决定函子同构$\psi:1\rightarrow GF.$\\
取$M\in \mathfrak{C}_{2},$设$A=GM.$则同构$v_{M}:M\rightarrow FA=FGM$决定函子同构$\phi :1\rightarrow FG,$即有$\phi_{M}=v_{M}.$\\
下面验证函子态射的交换性。\\
首先验证$$
\xymatrix{M\ar[r]^{\phi_{M}=v_{M}}\ar[d]_{w}&FG(M)\ar[d]^{FG(w)}\\
	N\ar[r]_{\phi_{N}=v_{N}}&FG(N)
}
$$
这只需注意到$FG(w)=w^{'}$,由上一个交换图表即得。\\
下面验证$\mathfrak{C}_{1}$中函子态射交换性,任意态射$f:A\rightarrow Y,$在$\mathfrak{C}_{2}$中有下交换图表
$$
\xymatrix{
FA\ar[r]^{v_{FA}}\ar[d]_{F(f)}&FG(FA)\ar[d]^{FG(F(f))}\\
FY\ar[r]_{v_{FY}}&FG(F(Y))
}
$$
从而$F(GF(f))=v_{FY}\circ F(f)\circ (v_{FA})^{-1}.$再注意到$F\psi_{A}=v_{FA}.$于是$F(\psi_{Y}\circ f\circ {\psi_{A}}^{-1})=v_{FY}\circ F(f)\circ (v_{FA})^{-1}.$由于F是忠实的,得到$GF(f)=\psi_{Y}\circ f\circ {\psi_{A}}^{-1}$.从而有交换图
$$
\xymatrix{A\ar[r]^{\psi_{A}}\ar[d]_{f} &GF(A)\ar[d]^{GF(f)}\\
Y\ar[r]^{\psi_{Y}}&GF(Y)   }
$$
证毕。\\
若$F,G:\mathfrak{C}\rightarrow \mathfrak{D}$是两个函子,我们用记号$Tran(F,G)$表示从F到G的自然变换全体.有下述引理\\
\textbf{Yoneda 引理:}设$\mathfrak{C}$是范畴,F是$\mathfrak{C}$到集合范畴((sets))的反变函子,$X\in Ob(\mathfrak{C})$,则有双射$$
y:Tran(Hom_{\mathfrak{C}}(\cdot,X),F)\rightarrow F(X).
$$
$$
y:\bar{\xi}\mapsto \bar{\xi}_{X}(id_{X})
$$
证明:设$\xi\in F(X).$对于$\mathfrak{C}$中态射$f:A\rightarrow X,$令$\bar{\xi}_{A}(f)=F(f)(\xi).$下面验证$\bar{\xi}$是自然变换。在$\mathfrak{C}$中$g:A\rightarrow B$为态射,须验证下图交换
$$
\xymatrix{Hom_{\mathfrak{C}}(A,X)\ar[r]^{\bar{\xi}_{A}}\ar[d]_{Hom_{\mathfrak{C}}(\cdot,X)(g)}&F(A)\ar[d]^{F(g)}\\
	Hom_{\mathfrak{C}}(B,X)\ar[r]_{\bar{\xi}_{B}}&F(B)
}
$$
任意$f\in Hom_{\mathfrak{C}}(A,X),F(g)\bar{\xi}_{A}(f)=F(g)F(f)(\xi)=F(f\circ g)(\xi)$(注意F是反变函子),\\
$\bar{\xi}_{B}(Hom_{\mathfrak{C}}(\cdot,X)(g))(f)=\bar{\xi}_{B}(f\circ g)=F(f\circ g)(\xi),$这就是上图交换,于是$\bar{\xi}$是自然变换,且$\bar{\xi}_{X}(id_{X})=F(id_{X})(\xi)=\xi.$即该映射是满射。\\
下证单射:$\bar{\xi}\in Tran(Hom_{\mathfrak{C}}(\cdot,X),F)$是自然变换,于是对于$\mathfrak{C}$中态射$f:A\rightarrow X$,有下交换图
$$
\xymatrix{Hom_{\mathfrak{C}}(X,X)\ar[r]^{\bar{\xi}_{X}}\ar[d]_{Hom_{\mathfrak{C}}(\cdot,X)(f)}&F(X)\ar[d]^{F(f)}\\
	Hom_{\mathfrak{C}}(A,X)\ar[r]_{\bar{\xi}_{A}}&F(A)}
$$
于是$\bar{\xi}_{A}(f)=\bar{\xi}_{A}(f)(id_{X})=F(f)\circ \bar{\xi}_{X}(id_{X}).$若有另外自然变换$\sigma\in Tran(Hom_{\mathfrak{C}}(\cdot,X)),$使得$\sigma_{X}(id_{X})=\bar{\xi}_{X}(id_{X}),$则$\bar{\xi}_{A}(f)=\sigma_{A}(f)$,于是$\bar{\xi}_{B}=\sigma_{B}$对任意$B\in Ob(\mathfrak{C})$成立,即$\bar{\xi}=\sigma .$此即单射。$\quad \bullet$\\
$Abel$范畴有拉回和推出.详细的说,\\
(1)设$\mathcal{A}$是加法范畴.则$\mathcal{A}$有拉回当且仅当其有核.\\
设$\mathcal{A}$有核,$b:B\longrightarrow D$和$g:C\longrightarrow D.$考虑$(b,g):B\oplus C\longrightarrow D$的核$Ker(b,g)$及合成
$$
f:Ker(b,g)\hookrightarrow B\oplus C\stackrel{(1,0)}{\longrightarrow}B
$$
和$$
-a:Ker(b,g)\hookrightarrow B\oplus C\stackrel{(0,1)}{\longrightarrow}C.
$$
则$(a,f,Ker(b,g))$是$(b,g)$的拉回.\\
证明:若$\mathcal{A}$有核,则直接验证上述构造是拉回.\\
即有拉回$$
\xymatrix{Ker(f,g)\ar[r]^{a}\ar[d]_{f} &C\ar[d]^{g}\\
	B\ar[r]^{b}&D   }
$$
为此,需验证(1)$bf=ga.$注意到$Ker(f,g)$是映射$(b,g):B\oplus C\longrightarrow D$的核,记$h:Ker(b,g)\longrightarrow B\oplus C,$于是有$(b,g)h=0,$展开得到$(b(1,0)+g(0,1))h=0,$即$bf-ga=0.$\\
(2)设另有$(t,s,X)$使得下图交换
$$
\xymatrix{X\ar[r]^{s}\ar[d]_{t} &C\ar[d]^{g}\\
	B\ar[r]^{b}&D   }
$$
需证明存在态射$r:X\longrightarrow Ker(b,g)$使得下图交换
$$
\xymatrix{X\ar@{-->}[dr]^{r}\ar[rrd]^{s}\ar[ddr]_{t}\\
   &Ker(b,g)\ar[r]^{a}\ar[d]_{f}&D\ar[d]^{g}\\
&B\ar[r]^{b}&D   }
$$
注意到$Ker(f,g)$是映射$(b,g):B\oplus C\longrightarrow D$的核,我们可构造一个映射$k:X\longrightarrow B\oplus C$使得$(b,g)k=0.$这样利用核的性质,我们便能得到一个映射$r:X\longrightarrow Ker(b,g).$这一映射或许就是我们要找的映射。
余积$B\oplus C$连同的态射记为$e_{1}:B\longrightarrow B\oplus C,e_{2}:C\longrightarrow B\oplus C.$\\
$(b,g)e_{1}t=bt,(b,g)(-e_{2})s=-gs.$从而$(b,g)(e_{1}t-e_{2}s)=0.$于是存在映射$r:X\longrightarrow Ker(b,g)$使得$hr=e_{1}t-e_{2}s.$下面验证:$ar=-(0,1)hr=s,fr=(1,0)hr=t.$满足交换性。
$$
\xymatrix{X\ar@{-->}[dr]^{r}\ar[rrrdd]^{s}\ar[dddrr]_{t}\\
	&Ker(b,g)\ar[dr]^{h}\\
	& &B\oplus C\ar[r]^{-(0,1)}\ar[d]_{(1,0)}\ar[rd]^{(b,g)}&C\ar[d]^{g}\\
	& &B\ar[r]^{b}&D   }
$$
设$\mathcal{A}$是有零对象的范畴,考虑下图$$
\xymatrix{
A^{'}\ar[r]&A\ar[d]\\
B^{'}\ar[r]&B
}
$$
其中$B^{'}\longrightarrow B$是某个态射$B\longrightarrow B^{''}$的核.则该图能扩展成交换图当且仅当$A^{'}\longrightarrow A$是复合态射$A\longrightarrow B\longrightarrow B^{''}$的核.\\
证明:设$A^{'}\longrightarrow A$是复合态射$A\longrightarrow B\longrightarrow B^{''}$的核,则存在唯一的态射$A^{'}\longrightarrow B^{'}$使得上图交换。若有X使得$X\longrightarrow A\longrightarrow B=X\longrightarrow B^{'}\longrightarrow B,$则$X\longrightarrow A\longrightarrow B\longrightarrow B^{''}=X\longrightarrow B^{'}\longrightarrow B\longrightarrow B^{''}.$由于$A^{'}\longrightarrow A$是复合态射$A\longrightarrow B\longrightarrow B^{''}$的核,因此存在唯一态射$X\longrightarrow A^{'}$使得$X\longrightarrow A^{'}\longrightarrow A=X\longrightarrow A.$并且
$$X\longrightarrow A^{'}\longrightarrow B^{'}\longrightarrow B=X\longrightarrow A^{'}\longrightarrow A\longrightarrow B=X\longrightarrow A\longrightarrow B=X\longrightarrow B^{'}\longrightarrow B.$$
由于$B^{'}\rightarrow B$是单射,$X\rightarrow A^{'}\rightarrow B^{'}=X\rightarrow B^{'}.$\\
反之,若有拉回
$$
\xymatrix{
	A^{'}\ar[r]\ar[d]&A\ar[d]\\
	B^{'}\ar[r]&B
}
$$
则$A^{'}\longrightarrow A$是复合态射$A\longrightarrow B\longrightarrow B^{''}$的核。事实上,若$X\rightarrow A\longrightarrow B\longrightarrow B^{''}=0$由于$B^{'}\rightarrow B$是$B\rightarrow B^{''}$的核,于是存在态射$X\rightarrow B^{'}$使得$X\rightarrow A\rightarrow B=X\rightarrow B^{'}\rightarrow B.$再根据拉回性质,存在态射$X\rightarrow A^{'}$使得$X\rightarrow A^{'}\rightarrow A=X\rightarrow A.$\\
(9 lemma)
如下是范畴中的一个交换图,所有行和列是正合列.
$$
\xymatrix{
	&0\ar[d]&0\ar[d]&0\ar[d]\\
	&A^{'}\ar[d]&A\ar[d]&A^{''}\ar[d]\\
0\ar[r]&	B^{'}\ar[r]\ar[d]&B\ar[r]\ar[d]&B^{''}\ar[r]\ar[d]&0\\
0\ar[r]&	C^{'}\ar[r]\ar[d]&C\ar[r]\ar[d]&C^{''}\ar[r]\ar[d]&0\\
&0&0&0&\\
}
$$
存在态射$A^{'}\rightarrow A$和$A\rightarrow A^{''}$使得图交换,并且$
0\rightarrow A^{'}\rightarrow A\rightarrow A^{''}\rightarrow 0$是正合列。\\
下图是一拉回$$
\xymatrix{
	B^{'}\ar[r]\ar[d]&C^{'}\ar[d]\\
	B\ar[r]&C
}
$$
其中$B\rightarrow C$是满射,$C^{'}\rightarrow C$是单射,则能扩充为下面交换图.\\
$$
\xymatrix{
	&&0\ar[d]&0\ar[d]\\
0\ar[r]	&A\ar@{=}[d]\ar[r]&B^{'}\ar[d]\ar[r]&C^{'}\ar[d]\ar[r]&0\\
	0\ar[r]&A\ar[r]&B\ar[r]\ar[d]&C\ar[r]\ar[d]&0\\
&&C^{''}\ar@{=}[r]\ar[d]&C^{''}\ar[d]&\\
	&&0&0&\\
}
$$
证明:令$C\rightarrow C^{''}$是$C^{'}\rightarrow C$的余核,$A\rightarrow B$是$B\rightarrow C$的核。由上面命题知$B^{'}\rightarrow B$是$B\rightarrow C\rightarrow C^{''}$的核,于是便能得到中间一列的正合性。接下来对下图应用“9引理”便得到
结果。
$$
\xymatrix{
	&0\ar[d]&0\ar[d]&0\ar[d]\\
	&A\ar@{=}[d]&B^{'}\ar[d]&C^{'}\ar[d]\\
0\ar[r]	&A\ar[d]\ar[r]&B\ar[d]\ar[r]&C\ar[d]\ar[r]&0\\
	0\ar[r]&0\ar[r]\ar[d]&C^{''}\ar@{=}[r]\ar[d]&C^{''}\ar[r]\ar[d]&0\\
	&0&0&0&\\
}
$$
这里注意到存在唯一地态射$B^{'}\rightarrow C^{'}$使得$$
\xymatrix{
	B^{'}\ar[r]\ar[d]&C^{'}\ar[d]\\
	B\ar[r]&C
}
$$交换,于是由"9引理"得到的态射$B^{'}\rightarrow C^{'}$核原来是一致的。
\end{document}