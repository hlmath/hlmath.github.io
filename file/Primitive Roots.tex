\documentclass[UTF8]{article}
\usepackage{ctex}
\usepackage{tikz}
\usepackage[all]{xy}
\usepackage[colorlinks=true]{hyperref}
\title{剩余类环的单位群}
%%\author{Lh}
\date{\today }
\usepackage[b5paper,left=10mm,right=10mm,top=15mm,bottom=15mm]{geometry}
\usepackage{amsthm,amsmath,amssymb}
\usepackage{mathrsfs}
\usepackage{indentfirst}
\begin{document}
 \maketitle
	设$N$是正整数,求$(\mathbb{Z}/N\mathbb{Z})^{\times}$的结构。设$N=p_{1}^{a_{1}}\cdots p_{r}^{a_{r}},$由中国剩余定理
	$$
	(\mathbb{Z}/N\mathbb{Z})^{\times}=(\mathbb{Z}/(p_{1}^{a_{1}}\cdots p_{r}^{a_{r}}\mathbb{Z}))^{\times}\cong \prod_{i=1}^{r}(\mathbb{Z}/p_{i}^{a_{i}}\mathbb{Z})^{\times}.
	$$
	于是我们只需算出$N=p^{a}$时$(\mathbb{Z}/N\mathbb{Z})^{\times}$的结构。\\
	\textbf{定理1:}当$p$是奇素数,或者当$p=2,a=1$,或$2$时,有限Abel群$(\mathbb{Z}/p^{a}\mathbb{Z})^{\times}$是循环群.若$a\geq 3$,则
	$$
	(\mathbb{Z}/2^{a}\mathbb{Z})^{\times}\cong (\mathbb{Z}_{2},+)\oplus (\mathbb{Z}_{2^{a-2}},+).
	$$
	证明:首先,易知$(\mathbb{Z}/p\mathbb{Z})^{\times}$是$p-1$阶循环群。 取$(\mathbb{Z}/p\mathbb{Z})^{\times}$的一个生成元$g$,则$g$在$(\mathbb{Z}/p^{a}\mathbb{Z})^{\times}$中阶必被$p-1$整除.事实上,设$g^{k}\equiv 1(mod p^{a})$则$p^{a}|g^{k}-1,$从而$p|g^{k}-1$,于是$g^{k}\equiv 1(mod p)$,但$p$是$(\mathbb{Z}/p\mathbb{Z})^{\times}$的一个生成元,从而$g$的阶为$p-1$,于是$p-1|k$.由$g^{p^{a-1}(p-1)}\equiv 1(mod p^{a})$知$g$在$(\mathbb{Z}/p^{a}\mathbb{Z})^{\times}$中阶数为$p^{a-1}(p-1)$的因数,从而设为$p^{k}(p-1),0\leq k\leq a-1,$于是$g^{'}=g^{p^{k}}$的阶数为$p-1$,令$z=1+p$,我们下面说明$z$的阶为$p^{a-1}.$\\
	\textbf{引理:}设$p$是奇素数,$z\in Z,z\equiv 1(mod\quad p)$,若$z$有素数分解$z=p_{1}^{a_{1}}\cdots p_{r}^{a_{r}}$,定义$ord_{p_{i}}(z)=a_{i},$
	即$ord_{p}(z)$表示$z$的素数分解中素数$p$出现的次数。\\
	则$a)ord_{p}(z^{p}-1)=ord_{p}(z-1)+1.$\\
	 $b)\forall k\in \mathbb{Z}^{+},ord_{p}(z^{p^{k}}-1)=ord_{p}(z-1)+k$.\\ 
	证明:令$z=1+xp,x\in Z$,那么$ord_{p}(z-1)=1+ord_{p}(x)$,二项展开
	$$
	z^{p}-1=(1+xp)^{p}-1=\binom{p}{1}(xp)+\binom{p}{2}(xp)^{2}+\cdots+\binom{p}{p-1}(xp)^{p-1}+(xp)^{p}.
	$$
	由此易看出
	$$
	ord_{p}(z^{p}-1)=ord_{p}(\binom{p}{1}(xp))=2+ord_{p}(x)=ord_{p}(z-1)+1.
	$$
$b)$用$a)$和归纳法。\\
 应用上述引理到$z=1+p$,则$ord_{p}(z^{p^{k-1}}-1)=k,\forall k\in \mathbb{Z}^{+}.$因此$z^{p^{a-2}}\neq 1(mod \quad p^{a}),z^{p^{a-1}}\equiv 1(mod\quad p^{a})$:即$z$在$(\mathbb{Z}/p^{a}\mathbb{Z})^{\times}$中的阶为$p^{a-1},$由于$(p^{a-1},p-1)=1$,$g^{'}z$的阶为$p^{a-1}(p-1)=|(\mathbb{Z}/p^{a}\mathbb{Z})^{\times}|$,	这就说明$(\mathbb{Z}/p^{a}\mathbb{Z})^{\times}$是循环群.\\
 若$p=2$,首先易证$(\mathbb{Z}/2\mathbb{Z})^{\times},(\mathbb{Z}/4\mathbb{Z})^{\times}$是循环群。
 若$a\geq 3$,类似上述引理,设$z=1+2x,$则$$
 z^{2}-1=4x^{2}+4x=4x(x+1)
 $$
	若$x$是偶数,则
	$$
	ord_{2}(z^{2}-1)=ord_{2}(4x(x+1))=ord_{2}(4x)=1+ord_{2}(2x)=1+ord_{2}(z-1)
	$$
	注意到$z^{2}=1+2(2x(x+1))$,因此
	$$
	ord_{2}((z^{2})^{2}-1)=ord_{2}(z^{2}-1)+1=ord_{2}(z-1)+2
	$$
	由此归纳下去便得到
	$$
	ord_{2}(z^{2^{k}}-1)=ord_{2}(z-1)+k
	$$
	特别地,取$x=2$,则$z=5$,于是$ord_{2}(5^{2^{k}}-1)=k+2,$因此若$a\geq 2$,则$5$在$(\mathbb{Z}/2^{a}\mathbb{Z})^{\times}$中阶为$2^{a-2}$。易证
	$2^{a}-1$在$(\mathbb{Z}/2^{a}\mathbb{Z})^{\times}$中阶为$2$.注意到在$\mathbb{Z}/2^{a}\mathbb{Z}$中$2^{a}-1\equiv -1(mod 2^{a})$且$5^{k}\equiv 1(mod4)$对任意正整数$k$成立,因此$5^{k}\neq -1(mod 2^{a})$,用$<5>$表示$5$在$(\mathbb{Z}/2^{a}\mathbb{Z})^{\times}$中生成的循环群,$<2^{a}-1>$表示$2^{a}-1$在$(\mathbb{Z}/2^{a}\mathbb{Z})^{\times}$生成的循环群,则$<5>\cap <2^{a}-1>=0,|<5>||<2^{a}-1>|=2^{a}$,于是$$(\mathbb{Z}/2^{a}\mathbb{Z})^{\times}=<5>\times<2^{a}-1>$$
\textbf{推论:}$(\mathbb{Z}/N\mathbb{Z})^{\times}$是循环群,当且仅当$N$满足下列条件\\
$(i)$$N=1,2,4.$\\
$(ii)N=p^{a},$这里$p$是奇素数.\\
$(iii)N=2p^{a},$这里$p$是奇素数.\\
证明:
当$N$满足$(i)$或$(ii)$中的条件时,由上一定理知$(\mathbb{Z}/N\mathbb{Z})^{\times}$是循环群。\\
若$p$是奇素数,则
$$
(\mathbb{Z}/2p^{a}\mathbb{Z})^{\times}\cong (\mathbb{Z}/2\mathbb{Z})^{\times}\times(\mathbb{Z}/p^{a}\mathbb{Z})^{\times}\cong (\mathbb{Z}/p^{a}\mathbb{Z})^{\times}
$$
即$(\mathbb{Z}/2p^{a}\mathbb{Z})^{\times}$是循环群。
反过来:若$N$不是上述形式,则$N$被$8$整除或$N$有两个不同的素数,对于第一种情形
,$N$可以写成$N=2^{a}M,(2,M)=1,a\geq 3.$于是
$$
(\mathbb{Z}/N\mathbb{Z})^{\times}\cong (\mathbb{Z}/2^{a}\mathbb{Z})^{\times}\times (\mathbb{Z}/M\mathbb{Z})^{\times}
$$
$(\mathbb{Z}/2^{a}\mathbb{Z})^{\times}$不是循环群,因此$(\mathbb{Z}/N\mathbb{Z})^{\times}$不是循环群。
对于第二种情形,$N$可以写成$N=p^{a}q^{b}M,$于是
$$
(\mathbb{Z}/N\mathbb{Z})^{\times}\cong (\mathbb{Z}/p^{a}\mathbb{Z})^{\times}\times (\mathbb{Z}/q^{b}\mathbb{Z})^{\times}\times (\mathbb{Z}/M\mathbb{Z})^{\times}
$$
由于$(\mathbb{Z}/p^{a}\mathbb{Z})^{\times},(\mathbb{Z}/q^{b}\mathbb{Z})^{\times}$的阶都是偶数,它们的阶不是互素的,从而它们的直积不是循环群,
$(\mathbb{Z}/N\mathbb{Z})^{\times}$不是循环群.\\
\end{document}