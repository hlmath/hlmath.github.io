\documentclass[UTF8]{article}
\usepackage{ctex}
\usepackage[colorlinks=true]{hyperref}
\title{\textbf{\huge{剩余符号与Hilbert符号}}}
\author{Lh}
\date{\today}
\usepackage[b5paper,left=10mm,right=10mm,top=15mm,bottom=15mm]{geometry}
\usepackage{amsthm,amsmath,amssymb}
\usepackage{mathrsfs}
\usepackage{tikz}
\usepackage[all]{xy}
\usetikzlibrary{cd}
\usepackage{fancyhdr}
\usepackage{color}
\newtheorem{thm}{Theorem}[section]
\newtheorem{defn}{Definition}[section]
\newtheorem{cor}{Corollary}[section]
\newtheorem{prop}{Proposition}[section]
\newtheorem{exa}{Example}[section]
\newtheorem{lem}{Lemma}[section]
\newtheorem{Rem}{Remark}[section]

\begin{document}
	\maketitle
	\tableofcontents
\section{数域中的剩余符号}
该部分主要参考\cite{lemmer}.

在初等数论中,我们知道整数环$\mathbb{Z}$中有二次剩余符号:设$p\in \mathbb{Z}$是奇素数,$a\in \mathbb{Z}$,若$p\nmid a$,则有
$$
\left(\frac{a}{p}\right)=1\Leftrightarrow x^{2}\equiv a\ mod \ p\text{有整数解.}
$$
对于二次互反律,我们有Eular判别法:设$p$是素数,$(a,p)=1$,则有
$$
\left(\frac{a}{p}\right)\equiv a^{\frac{p-1}{2}}\ mod \ p.
$$
Gauss首先证明下面互反律:
设$p,q$为不同的奇素数,则有
$$
\left(\frac{p}{q}\right)\left(\frac{q}{p}\right)=(-1)^{\frac{p-1}{2}\frac{q-1}{2}}.
$$
利用该互反律我们可以很容易计算一些二次剩余符号,例如:
$$
\left(\frac{5}{17}\right)=\left(\frac{17}{5}\right)=\left(\frac{2}{5}\right)=-1.
$$
下面我们说明二次互反律存在统一的推广。
\subsection{基本定义及性质}
设$k$是一个数域,$\mathcal{O}_{k}$是$k$的代数整数环。$\mathfrak{p}$是一个$\mathcal{O}_{k}$中素理想,则$N_{k|\mathbb{Q}}(\mathfrak{p})=p^{f}$,其中$f$是剩余类域$(\mathcal{O}_{k}/\mathfrak{p})/(\mathbb{Z}/p\mathbb{Z})$的扩张次数。下令$q=p^{f}$。对于任意$\mathcal{O}_{k}$中元素$\alpha\notin \mathfrak{p}$(即$\alpha$与$\mathfrak{p}$互素),有
$$
\alpha^{q-1}\equiv 1\ mod \ \mathfrak{p}.
$$
\begin{lem}
若$k$中包含$n$次单位根$\xi_{n}$,且素理想$\mathfrak{p}$与$n$互素,
则$\xi_{n}\ mod \ \mathfrak{p}$在$(\mathcal{O}_{k}/\mathfrak{p})^{*}$中生成的子群阶为$n$.
\end{lem}
\begin{proof}
(1)若$n$是素数幂,因$n$与$\mathfrak{p}$互素,$n\notin \mathfrak{p}\cap \mathbb{Z}=(p)$.设$n=l^{r},$其中$l$是素数,且$(l,p)=1.$ $K$中包含分圆域$\mathbb{Q}(\xi_{l^{r}})$,若$1-\xi_{l^{r}}\in \mathfrak{p}$,则
$$
l\in (1-\xi_{l^{r}})\mathcal{O}_{k}\subseteq \mathfrak{p}.
$$
与$(l,p)=1$矛盾!类似的,对于任意$1\leq a<l^{r}$,可证明若$1-\xi_{l^{r}}^{a}\in \mathfrak{p}$,则$
l\in (1-\xi_{l^{r}}^{a})\mathcal{O}_{k}\subseteq \mathfrak{p}.$矛盾!从而$1-\xi_{l^{r}}^{a}\notin \mathfrak{p}$,由此可知$l^{r}$次单位根模$\mathfrak{p}$两两不同余。
(2)若$n$不是素数幂,即$n$中至少包含两个不同的素因子,且$(p,n)=1$,则$1-\xi_{n}$是单位。对于$1\leq a<n$,若$n=am$,其中$m$有两个不同的素因子,则$1-\xi_{n}^{a}$仍是单位,若$m$是素数幂,由(1)仍有$1-\xi_{n}^{a}\notin \mathfrak{p}.$

综上,当$n$与$\mathfrak{p}$互素时,对任意$1\leq a<n$,总有$1-\xi_{n}^{a}\notin \mathfrak{p}$成立,从而$n$次单位根模$\mathfrak{p}$两两不同余
\end{proof}
因此$n| |(\mathcal{O}_{k}/\mathfrak{p})^{*}|=q-1.$于是$\alpha^{\frac{q-1}{n}}$是$(\mathcal{O}_{k}/\mathfrak{p})^{*}$中一个$n$次单位根:即存在唯一($(\mathcal{O}_{k}/\mathfrak{p})^{*}$是循环群,$n$次单位根模$\mathfrak{p}$两两不同余)的$n$次单位根,记为$(\frac{\alpha}{\mathfrak{p}})$,使得$$
\alpha^{\frac{q-1}{n}}\equiv \left(\frac{\alpha}{\mathfrak{p}}\right)\ mod \ \mathfrak{p}.
$$
称符号$(\frac{\alpha}{\mathfrak{p}})$为$k$中$n-$次剩余符号。回顾$\mathbb{Z}$上二次剩余符号,当时我们称上述同余式为\emph{欧拉判别法}。
上述符号中可将$\mathfrak{p}$替换称任意与$n$互素的理想$\mathfrak{a}$:设$\mathfrak{a}=\prod{\mathfrak{p}}$为素理想分解,规定
$(\alpha/\mathfrak{a})=\prod(\alpha/\mathfrak{p})$。
像整数环中二次剩余那样,对一般的$n$次剩余,有下面类似的性质
\begin{prop}
	(\cite{lemmer}Proposition 4.1)设$k$是包含$n$次本原单位根$\xi_{n}$的代数数域,$\mathfrak{p}$是$\mathcal{O}_{k}$的素理想,则有:
	\begin{itemize}
	\item[i)]若$\alpha\equiv\beta\ mod \ \mathfrak{p}$,则$(\frac{\alpha}{\mathfrak{p}})=\left(\frac{\beta}{\mathfrak{p}}\right)$;
  \item[ii)]$\frac{\alpha}{\mathfrak{p}}=1$当且仅当$\alpha$是模$\mathfrak{p}\ $n次剩余,即当且仅当存在$\xi\in \mathcal{O}_{k}-\mathfrak{p}$使得$\alpha\equiv \xi^{n}\ mod \ \mathfrak{p}$;
\item[iii)]设素数$p\equiv 1\ mod \ m$,$a\in\mathbb{Z}$是模$p\ $m-次剩余当且仅当对于$\mathbb{Q}(\xi_{m})$中任何在$p$上的素理想$\mathfrak{p}$都有$(\frac{a}{\mathfrak{p}})$.
\end{itemize}
\end{prop}
下面命题是考虑域扩张之间的剩余符号的关系。
\begin{prop}
	(\cite{lemmer}Prop4.2)设$K/F$是正规扩张,Galois群为$G$,$k/F$是其子扩张,且$\xi_{n}\in k$.则
	\begin{itemize}
		\item[i)] 对于每个$\sigma\in G$,任意的$\alpha\in K^{*},$和与$n$互素的理想$\mathfrak{a}$,我们有$\left(\frac{\alpha}{\mathfrak{a}}\right)^{\sigma}_{K}=\left(\frac{\alpha^{\sigma}}{\mathfrak{a}^{\sigma}}\right)_{K}$.
		\item[ii)]若$\mathfrak{p}$在域扩张$K/k$上一个素理想为$\mathfrak{P}$,且$f(\mathfrak{P}/\mathfrak{p})=1$,则对任意
		$\alpha\in \mathcal{O}_{k}$,有$\left(\frac{\alpha}{\mathfrak{P}}\right)_{K}=\left(\frac{\alpha}{\mathfrak{p}}\right)_{k}$.
		\item[iii)] 若$K|k$是Abel扩张,$\mathfrak{p}$是$\mathcal{O}_{K}$中一个素理想,$N=N_{K|k}$表示相对范数,则对任意$\alpha\in \mathcal{O}_{k}$,有
		$\left(\frac{\alpha}{\mathfrak{p}}\right)_{K}=\left(\frac{N\alpha}{\mathfrak{p}}\right)_{k}$.
		\item[iv)]设$K|k$是次数为$n$的循环扩张,设存在$\mathcal{O}_{k}$中素理想$\mathfrak{p}$使得$\mathfrak{p}\nmid n\mathcal{O}_{k}$且$\mathfrak{p}$在$K|k$扩张下是完全分歧的,即$\mathfrak{p}\mathcal{O}_{K}=\mathfrak{P}^{n}$。则
		$\left(\frac{N_{K|k}\alpha}{\mathfrak{p}}\right)_{k}=1$对任意$\alpha\in \mathcal{O}_{K}-\mathfrak{P}$成立。
	\end{itemize}
	
\end{prop}
\begin{proof}
	ii)设$\mathfrak{P}$是$\mathcal{O}_{K}$中一个在$\mathfrak{p}$上的素理想,$f(\mathfrak{P}/\mathfrak{p})=1$,则$N(\mathfrak{P})=N(\mathfrak{p})=q.$因此
	$$
	\left(\frac{\alpha}{\mathfrak{P}}\right)_{K}\equiv \alpha^{\frac{q-1}{n}}\ mod \ \mathfrak{P}\ \ and \ \ \left(\frac{\alpha}{\mathfrak{p}}\right)_{k}\equiv \alpha^{\frac{q-1}{n}}\ mod \ \mathfrak{p}.
	$$
	由此有$$\left(\frac{\alpha}{\mathfrak{P}}\right)_{K}\equiv \left(\frac{\alpha}{\mathfrak{p}}\right)_{k}\ mod \ \mathfrak{P}.$$
	前面已说明若$n$与$\mathfrak{P}$互素,则$n$次单位根模$\mathfrak{P}$两两不同余,从而由上面同余式知$\left(\frac{\alpha}{\mathfrak{P}}\right)_{K}= \left(\frac{\alpha}{\mathfrak{p}}\right)_{k}$.
\end{proof}

数域中$n$次剩余符号也可由$Artin$符号给出。

首先回顾赋值扩张的一些结论(见\cite{Ne} Chapter II,section 8):设$K|k$是数域的扩张,$\mathfrak{p}$表示$\mathcal{O}_{k}$中任意一个素理想,设$\mathfrak{p}$在$k$上决定的赋值为$\nu$,$k_{\nu}$是$k$关于$\nu$的完备化。则$v$到$K$上的赋值延拓完全由
$\mathfrak{p}$上$\mathcal{O}_{K}$中的素理想决定。事实上,设$\mathfrak{p}\mathcal{O}_{K}=\mathfrak{P}_{1}^{e_{1}}\cdots\mathfrak{P}_{g}^{e_{g}}$,则$v$的所有延拓恰为
$\omega_{i}=\frac{1}{e_{i}}v_{\mathfrak{P}_{i}},i=1\cdots,g.$
且$[K_{\omega_{i}}:k_{\nu}]=e_{i}f_{i}$,其中$e_{i}=e(\mathfrak{P}_{i}|\mathfrak{p})=e(\mathfrak{P}_{\omega_{i}}|\mathfrak{p}_{v}),$ $f_{i}=f(\mathfrak{P}_{i}/\mathfrak{p})=f(\mathfrak{P}_{\omega_{i}}|\mathfrak{p}_{v})$。其中$\mathfrak{P}_{\omega_{i}},\mathfrak{p}_{v}$分别为局部域$K_{\omega_{i}},k_{v}$中唯一的素理想。若$K|k$是Galois扩张,则$k$中任意素理想$\mathfrak{p}$的分歧指数$e(\mathfrak{P}_{i}|\mathfrak{p})$相同。若想说明$\mathfrak{p}$在域扩张$K|k$下是非分歧的,只需证明任一局部域扩张$K_{\omega_{i}}|k_{v}$是非分歧扩张。\\


设$k$是包含$n-$次本原单位根$\zeta_{n}$的数域。$\mathfrak{p}$表示$\mathcal{O}_{k}$中任意一个素理想,设$\mathfrak{p}$在$k$上决定的赋值为$\nu$,
$k_{\nu}$是$k$关于$\nu$的完备化。$\alpha\in k^{*}$,令$K=k(\sqrt[n]{\alpha})$,
用$S$表示$k$的所有阿基米德素除子和$k$中所有整除$n$的非阿基米德素除子组成的集合。若$a_{1},\cdots,a_{r}\in k*$,用$S(a_{1},\cdots,a_{r})$表示$S$与
$$S'(a_{1},\cdots,a_{r})=\{\ v\ |\text{存在某个指标i,}1\leq i\leq r,v(a_{i})\neq 0,v\text{是k的素除子}\}$$
的并集。对于$a\in k^{*}$,用$I^{S(a)}$表示$k$的所有与$S(a)$中有限素除子(即素理想)互素的$k$的分式理想的全体。显然若$\mathfrak{p}\in I^{S(a)}$,则$\mathfrak{p}\nmid n\alpha.$


下面就取定这样一个素理想$\mathfrak{p}$,则
$\mathfrak{p}$在$K|k$上非分歧:
只需说明
$k_{\nu}(\sqrt[n]{\alpha})|k_{\nu}$是非分歧扩张,由于$\mathfrak{p}\nmid \alpha,$ $\nu(\alpha)=0,$从而$\alpha$是$\mathcal{O}_{k_{\nu}}$中单位,又因$\mathfrak{p}\nmid n$,设$N_{k|\mathbb{Q}}\mathfrak{p}=p^{f}$,则$(n,p)=1$,即 $n$与$\mathcal{O}_{k_{\nu}}$的剩余类域特征互素,由\cite{Ne}中
ChapterV,section3中lemma 3.3知$k_{\nu}(\sqrt[n]{\alpha})|k_{\nu}$是非分歧扩张。

对于满足$\mathfrak{p}\nmid n\alpha$的素理想$\mathfrak{p}$,我们断言
$$
\left(\frac{K|k}{\mathfrak{p}}\right)\sqrt[n]{\alpha}=\left(\frac{\alpha}{\mathfrak{p}}\right)_{k}\sqrt[n]{\alpha}.
$$
\begin{proof}

取$K$中一个在$\mathfrak{p}$上的素理想$\mathfrak{P}$,$K|k$是循环扩张,由Artin符号定义
$$
\left(\frac{K|k}{\mathfrak{p}}\right)\sqrt[n]{\alpha}\equiv (\sqrt[n]{\alpha})^{N\mathfrak{p}}\ mod \ \mathfrak{P},
$$
而右侧$\left(\frac{\alpha}{\mathfrak{p}}\right)_{k}\sqrt[n]{\alpha}\equiv \alpha^{(N\mathfrak{p}-1)/n}\sqrt[n]{\alpha} \ mod \ \mathfrak{P}$.由此
$$
\left(\frac{K|k}{\mathfrak{p}}\right)\sqrt[n]{\alpha}\equiv\left(\frac{\alpha}{\mathfrak{p}}\right)_{k}\sqrt[n]{\alpha}\ mod \ \mathfrak{P}.
$$
设$\left(\frac{K|k}{\mathfrak{p}}\right)\sqrt[n]{\alpha}=\zeta \sqrt[n]{\alpha},$其中$\zeta$是某个n次单位根。$\mathfrak{p}\nmid \alpha\Rightarrow\mathfrak{P}\nmid \sqrt[n]{\alpha}$,故上一同余式可导出,
$$
\zeta\equiv \left(\frac{\alpha}{\mathfrak{p}}\right)_{k}\ mod \ \mathfrak{P}.
$$
类似前文,该同余实为相等.由此得到断言成立。
\end{proof}
\subsection{例}
\begin{exa}
	$\mathbb{Z}$中二次剩余符号.默认已被读者熟悉,这里不再赘述。
\end{exa}
\begin{exa}
	Gauss整数环$\mathbb{Z}[i]$中二次剩余.$2$在域扩张$\mathbb{Q}(i)|\mathbb{Q}$下是分歧的,$2=(1+i)(1-i)=i^{3}(1+i)^{2}$,其中$1+i$是$\mathbb{Z}[i]$中素元。设$\pi=a+bi,\lambda=c+di$是两个不同的素元,且$\pi\equiv \lambda\equiv 1\ mod \ 2$(从而与$1+i$互素)。用$\left[\frac{\cdot}{\cdot}\right]$表示$\mathbb{Z}[i]$中二次剩余符号,则有
	$$\left[\frac{\pi}{\lambda}\right]=\left[\frac{\lambda}{\pi}\right].$$
	作为补充律,有
	$$\left[\frac{i}{a+bi}\right]=(-1)^{b/2},\left[\frac{1+i}{a+bi}\right]=\left(\frac{2}{a+b}\right).$$
	其中$\left(\frac{\cdot}{\cdot}\right)$表示$\mathbb{Z}$中二次剩余符号(证明见\cite{lemmer}Chapter5,Proposition5.1).
\end{exa}
\begin{exa}
	$\mathbb{Z}$中四次剩余.用$\left(\frac{\cdot}{\cdot}\right)_{4}$表示$\mathbb{Z}$中四次剩余符号。对于素数$p\equiv 3\ mod \ 4$,因$\left(\frac{-1}{p}\right)=-1$,易证得对于与$p$互素的整数$a\in\mathbb{Z}$
	$$
	\left(\frac{a}{p}\right)_{4}=1\Leftrightarrow \left(\frac{a}{p}\right)=1.
	$$
	上面等价号右侧表示二次剩余,左侧表示四次剩余。故四次剩余一般是研究$\equiv 1\ mod \ 4$的素数$p$。注意到若$\left(\frac{a}{p}\right)=1$,则$\left(\frac{a}{p}\right)_{4}$取值为$\pm 1$.下设素数$p\equiv 1 \ mod \ 4$.
	关于$\mathbb{Z}$上的四次剩余符号$\left(\frac{\cdot}{p}\right)_{4}$,有如下一些结果:
	\begin{itemize}
		\item (\cite{lemmer}Proposition5.3)设$p\equiv 1\mod 4$是素数,令$i$是满足$i\equiv b/a\ mod\ p$的整数。则$$
	\left(\frac{2}{p}\right)_{4}\equiv i^{\frac{ab}{2}}\ (mod\ p).
		$$
		\item (\cite{lemmer}Propositio 5.4)令$p=a^{2}+b^{2}=c^{2}+2d^{2}=e^{2}-2f^{2}=8n+1$是素数,假设$b$是偶数。则
		$$
		\left(\frac{2}{p}\right)_{4}=(-1)^{b/4}=\left(\frac{2}{c}\right)=(-1)^{n+d/2}=\left(\frac{-2}{e}\right).
		$$
		\item (\cite{lemmer}Propositio 5.5)设$p=a^{2}+b^{2},q$是两个素数,且$\left(\frac{q}{p}\right)=1$,假设$2|b$.令$\sigma$是同余式$\sigma^{2}\equiv p \ mod \ q$的一个解,则
		$$
		\left(\frac{q^{*}}{p}\right)_{4}=\left(\frac{\sigma(b+\sigma)}{q}\right).
		$$
		这里$q^{*}=(-1)^{\frac{q-1}{2}}q.$
		\item 设$p=a^{2}+b^{2}$,$q=c^{2}+d^{2}$是素数,其中$b,d$是偶数且$\left(\frac{p}{q}\right)=1$,则
		$$
		\left(\frac{q}{p}\right)_{4}\equiv \left(\frac{a/b-c/d}{a/b+c/d}\right)^{\frac{q-1}{4}} \ mod \ q.
		$$
		另有几个等价公式,请看\cite{lemmer}第五章。
	\end{itemize}

	
\end{exa}
\begin{exa}
	Gauss整数环中四次剩余。称$\mathbb{Z}[i]$中满足$\alpha\equiv 1 \ mod \ (1+i)^{3}$的非单位元为\emph{准素}元.用$\left[\frac{\cdot}{\cdot}\right]_{4}$表示$\mathbb{Z}[i]$中四次剩余符号($\mathbb{Z}[i]$中四次剩余符号在不同文献中写法不同,例如在\cite{lemmer}中第6.3节,仍用$\left[\frac{\cdot}{\cdot}\right]$表示$\mathbb{Z}[i]$中四次剩余符号。请读者注意区分),则有性质
	\begin{itemize}
		\item 设$\pi$是$\mathbb{Z}[i]$中奇素数(即$N(\pi)$是$\mathbb{Z}$中奇素数),则$$\left[\frac{\alpha\beta}{\pi}\right]_{4}=\left[\frac{\alpha}{\pi}\right]_{4}\left[\frac{\beta}{\pi}\right]_{4},\forall \alpha,\beta\in \mathbb{Z}[i].$$
		$$ 
		\left[\frac{\overline{\alpha}}{\overline{\pi}}\right]_{4}=\overline{\left[\frac{\alpha}{\pi}\right]}_{4}=\left[\frac{\alpha}{\pi}\right]_{4}^{-1},
		$$
		这里上划线表示复共轭。
	\item
	互反律:
	如果$\pi$和$\lambda$是$\mathbb{Z}[i]$中两个不同的准素素数,则有
	$$
	\left[\frac{\lambda}{\pi}\right]_{4}=\left[\frac{\pi}{\lambda}\right]_{4}(-1)^{(N(\lambda)-1)(N(\pi)-1)/16}.
	$$
	作为补充律有:
	$$
	\left[\frac{i}{\pi}\right]_{4}=i^{-(a-1)/2}
	$$
	$$
	\left[\frac{1+i}{\pi}\right]_{4}=i^{(a-b-1-b^{2})/4}.
	$$
	这里$a,b$满足$\pi=a+bi$(证明见\cite{lemmer}6.3节或\cite{rosen}9.9节).
\end{itemize}
	关于进一步的性质,请看\cite{lemmer}.
	\end{exa}
	我们用一个例子说明$\mathbb{Z}$中四次剩余符号和$\mathbb{Z}[i]$中四次剩余符号是不同的。在$\mathbb{Z}$中,用欧拉判别法
$$
\left(\frac{2}{17}\right)_{4}\equiv 2^{\frac{17-1}{2}}=16 \ mod\ 17
$$
故$\left(\frac{2}{17}\right)_{4}=-1$.
而在$\mathbb{Z}[i]$中
$17=(1+4i)(1-4i)$是准素分解($p\equiv 1\ mod \ 4$的素数都有准素分解),故
$$
\left[\frac{2}{17}\right]_{4}=\left[\frac{2}{1+4i}\right]_{4}\left[\frac{2}{1-4i}\right]_{4}=\left[\frac{2}{1+4i}\right]_{4}\left[\frac{2}{1+4i}\right]_{4}^{-1}=1.
$$
\section{Hilbert符号}
该部分的定理及命题主要参考\cite{Ne}.
 \subsection{定义}
 下设$K$是局部域,或$K=\mathbb{R},K=\mathbb{C}.$ 假设$K$中包含$n$次单位根群$\mu_{n}$,这里$n$与$K$的特征互素(若
 $char(K)=0$,对$n$没有要求).
  是由Kummer理论$L=K(\sqrt[n]{K^*})$是$K$的指数为$n$的极大Abel扩张。且可证得(\cite{Ne}Chapter V,Proposition1.5)
  $$N_{L|K}L^*=K^{*n},$$
 由此,局部类域论给出同构
 $$ K^*/K^{*n}\cong G(L|K),\eqno{(1)}$$
 同构由$a\mapsto (a,L|K)$给出。
 这里我们简写$Gal(L|K)$为$G(L|K)$,下文亦是如此。
 
 
 另一方面,通过Kummer配对,我们有同构
 $$
 K^*/K^{*n}\cong Hom(G(L|K),\mu_{n}),\eqno{(2)}
 $$
 同构由$$a\mapsto \varphi_{a}:\chi \mapsto \frac{\chi(\sqrt[n]{a})}{\sqrt[n]{a}}$$给出。
 
 我们想把上面两个同构联系起来。为此考虑双线性映射
 $$
 G(L|K)\times Hom(G(L|K),\mu_{n})\rightarrow \mu_{n},\ \ (\sigma,\chi)\mapsto\chi(\sigma),\eqno{(3)}
 $$
 
 该配对是非退化的:若对任意$\sigma\in G(L|K),\chi(\sigma)=1$,则自然有$\chi=1$.反之,若对任意
 $$\chi\in Hom(G(L|K),\mu_{n}),\ \ \chi(\sigma)=1.$$
 若$\sigma\neq 1,$记$\sigma$生成的子群为$<\sigma>$,则$\chi\in Hom(G(L|K)/<\sigma>,\mu_{n}),$考虑阶数$$|G(L|K)|=|Hom(G(L|K),\mu_{n})|<|Hom(G(L|K)/<\sigma>,\mu_{n})|=|G(L|K)/<\sigma>|,$$显然这是不可能的,矛盾!故$\sigma=1$.
 
 现在我们利用上面(1),(2)和(3),便可得到一个双线性配对,
 $$
 \left(\frac{\ ,\ }{\mathfrak{p}}\right):K^*/K^{*n}\times K^*/K^{*n}\longrightarrow \mu_{n},\eqno{(*)}
 $$
 这里$\mathfrak{p}$是局部域$K$的唯一的素理想,上述映射元素对应为
 $$
 \left(\frac{a,b}{\mathfrak{p}}\right)=\frac{(a,L|K)\sqrt[n]{b}}{\sqrt[n]{b}}.
 $$
 
 
 这里注意到,我们取定$(*)$式中的第一个$K^*/K^{*n}\cong G(L|K),$而第二个$K^*/K^{*n}$是同构于\\
 $Hom(G(L|K),\mu_{n})$(一些文献可能会恰好相反)。上述符号$\left(\frac{\ ,\ }{\mathfrak{p}}\right)$称为\emph{Hilbert符号}.
 关于Hilbert符号,有如下性质.
 \begin{prop}
 	(\cite{Ne}Chapter V,Proposition 3.1)对于$a,b\in K^*$,Hilbrt符号$\left(\frac{a,b }{\mathfrak{p}}\right)\in \mu_{n}$满足
 	$$
 	(a,K(\sqrt[n]{b})|K)\sqrt[n]{b}=\left(\frac{a,b}{\mathfrak{p}}\right)\sqrt[n]{b}.
 	$$
 \end{prop}
\begin{prop}	记号如上,
	\begin{itemize}
		\item[(i)]$\left(\frac{aa',b}{\mathfrak{p}}\right)=\left(\frac{a,b}{\mathfrak{p}}\right)\left(\frac{a',b}{\mathfrak{p}}\right)$,
		\item[(ii)] $\left(\frac{a,bb'}{\mathfrak{p}}\right)=\left(\frac{a,b}{\mathfrak{p}}\right)\left(\frac{a,b'}{\mathfrak{p}}\right),$
		\item[(iii)] $\left(\frac{a,b}{\mathfrak{p}}\right)=1\Leftrightarrow a$是域扩张$K(\sqrt[n]{b})|K$的一个范,
		\item[(iv)] $\left(\frac{a,b}{\mathfrak{p}}\right)=\left(\frac{b,a }{\mathfrak{p}}\right)^{-1},$
		\item[(v)] $\left(\frac{a,1-a}{\mathfrak{p}}\right)=\left(\frac{a,-1}{\mathfrak{p}}\right)=1,$
		\item[(vi)]若对任意$b\in K^*$,有$\left(\frac{a,b}{\mathfrak{p}}\right)=1$,则$a\in K^{*n}$.
	\end{itemize}
\end{prop}

\subsection{例}
下面给出Hilbert符号的一个应用。
\begin{exa}
	令$F=\mathbb{Q}(\sqrt{2},i),L=\mathbb{Q}(\sqrt{2},\sqrt{p_{1}p_{2}},i)$,其中$p_{1},p_{2}$是奇素数,且$p_{1}\equiv p_{2}\equiv 5\ mod \ 8$.判断$F$的基本单位$\varepsilon_{2}=1+\sqrt{2}$和$\sqrt{i}$是否是$L$中元素的范.这里$i^{2}=-1$.
\end{exa}
  注意到$L|F$是二次扩张,从而是循环扩张。对于循环扩张,我们有Hesse Norm定理(\cite{Ne}Chapter VI,Corollary4.5)
  \begin{thm}
  	(Hesse Norm Theorem)设$L|K$是循环扩张,则$x\in K^*$是$L$中元素的范当且仅当它是每一个局部域扩张
  	$L_{\mathfrak{P}}|K_{\mathfrak{p}}(\mathfrak{P}|\mathfrak{p})$的一个元素的范.
  \end{thm}
 回到例子中,我们只需判断$\varepsilon_{2},\sqrt{i}$是否是局部范.
 注意到$L=F(\sqrt{p_{1}p_{2}})$,由上面第二个命题的(iii),我们需要对$\mathcal{O}_{F}$中每个素理想$\mathfrak{p}$,在$F$关于$\mathfrak{p}$的完备化$F_{\mathfrak{p}}$中计算
 $\left(\frac{\varepsilon_{2},p_{1}p_{2}}{\mathfrak{p}}\right),\left(\frac{\sqrt{i},p_{1}p_{2}}{\mathfrak{p}}\right).$
 
 为此,设$\mathfrak{p}$是$F$中不在$2$(2在$F$上惯性)上的素理想,$\left(\frac{\ , \ }{\mathfrak{p}}\right)$表示$2-$次Hiilbert符号,即定义中的$n$
 取2.
 分下面几种情形进行计算:
 \begin{itemize}
 	\item[(1)] 若$\mathfrak{p}$不在$p_{1},p_{2}$上,则$v_{\mathfrak{p}}(\varepsilon_{2})=v_{\mathfrak{p}}(p_{1}p_{2})=v_{\mathfrak{p}}(\sqrt{i})=0$,从而$
 	\left(\frac{p_{1}p_{2},\varepsilon_{2}}{\mathfrak{p}}\right)=\left(\frac{p_{1}p_{2},\sqrt{i}}{\mathfrak{p}}\right)=1.$
 	\item[(2)] 若$\mathfrak{p}$为$p_{1}$上素理想,则$v_{\mathfrak{p}}(\varepsilon_{2})=v_{\mathfrak{p}}(\sqrt{i})=0$,$F|\mathbb{Q}$有三个中间域,$p_{1}$在上面均非分歧,从而$p_{1}$在
 	$F|\mathbb{Q}$上非分歧,故$v_{\mathfrak{p}}(p_{1}p_{2})=v_{\mathfrak{p}}(p_{1})=1.$因2-次Hilbert符号只取值$\pm 1,$Hilbert符号是关于两个变量$a,b$对称的,为简记,对任意域$E,$用$\widehat{E}$表示$E(\sqrt{E^{*}})$.
 	下面推理需用下述命题
 	\begin{prop}
 			(\cite{Ne}Chapter IV,Proposition6.4 )若$L|K,L'|K'$是有限Galois扩张,$K\subseteq K',L\subseteq L',$令$\sigma\in G$,则有下述交换图
 		$$
 		\xymatrix{
 			{K'}^{*}\ar[rr]^{(\ ,L'|K')}\ar[d]_{N_{K'|K}}& &G(L'|K')^{ab}\ar[d]^{res}\\
 			K^{*}\ar[rr]^{(\ ,L|K)}&&G(L|K)^{ab}	
 		}
 		$$
 		这里$res$表示限制映射.
 	\end{prop}
 下面开始计算Hilbert符号.
 	$$
 	\left(\frac{p_{1}p_{2},\varepsilon_{2}}{\mathfrak{p}_{F}}\right)=	\left(\frac{p_{1},\varepsilon_{2}}{\mathfrak{p}_{F}}\right)=\left(\frac{\varepsilon_{2},p_{1}}{\mathfrak{p}_{F}}\right)=\frac{(\varepsilon_{2},\widehat{F_{\mathfrak{p}}}|F_{\mathfrak{p}})\sqrt{p_{1}}}{\sqrt{p_{1}}}
 	$$
 	$$=
 	\frac{(N_{F_{\mathfrak{p}}|\mathbb{Q}_{\mathfrak{p}}(\sqrt{2})}(\varepsilon_{2}),\widehat{\mathbb{Q}_{\mathfrak{p}}(\sqrt{2})}|\mathbb{Q}_{\mathfrak{p}}(\sqrt{2}))\sqrt{p_{1}}}{\sqrt{p_{1}}}
 	$$
 	$p_{1}$在$\mathbb{Q}(\sqrt{2})$中是惯性的,这里$\mathbb{Q}_{\mathfrak{p}}(\sqrt{2})$表示$\mathbb{Q}(\sqrt{2})$关于$p_{1}$上唯一素理想$\mathfrak{p}_{\mathbb{Q}(\sqrt{2})}$的完备化。因此$$[\mathbb{Q}_{\mathfrak{p}}(\sqrt{2}):\mathbb{Q}_{\mathfrak{p}}]=e(\mathfrak{p}_{\mathbb{Q}(\sqrt{2})}/p_{1})f(\mathfrak{p}_{\mathbb{Q}(\sqrt{2})}/p_{1})=1,$$
 	故$F_{\mathfrak{p}}=\mathbb{Q}_{\mathfrak{p}}(\sqrt{2},i)=\mathbb{Q}_{\mathfrak{p}}(\sqrt{2}).
 	$从而上式
 	$$
 	=\frac{(\varepsilon_{2},\widehat{\mathbb{Q}_{\mathfrak{p}}(\sqrt{2})}|\mathbb{Q}_{\mathfrak{p}}(\sqrt{2}))\sqrt{p_{1}}}{\sqrt{p_{1}}}=\frac{(-1,\widehat{\mathbb{Q}_{p_{1}}}|\mathbb{Q}_{p_{1}})\sqrt{p_{1}}}{\sqrt{p_{1}}}=\left(\frac{p_{1},-1}{p_{1}}\right)
 	$$
 	注意到Hilbert符号在一定条件下转化为n-次剩余符号:
 	一般地,设局部域$K$的剩余类域特征为$p$,且$(p,n)=1$,这里$n$是指最开始定义中$K$中包含n次单位根群。因$U_{K}=\mu_{q-1}\times U^{(1)}_{K}$,每个单位$u\in U_{K}$存在唯一分解
 	$$
 	u=\omega(u)<u>
 	$$
 	其中$\omega(u)\in \mu_{q-1},<u>\in U^{(1)}_{K}$.设$\pi$是$K$的任意一致化子,即$(\pi)=\mathfrak{p}$,则n-次Hilbert符号(
 	\cite{Ne}Chapter V,Proposition 3.4)
 	$$
 	\left(\frac{\pi,u}{\mathfrak{p}}\right)=\omega(u)^{(q-1)/n}, \eqno{(4)}
 	$$
 	这里$q$是$K$的剩余类域中的元素个数。简记
 	$$
 		\left(\frac{u}{\mathfrak{p}}\right):=	\left(\frac{\pi,u}{\mathfrak{p}}\right) \ for\ \ u\in U_{K}.
 	$$
 	
 	且有如下命题
 	\begin{prop}
 		(\cite{Ne}Chapter V,Proposition 3.5)若$(n,p)=1$且$u\in U_{K}$.则有
 		$$
 		\left(\frac{u}{\mathfrak{p}}\right)=1 \Leftrightarrow u \text{是}mod\ \mathfrak{p}\text{的n次幂元.}
 		$$
 	\end{prop}
 由此,在$\mathbb{Q}_{p_{1}}$中
 \begin{displaymath}
 \left(\frac{p_{1},-1}{p_{1}}\right)= \left\{ \begin{array}{ll}
 	1 & \textrm{-1 是$\mathbb{Z}_{\mathfrak{p}}/p\mathbb{Z}_{\mathfrak{p}}$中平方元}\\
 	-1 & \textrm{-1 不是$\mathbb{Z}_{\mathfrak{p}}/p\mathbb{Z}_{\mathfrak{p}}$中平方元}\\
 \end{array} \right.
\end{displaymath}
 因$\mathbb{Z}_{\mathfrak{p}}/p\mathbb{Z}_{\mathfrak{p}}\cong \mathbb{Z}/p\mathbb{Z}$上述恰为二次剩余的性质。
 故最终有
 $$
 \left(\frac{p_{1}p_{2},\varepsilon_{2}}{\mathfrak{p}_{F}}\right)=\left(\frac{-1}{p_{1}}\right)=1(p_{1}\equiv 5\ mod \ 8).
 $$
 同样地,
 $$
 \left(\frac{p_{1}p_{2},\sqrt{i}}{\mathfrak{p}_{F}}\right)=
 \left(\frac{p_{1},\sqrt{i}}{\mathfrak{p}_{F}}\right)
  =\left(\frac{\sqrt{i},p_{1}}{\mathfrak{p}_{F}}\right)
  =\frac{(\sqrt{i},\widehat{F_{\mathfrak{p}}}|F_{\mathfrak{p}})\sqrt{p_{1}}}{\sqrt{p_{1}}}$$
  $$
  =\frac{(N_{F_{\mathfrak{p}}|\mathbb{Q}_{\mathfrak{p}}(i)}(\sqrt{i}),\widehat{\mathbb{Q}_{\mathfrak{p}}(i)}|\mathbb{Q}_{\mathfrak{p}}(i))\sqrt{p_{1}}}{\sqrt{p_{1}}},
  $$
 因$\sqrt{i}=\frac{1}{\sqrt{2}}(1+i),$故$N_{F_{\mathfrak{p}}|\mathbb{Q}_{\mathfrak{p}}}(i)=\frac{1}{\sqrt{2}}(1+i)\frac{-1}{\sqrt{2}}(1+i)=-i,$而$-1=i^{2}\in(\mathbb{Q}_{\mathfrak{p}}(i))^{2}$,于是上面等式
 $$
 =\frac{(i^{2}\cdot i,\widehat{\mathbb{Q}_{\mathfrak{p}}(i)}|\mathbb{Q}_{\mathfrak{p}}(i))\sqrt{p_{1}}}{\sqrt{p_{1}}}
 =\frac{( i,\widehat{\mathbb{Q}_{\mathfrak{p}}(i)}|\mathbb{Q}_{\mathfrak{p}}(i))\sqrt{p_{1}}}{\sqrt{p_{1}}}$$
 $$
 =\left(\frac{p_{1},i}{p_{1}}\right)=i^{\frac{p_{1}-1}{2}}=-1(p_{1}\equiv 5\ mod \ 8).
 $$
 上面倒数第二个等号是利用(4)式。
 	\item[(3)] 若$\mathfrak{p}$是$p_{2}$上素理想,则类似于$(2)$,有$v_{\mathfrak{p}}(\varepsilon_{2})=v_{\mathfrak{p}}(p_{1}p_{2})=1,$因此$\left(\frac{p_{1}p_{2},\varepsilon_{2}}{\mathfrak{p}}\right)=\left(\frac{-1}{p_{2}}\right)=1(p_{2}\equiv 5\ mod \ 8).$
 \end{itemize}
 我们尚未考虑$F$中在$2$上的素理想,但由\emph{Hilbert乘积公式}:
 对于任意$a,b\in F^*$,有
 $$
 \prod_{\mathfrak{p}}\left(\frac{a,b}{\mathfrak{p}}\right)=1.
 $$
 故对于$F$上的每个素理想$\mathfrak{p}$,均有$\left(\frac{p_{1}p_{2},\varepsilon_{2}}{\mathfrak{p}}\right)=1.$
 从而由Haees Norm Theorem ,$\varepsilon_{2}\in N_{L|K}(L^*)$,因在(2)的计算中有$\left(\frac{p_{1}p_{2},\sqrt{i}}{\mathfrak{p}}\right)=-1$,故有$\sqrt{i}\notin N_{L|K}(L^*)$.至此,我们完成了判断.
 
 
 \begin{thebibliography}{99}
		\bibitem{Ne} Neukirch.\emph{Algebraic Number Theory}.
		\bibitem{wwl} 李文威.\emph{代数学方法}(第一卷).Vol,67.1.现代数学基础丛书.北京:高等教育出版社,2019.
		\bibitem{xkz}张贤科.\emph{代数数论导引}.
		\bibitem{lang}Serge lang.\emph{Algebra}.
		\bibitem{fkq}冯克勤.\emph{代数数论}.哈尔滨:哈尔滨工业大学出版社,2018.
		\bibitem{lemmer} Franz Lemmermeyer.\emph{Reciprocity Laws:From Euler to Eisenstein}.Springer Monographs in Mathematics.Springer,Berlin,2000.
  \bibitem{rosen}Rosen.\emph{A classical Introduction to Modern Number Theory.} Graduate Texts in Mathematica;84. 
	\end{thebibliography}
\end{document}